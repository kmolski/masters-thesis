\chapter{Podsumowanie} \label{ch:podsumowanie}

Głównym celem pracy była budowa klastra \english{Big Data} składającego się z~minikomputerów
wspierających akcelerację GPU, a~następnie badanie jego wydajności dla rozproszonych programów na
platformy Apache Hadoop oraz Apache Spark.

Mając na uwadze potencjalną wydajność i~koszt dostępnych minikomputerów jednopłytkowych, do realizacji
klastra wybrano model Jetson Nano w~wersji z~4 GB pamięci RAM. Zbudowano klaster składający się z~trzech
takich minikomputerów, pełniących funkcję węzłów wykonawczych, a~także z~minikomputera Raspberry Pi
4B, którego zadaniem było zarządzanie całym klastrem.

Proces instalacji oprogramowania i~konfiguracji klastra został w~dużej mierze zautomatyzowany przy
użyciu programu Ansible, co pozwoliło na łatwą instalację platform Hadoop oraz Spark w~wielu
środowiskach. Opracowano też skrypty do automatycznego uruchamiania testów wydajności i~zbierania
ich wyników, co ułatwiło porównanie klastra minikomputerów Jetson Nano z~pojedynczym komputerem
stacjonarnym.

Następnie przygotowano zestaw programów testowych, który posłużył do przeprowadzenia testów wydajności
na zbudowanym klastrze. Każdy program testowy umożliwia wykonanie zarówno na procesorze głównym,
jak i~z~wykorzystaniem technologii CUDA, dzięki czemu określono przyrost wydajności wynikający
z~użycia akceleracji GPU. Co więcej, dla jednego z~programów testowych dostarczono też obsługę
standardu OpenCL.

Dla testowego zestawu programów wykonano pomiary czasu wykonania, których wyniki posłużyły do dalszej
analizy i~porównania wydajności klastra z~konkurencyjnymi rozwiązaniami. Testy wydajności zostały
przeprowadzone z~uwzględnieniem procesu kompilacji Just-In-Time dla języków Java oraz CUDA.
\newpage

\section{Wnioski}

W testach programu PiEstimation z~próbkowaniem 1 miliona punktów klaster minikomputerów Jetson Nano
nie osiągnął przyspieszenia względem klastra Raspberry Pi 4 opisanego w~artykule \cite{rpi-cluster-2}.
Oznaczało to, że klaster komputerów Jetson nie był też konkurencyjny pod względem stosunku wydajności
do ceny.

Charakterystyka wydajności dla implementacji w języku CUDA stawała się coraz lepsza wraz ze wzrostem
liczby próbkowanych punktów, jednak dla tych wartości jedynym punktem odniesienia był komputer
stacjonarny bazujący na procesorze AMD Ryzen.

Dla programu PiEstimation i~próbkowania 100 milionów punktów klaster Jetson Nano osiągnął wynik
porównywalny z~komputerem PC, a~przy próbkowaniu 1 miliarda punktów okazał się szybszy o~ponad 36\%.
Z~drugiej strony, estymacja liczby $\pi$ była zadaniem, które w~najmniejszym stopniu przypominało
aplikacje do przetwarzania zbiorów \english{Big Data}.

Niestety test PiEstimation to jedyny przypadek, w~którym klaster oparty o~minikomputery jednopłytkowe
mógł równać się z~komputerem stacjonarnym. Testy programów FuzzyGen, FuzzyCompute oraz FuzzyFilter
wykazały, że wydajność klastra Jetson Nano nie jest konkurencyjna dla zadań rozproszonych
o~charakterystyce typowej dla obliczeń \english{Big Data}.

Za główny czynnik ograniczający wydajność programów uznano system plików HDFS, który przechowywał
dane na kartach microSD. W~kolejnych badaniach warto rozpatrzyć użycie innych nośników danych,
takich jak karty microSD o~wyższych klasach wydajności i~dyski twarde obsługujące interfejs USB 3.0.

W~przypadku programu FuzzyFilter interesującym kierunkiem badań jest dalsza optymalizacja kernela
CUDA oraz rozszerzenie funkcjonalności programu o~operacje łączeń, grupowania i~filtrowania z~użyciem
bardziej złożonych operacji logicznych. W tym celu pomocne może okazać się użycie platformy Apache
Hive wraz z~biblioteką FuzzyHive \cite{fuzzy-hive}, która dostarcza gotowych implementacji operacji
rozmytych dla dużych zbiorów danych. 

Kolejnym wartym rozważenia kierunkiem badań jest wpływ akceleracji GPU na aplikacje w~innych
dziedzinach. Minikomputery jednopłytkowe Jetson Nano posiadają wydajne procesory graficzne,
których zastosowanie może znacząco przyspieszyć aplikacje grafiki komputerowej i~uczenia maszynowego.

Programy testowe oraz skrypty administracyjne wykonane na potrzeby pracy stały się dobrą podstawą do
pomiarów i~analizy wydajności klastrów obliczeniowych \english{Big Data}. Szczególnie pomocne okazały
się skrypty instalacyjne Ansible, które znacznie przyspieszyły proces konfiguracji klastra w~wielu
środowiskach, co znacząco usprawniło proces badań.
