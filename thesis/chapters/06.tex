\chapter{Dokumentacja techniczna} \label{ch:techdoc}

\section*{Klasa QmcUtils}

Klasa QmcUtils (listing \ref{lst:qmc-utils}) dostarcza metod użytecznych w implementacjach metody Monte Carlo.

\begin{lstlisting}[
    language=Java,
    label=lst:qmc-utils,
    caption={Implementacja metod użytecznych w implementacjach metody Monte Carlo.}
]
public final class QmcUtils {

    private static final int[] BASES = {2, 3};
    private static final int DIMENSIONS = BASES.length;
    private static final int[] MAX_DIGITS = {63, 40};

    private QmcUtils() {}

    public static double[] getRandomPoint(long index) {
        double[] point = new double[DIMENSIONS];

        double[][] q = new double[DIMENSIONS][];
        int[][] d = new int[DIMENSIONS][];
        for (int i = 0; i < DIMENSIONS; i++) {
            q[i] = new double[MAX_DIGITS[i]];
            d[i] = new int[MAX_DIGITS[i]];
        }

        for (int i = 0; i < DIMENSIONS; i++) {
            long k = index;
            point[i] = 0;

            for (int j = 0; j < MAX_DIGITS[i]; j++) {
                q[i][j] = (j == 0 ? 1.0 : q[i][j - 1]) / BASES[i];
                d[i][j] = (int) (k % BASES[i]);
                k = (k - d[i][j]) / BASES[i];
                point[i] += d[i][j] * q[i][j];
            }

            for (int j = 0; j < MAX_DIGITS[i]; j++) {
                d[i][j]++;
                point[i] += q[i][j];
                if (d[i][j] < BASES[i]) {
                    break;
                }
                d[i][j] = 0;
                point[i] -= (j == 0 ? 1.0 : q[i][j - 1]);
            }
        }

        return point;
    }

    public static boolean isPointInside(double[] point) {
        final double x = point[0] - 0.5;
        final double y = point[1] - 0.5;

        return x * x + y * y <= 0.25;
    }
}
\end{lstlisting}
\newpage

\section*{Klasa FuzzyUtils}

Klasa FuzzyUtils (listing \ref{lst:fuzzy-utils}) implementuje operacje na zbiorach rozmytych na procesorze głównym.

\begin{lstlisting}[
    language=Java,
    label=lst:fuzzy-utils,
    caption={Implementacja operacji rozmytych na procesor główny.}
]
public final class FuzzyUtils {

    private static final int SET_SIZE = 4;
    public static final int SETS_IN_RECORD = 64;
    public static final int RECORD_SIZE = SET_SIZE * SETS_IN_RECORD;
    public static final int RECORD_BYTES = RECORD_SIZE * Float.BYTES;

    private FuzzyUtils() {}

    public static float[] fuzzyUnion(float[] target, float[] source) {
        int n = Math.min(target.length, source.length);
        for (int i = 0; i < n; ++i) {
            target[i] = Math.max(target[i], source[i]);
        }
        return target;
    }

    public static float[] fuzzyComplement(float[] set) {
        int n = set.length;
        float[] result = new float[n];
        for (int i = 0; i < n; ++i) {
            result[i] = 1.0f - set[i];
        }
        return result;
    }

    public static void generateRandomSets(long nRecords, Consumer<byte[]> consumer) {
        var random = new Random();
        for (int i = 0; i < nRecords; ++i) {
            var bytes = new byte[RECORD_BYTES];
            var buffer = ByteBuffer.wrap(bytes);
            for (int j = 0; j < RECORD_SIZE; ++j) {
                buffer.putFloat(random.nextFloat());
            }

            consumer.accept(bytes);
        }
    }

    public static void performOps(byte[] inputRecord) {
        var buffer = ByteBuffer.wrap(inputRecord);

        var sets = new float[SETS_IN_RECORD * 2][];
        for (int i = 0; i < SETS_IN_RECORD; ++i) {
            var set = sets[i] = new float[SET_SIZE];
            for (int j = 0; j < SET_SIZE; ++j) {
                set[j] = buffer.getFloat();
            }
        }
        for (int i = 0; i < SETS_IN_RECORD; ++i) {
            sets[i + SETS_IN_RECORD] = FuzzyUtils.fuzzyComplement(sets[i]);
        }

        var results = Arrays.stream(sets).map(float[]::clone).toArray(float[][]::new);
        for (int i = 0; i < sets.length; ++i) {
            for (int j = 0; j < sets.length; ++j) {
                if (i != j) {
                    results[i] = FuzzyUtils.fuzzyUnion(results[i], results[j]);
                }
            }
        }

        buffer.rewind();
        for (int i = 0; i < SETS_IN_RECORD; ++i) {
            for (int j = 0; j < SET_SIZE; ++j) {
                buffer.putFloat(results[i][j]);
            }
        }
    }

    public static void forEachRecord(CUdeviceptr inputRecords, long nRecords, Consumer<byte[]> consumer) {
        long byteCount = nRecords * RECORD_BYTES;
        var bytes = new byte[(int) byteCount];
        cuMemcpyDtoH(Pointer.to(bytes), inputRecords, byteCount);

        var buffer = ByteBuffer.wrap(bytes);
        for (int i = 0; i < nRecords; ++i) {
            var writableBuf = new byte[RECORD_BYTES];
            buffer.get(writableBuf, 0, RECORD_BYTES);
            consumer.accept(writableBuf);
        }
    }
}
\end{lstlisting}
\newpage

\section*{Klasa JcudaUtils}

Klasa JcudaUtils (listing \ref{lst:jcuda-utils}) zawiera metody pomocnicze do wywoływania programów na procesorze graficznym.

\begin{lstlisting}[
    language=Java,
    label=lst:jcuda-utils,
    caption={Metody pomocnicze do wywoływania programów na procesorze graficznym.}
]
public final class JcudaUtils {

    private JcudaUtils() {}

    public static CUcontext createCudaContext() {
        JCudaDriver.setExceptionsEnabled(true);
        JCurand.setExceptionsEnabled(true);
        cuInit(0);

        var device = new CUdevice();
        cuDeviceGet(device, 0);

        var ctx = new CUcontext();
        cuCtxCreate(ctx, 0, device);
        return ctx;
    }

    public static CUdeviceptr allocateDeviceMemory(long size) {
        var deviceMemory = new CUdeviceptr();
        cuMemAlloc(deviceMemory, size);
        return deviceMemory;
    }

    public static curandGenerator createRandomGenerator() {
        var generator = new curandGenerator();
        curandCreateGenerator(generator, CURAND_RNG_PSEUDO_PHILOX4_32_10);
        curandSetPseudoRandomGeneratorSeed(generator, System.currentTimeMillis());
        return generator;
    }

    public static void freeResources(NativePointerObject... resources) {
        Objects.requireNonNull(resources);

        for (var res : resources) {
            if (res instanceof curandGenerator) {
                curandDestroyGenerator((curandGenerator) res);
            } else if (res instanceof CUdeviceptr) {
                cuMemFree((CUdeviceptr) res);
            } else {
                throw new IllegalArgumentException(String.format("No handler for resource type: %s", res.getClass().getName()));
            }
        }
    }

    public static CUdeviceptr qmcGeneratePoints(long nPoints, long seqOffset) throws IOException {
        int blockSizeX = 256;
        int gridSizeX = (int) Math.ceil((double) nPoints / blockSizeX);

        var guesses = JcudaUtils.allocateDeviceMemory(nPoints * Sizeof.SHORT);
        var kernelParams = Pointer.to(
                Pointer.to(guesses),
                Pointer.to(new long[]{nPoints}),
                Pointer.to(new long[]{seqOffset})
        );

        cuLaunchKernel(
                JcudaUtils.loadFunctionFromPtx("/CudaQmcKernel.ptx", "qmc_mapper"),
                gridSizeX, 1, 1,
                blockSizeX, 1, 1,
                0, null,
                kernelParams, null
        );
        cuCtxSynchronize();

        return guesses;
    }

    public static long sumShortArray(CUdeviceptr inputArray, long size) throws IOException {
        Objects.requireNonNull(inputArray);

        int blockSizeX = 256;
        int gridSizeX = (int) Math.ceil((double) size / blockSizeX);

        var partialSums = JcudaUtils.allocateDeviceMemory((long) gridSizeX * Sizeof.SHORT);
        var kernelParams = Pointer.to(
                Pointer.to(inputArray),
                Pointer.to(partialSums),
                Pointer.to(new long[]{size})
        );

        cuLaunchKernel(
                JcudaUtils.loadFunctionFromPtx("/CudaReduction.ptx", "reduce_short"),
                gridSizeX, 1, 1,
                blockSizeX, 1, 1,
                blockSizeX * Sizeof.SHORT, null,
                kernelParams, null
        );
        cuCtxSynchronize();

        var sums = new short[gridSizeX];
        cuMemcpyDtoH(Pointer.to(sums), partialSums, (long) gridSizeX * Sizeof.SHORT);
        JcudaUtils.freeResources(partialSums);
        return IntStream.range(0, sums.length).map(i -> sums[i]).sum();
    }

    public static CUdeviceptr fuzzyPerformOps(byte[] inputRecords, long nRecords) throws IOException {
        Objects.requireNonNull(inputRecords);

        int blockSizeX = 1024 / FuzzyUtils.SETS_IN_RECORD;
        int gridSizeX = (int) Math.ceil((double) inputRecords.length / blockSizeX);

        long nBytes = nRecords * FuzzyUtils.RECORD_BYTES;
        var records = JcudaUtils.allocateDeviceMemory(nBytes);
        var temp = JcudaUtils.allocateDeviceMemory(nBytes * 2);
        cuMemcpyHtoD(records, Pointer.to(inputRecords), nBytes);

        var kernelParams = Pointer.to(
                Pointer.to(records),
                Pointer.to(temp),
                Pointer.to(new long[]{nRecords})
        );
        cuLaunchKernel(
                JcudaUtils.loadFunctionFromPtx("/CudaFuzzyCompute.ptx", "fuzzy_compute"),
                gridSizeX, 1, 1,
                blockSizeX, FuzzyUtils.SETS_IN_RECORD, 1,
                0, null,
                kernelParams, null
        );
        cuCtxSynchronize();

        return records;
    }

    public static CUdeviceptr fuzzyFilterRows(
            float[] rowValues, float[] membershipFnParams, float threshold, int nRows, int nFilters
    ) throws IOException {
        Objects.requireNonNull(rowValues);
        Objects.requireNonNull(membershipFnParams);

        long rowValueBytes = (long) rowValues.length * Sizeof.FLOAT;
        var rowValuesPtr = JcudaUtils.allocateDeviceMemory(rowValueBytes);
        cuMemcpyHtoD(rowValuesPtr, Pointer.to(rowValues), rowValueBytes);

        long membershipFnParamsBytes = (long) membershipFnParams.length * Sizeof.FLOAT;
        var membershipFnParamsPtr = JcudaUtils.allocateDeviceMemory(membershipFnParamsBytes);
        cuMemcpyHtoD(membershipFnParamsPtr, Pointer.to(membershipFnParams), membershipFnParamsBytes);

        int blockSizeX = 256;
        int gridSizeX = (int) Math.ceil((double) nRows / blockSizeX);
        var member = JcudaUtils.allocateDeviceMemory((long) nRows * Sizeof.SHORT);
        var kernelParams = Pointer.to(
                Pointer.to(member),
                Pointer.to(rowValuesPtr),
                Pointer.to(membershipFnParamsPtr),
                Pointer.to(new float[]{threshold}),
                Pointer.to(new int[]{nRows}),
                Pointer.to(new int[]{nFilters})
        );

        cuLaunchKernel(
                JcudaUtils.loadFunctionFromPtx("/CudaFuzzyFilter.ptx", "fuzzy_filter"),
                gridSizeX, 1, 1,
                blockSizeX, 1, 1,
                0, null,
                kernelParams, null
        );
        cuCtxSynchronize();

        JcudaUtils.freeResources(rowValuesPtr, membershipFnParamsPtr);
        return member;
    }

    public static CUfunction loadFunctionFromPtx(String ptxPath, String funcName) throws IOException {
        Objects.requireNonNull(ptxPath);
        Objects.requireNonNull(funcName);

        var resource = JcudaUtils.class.getResourceAsStream(ptxPath);
        var ptx = JcudaUtils.toNullTerminatedByteArray(resource);

        var module = new CUmodule();
        cuModuleLoadData(module, ptx);

        var func = new CUfunction();
        cuModuleGetFunction(func, module, funcName);
        return func;
    }

    private static byte[] toNullTerminatedByteArray(InputStream inStream) throws IOException {
        Objects.requireNonNull(inStream);

        var outStream = new ByteArrayOutputStream();
        inStream.transferTo(outStream);
        outStream.write(0);
        return outStream.toByteArray();
    }
}
\end{lstlisting}
\newpage

\section*{Klasa CpuQmcMapper} \label{ch:cpu-qmc-mapper}

Klasa CpuQmcMapper (listing \ref{lst:cpu-qmc-mapper}) implementuje komponent Mapper dla
platformy Apache Hadoop, którego zadaniem w programie PiEstimation jest generowanie punktów na
podstawie sekwencji Haltona i~zliczanie tych, które znajdują się w obrębie koła. Komponent
próbkuje punkty dla pewnego przedziału, określonego przez parametry \lstinline{offset} 
i~\lstinline{"size"}. Wyniki są zapisywane do kontekstu z~kluczem typu \lstinline{BooleanWritable}
o~wartości true lub false (w~zależności od tego, czy punkt jest wewnątrz czy na zewnątrz)
i~wartością typu LongWritable, reprezentującą liczbę punktów.

\begin{lstlisting}[
    language=Java,
    label=lst:cpu-qmc-mapper,
    caption={Implementacja komponentu Mapper na procesor główny dla programu PiEstimation.}
]
public class CpuQmcMapper extends Mapper<LongWritable, LongWritable, BooleanWritable, LongWritable> {

    public void map(LongWritable offset, LongWritable size, Context context) throws IOException, InterruptedException {
        long numInside = 0L;
        long numOutside = 0L;

        for (long i = 0; i < size.get(); ++i) {
            final double[] point = QmcUtils.getRandomPoint(offset.get() + i);
            if (QmcUtils.isPointInside(point)) {
                numInside++;
            } else {
                numOutside++;
            }
        }

        context.write(new BooleanWritable(true), new LongWritable(numInside));
        context.write(new BooleanWritable(false), new LongWritable(numOutside));
    }
}
\end{lstlisting}
\newpage

\section*{Klasa AparapiQmcMapper} \label{ch:aparapi-qmc-mapper}

Klasa AparapiQmcMapper (listing \ref{lst:aparapi-qmc-mapper}) implementuje komponent Mapper dla
platformy Apache Hadoop, którego zadaniem jest generowanie punktów na podstawie sekwencji
Haltona i~zliczanie punktów znajdujących się w~środku koła. Generowanie i~klasyfikacja punktów
jest wywoływane za pomocą klasy Kernel z biblioteki Aparapi, przez co może być wykonywane na
procesorze graficznym. 

\begin{lstlisting}[
    language=Java,
    label=lst:aparapi-qmc-mapper,
    caption={Implementacja komponentu Mapper dla programu PiEstimation, używająca biblioteki Aparapi.}
]
public class AparapiQmcMapper extends Mapper<LongWritable, LongWritable, BooleanWritable, LongWritable> {

    @Override
    public void map(LongWritable offset, LongWritable size, Context context) throws IOException, InterruptedException {
        int isize = (int) size.get();
        int ioffset = (int) offset.get();
        final boolean[] guesses = new boolean[isize];

        float[][] q = new float[isize][];
        int[][] d = new int[isize][];
        for (int i = 0; i < isize; i++) {
            q[i] = new float[63];
            d[i] = new int[63];
        }

        Kernel kernel = new Kernel() {

            private float getRandomPoint(long index, float[] q, int[] d, int base, int maxDigits) {

                float value = 0.0f;
                long k = index;

                for (int i = 0; i < maxDigits; i++) {
                    q[i] = (i == 0 ? 1.0f : q[i - 1]) / base;
                    d[i] = (int) (k % base);
                    k = (k - d[i]) / base;
                    value += d[i] * q[i];
                }

                boolean cont = true;
                for (int i = 0; i < maxDigits; i++) {
                    if (cont) {
                        d[i]++;
                        value += q[i];
                        if (d[i] < base) {
                            cont = false;
                        } else {
                            value -= (i == 0 ? 1.0f : q[i - 1]);
                        }
                    }
                }

                return value;
            }

            @Override
            public void run() {
                final int i = getGlobalId();
                final long indexOffset = i + ioffset;

                final float x = getRandomPoint(indexOffset, q[i], d[i], 2, 63) - 0.5f;
                final float y = getRandomPoint(indexOffset, q[i], d[i], 3, 40) - 0.5f;

                guesses[i] = (x * x + y * y > 0.25f);
            }
        };
        kernel.execute(Range.create(isize));

        long numInside = 0L;
        long numOutside = 0L;
        for (boolean isOutside : guesses) {
            if (isOutside) {
                numOutside++;
            } else {
                numInside++;
            }
        }

        kernel.dispose();
        context.write(new BooleanWritable(true), new LongWritable(numInside));
        context.write(new BooleanWritable(false), new LongWritable(numOutside));
    }
}
\end{lstlisting}
\newpage

\section*{Klasa JcudaQmcMapper} \label{ch:jcuda-qmc-mapper}

Klasa JcudaQmcMapper (listing \ref{lst:jcuda-qmc-mapper}) implementuje komponent Mapper dla
platformy Hadoop, którego zadaniem jest generowanie punktów na podstawie sekwencji Haltona
i~zliczanie punktów znajdujących się w~środku koła. Generowanie i~klasyfikacja punktów jest
wywoływane za pomocą biblioteki JCuda, prez co jest wykonywane na procesorze graficznym. 

\begin{lstlisting}[
    language=Java,
    label=lst:jcuda-qmc-mapper,
    caption={Implementacja komponentu Mapper dla programu PiEstimation, używająca biblioteki JCuda.}
]
public class JcudaQmcMapper extends Mapper<LongWritable, LongWritable, BooleanWritable, LongWritable> {

    @Override
    public void map(LongWritable offset, LongWritable size, Context context) throws IOException, InterruptedException {
        var ctx = JcudaUtils.createCudaContext();

        var guesses = JcudaUtils.qmcGeneratePoints(size.get(), offset.get());
        long numOutside = JcudaUtils.sumShortArray(guesses, size.get());
        long numInside = size.get() - numOutside;

        context.write(new BooleanWritable(true), new LongWritable(numInside));
        context.write(new BooleanWritable(false), new LongWritable(numOutside));
        JcudaUtils.freeResources(guesses);
    }
}
\end{lstlisting}
\newpage

\section*{Klasa CpuQmcFunction} \label{ch:cpu-qmc-function}

Klasa CpuQmcFunction (listing \ref{lst:cpu-qmc-function} jest odpowiednikiem klasy CpuQmcMapper
(listing \ref{lst:cpu-qmc-mapper}) dla platformy Spark.

\begin{lstlisting}[
    language=Java,
    label=lst:cpu-qmc-function,
    caption={Funkcja na procesor główny dla programu PiEstimation.}
]
public class CpuQmcFunction {

    public static Long call(Tuple2<Long, Long> input) {
        long nSamples = input._2();
        long offset = input._1() * nSamples;

        long numInside = 0L;
        for (long i = 0; i < nSamples; ++i) {
            final double[] point = QmcUtils.getRandomPoint(offset + i);
            if (QmcUtils.isPointInside(point)) {
                numInside++;
            }
        }

        return numInside;
    }
}
\end{lstlisting}
\newpage

\section*{Klasa AparapiQmcFunction} \label{ch:aparapi-qmc-function}

Klasa AparapiQmcFunction (listing \ref{lst:aparapi-qmc-function} jest odpowiednikiem klasy AparapiQmcMapper
(listing \ref{lst:aparapi-qmc-mapper}) dla platformy Spark.

\begin{lstlisting}[
    language=Java,
    label=lst:aparapi-qmc-function,
    caption={Funkcja dla programu PiEstimation, wykorzystująca bibliotekę Aparapi.}
]
public class AparapiQmcFunction {

    public static Long call(Tuple2<Long, Long> input) {
        int nSamples = Math.toIntExact(input._2());
        int offset = (int) (input._1() * nSamples);
        final var guesses = new boolean[nSamples];

        var q = new float[nSamples][];
        var d = new int[nSamples][];
        for (int i = 0; i < nSamples; ++i) {
            q[i] = new float[63];
            d[i] = new int[63];
        }

        var kernel = new Kernel() {

            private float getRandomPoint(long index, float[] q, int[] d, int base, int maxDigits) {

                float value = 0.0f;
                long k = index;

                for (int i = 0; i < maxDigits; ++i) {
                    q[i] = (i == 0 ? 1.0f : q[i - 1]) / base;
                    d[i] = (int) (k % base);
                    k = (k - d[i]) / base;
                    value += d[i] * q[i];
                }

                boolean cont = true;
                for (int i = 0; i < maxDigits; ++i) {
                    if (cont) {
                        d[i]++;
                        value += q[i];
                        if (d[i] < base) {
                            cont = false;
                        } else {
                            value -= (i == 0 ? 1.0f : q[i - 1]);
                        }
                    }
                }

                return value;
            }

            @Override
            public void run() {
                final int i = getGlobalId();
                final long indexOffset = i + offset;

                final float x = getRandomPoint(indexOffset, q[i], d[i], 2, 63) - 0.5f;
                final float y = getRandomPoint(indexOffset, q[i], d[i], 3, 40) - 0.5f;

                guesses[i] = (x * x + y * y <= 0.25f);
            }
        };
        kernel.execute(Range.create(nSamples));

        long numInside = 0L;
        for (boolean isInside : guesses) {
            if (isInside) {
                numInside++;
            }
        }
        kernel.dispose();

        return numInside;
    }
}
\end{lstlisting}
\newpage

\section*{Klasa JcudaQmcFunction} \label{ch:jcuda-qmc-function}

Klasa JcudaQmcFunction (listing \ref{lst:jcuda-qmc-function} jest odpowiednikiem klasy JcudaQmcMapper
(listing \ref{lst:jcuda-qmc-mapper}) dla platformy Spark.

\begin{lstlisting}[
    language=Java,
    label=lst:jcuda-qmc-function,
    caption={Funkcja dla programu PiEstimation, wykorzystująca bibliotekę JCuda.}
]
public class JcudaQmcFunction {

    public static Long call(Tuple2<Long, Long> input) throws IOException {
        long nSamples = input._2();
        long offset = input._1() * nSamples;
        var ctx = JcudaUtils.createCudaContext();

        var guesses = JcudaUtils.qmcGeneratePoints(nSamples, offset);
        long numOutside = JcudaUtils.sumShortArray(guesses, nSamples);
        long numInside = nSamples - numOutside;

        JcudaUtils.freeResources(guesses);
        return numInside;
    }
}
\end{lstlisting}
\newpage

\section*{Klasa CpuFuzzyGenMapper} \label{ch:cpu-fuzzygen-mapper}

Klasa CpuFuzzyGenMapper (listing \ref{lst:cpu-fuzzygen-mapper}) generuje podaną liczbę losowych
wartości zmiennoprzecinkowych na procesorze głównym.

\begin{lstlisting}[
    language=Java,
    label=lst:cpu-fuzzygen-mapper,
    caption={Komponent Mapper na procesor główny dla programu FuzzyGen.}
]
public class CpuFuzzyGenMapper extends Mapper<LongWritable, NullWritable, NullWritable, BytesWritable> {

    @Override
    public void map(LongWritable key, NullWritable ignored, Context context) {
        var nRecords = key.get();
        FuzzyUtils.generateRandomSets(nRecords, buf -> HadoopJobUtils.writeByteRecord(context, buf));
    }
}
\end{lstlisting}
\newpage

\section*{Klasa JCudaFuzzyGenMapper} \label{ch:jcuda-fuzzygen-mapper}

Klasa JCudaFuzzyGenMapper (listing \ref{lst:jcuda-fuzzygen-mapper}) generuje podaną liczbę losowych
wartości zmiennoprzecinkowych na procesorze graficznym.

\begin{lstlisting}[
    language=Java,
    label=lst:jcuda-fuzzygen-mapper,
    caption={Komponent Mapper na procesor graficzny dla programu FuzzyGen.}
]
public class JcudaFuzzyGenMapper extends Mapper<LongWritable, NullWritable, NullWritable, BytesWritable> {

    @Override
    public void map(LongWritable key, NullWritable ignored, Context context) {
        var ctx = JcudaUtils.createCudaContext();
        var nRecords = key.get();
        var floatCount = nRecords * RECORD_SIZE;
        var byteCount = floatCount * Sizeof.FLOAT;

        var randOutput = JcudaUtils.allocateDeviceMemory(byteCount);
        var generator = JcudaUtils.createRandomGenerator();

        curandGenerateUniform(generator, randOutput, floatCount);
        cuCtxSynchronize();

        FuzzyUtils.forEachRecord(randOutput, nRecords, buf -> HadoopJobUtils.writeByteRecord(context, buf));
        JcudaUtils.freeResources(generator, randOutput);
    }
}
\end{lstlisting}
\newpage

\section*{Klasa CpuFuzzyGenFunction} \label{ch:cpu-fuzzygen-function}

Klasa CpuFuzzyGenFunction (listing \ref{lst:cpu-fuzzygen-function}) jest odpowiednikiem
klasy CpuFuzzyGenMapper (listting \ref{lst:cpu-fuzzygen-mapper}) dla platformy Spark.

\begin{lstlisting}[
    language=Java,
    label=lst:cpu-fuzzygen-function,
    caption={Funkcja na procesor główny dla programu FuzzyGen.}
]
public class CpuFuzzyGenFunction {

    public static Iterator<byte[]> call(Long nRecords) {
        var results = new ArrayList<byte[]>();
        FuzzyUtils.generateRandomSets(nRecords, results::add);
        return results.iterator();
    }
}
\end{lstlisting}
\newpage

\section*{Klasa JCudaFuzzyGenFunction} \label{ch:jcuda-fuzzygen-function}

Klasa JCudaFuzzyGenFunction (listing \ref{lst:jcuda-fuzzygen-function}) jest odpowiednikiem
klasy JCudaFuzzyGenMapper (listting \ref{lst:jcuda-fuzzygen-mapper}) dla platformy Spark.

\begin{lstlisting}[
    language=Java,
    label=lst:jcuda-fuzzygen-function,
    caption={Funkcja na procesor graficzny dla programu FuzzyGen.}
]
public class JcudaFuzzyGenFunction {

    public static Iterator<byte[]> call(Long nRecords) {
        var ctx = JcudaUtils.createCudaContext();
        long floatCount = nRecords * RECORD_SIZE;
        long byteCount = floatCount * Sizeof.FLOAT;

        var randOutput = JcudaUtils.allocateDeviceMemory(byteCount);
        var generator = JcudaUtils.createRandomGenerator();

        curandGenerateUniform(generator, randOutput, floatCount);
        cuCtxSynchronize();

        var results = new ArrayList<byte[]>();
        FuzzyUtils.forEachRecord(randOutput, nRecords, results::add);
        JcudaUtils.freeResources(generator, randOutput);
        return results.iterator();
    }
}
\end{lstlisting}
\newpage

\section*{Klasa CpuFuzzyComputeMapper} \label{ch:cpu-fuzzycompute-mapper}

Klasa CpuFuzzyComputeMapper (listing \ref{lst:cpu-fuzzycompute-mapper}) wykonuje operacje na zbiorach rozmytych wykorzystując procesor główny.

\begin{lstlisting}[
    language=Java,
    label=lst:cpu-fuzzycompute-mapper,
    caption={Komponent Mapper na procesor główny dla programu FuzzyCompute.}
]
public class CpuFuzzyComputeMapper extends Mapper<NullWritable, BytesWritable, NullWritable, BytesWritable> {

    @Override
    public void map(NullWritable ignored, BytesWritable value, Context context) throws IOException, InterruptedException {
        var bytes = value.getBytes();
        FuzzyUtils.performOps(bytes);
        context.write(NullWritable.get(), new BytesWritable(bytes));
    }
}
\end{lstlisting}
\newpage

\section*{Klasa JCudaFuzzyComputeMapper} \label{ch:jcuda-fuzzycompute-mapper}

Klasa JCudaFuzzyComputeMapper (listing \ref{lst:jcuda-fuzzycompute-mapper}) wykonuje operacje na zbiorach rozmytych wykorzystując procesor graficzny.

\begin{lstlisting}[
    language=Java,
    label=lst:jcuda-fuzzycompute-mapper,
    caption={Komponent Mapper na procesor graficzny dla programu FuzzyCompute.}
]
public class JcudaFuzzyComputeMapper extends Mapper<NullWritable, BytesWritable, NullWritable, BytesWritable> {

    @Override
    public void run(Context context) throws IOException, InterruptedException {
        setup(context);

        try {
            var ctx = JcudaUtils.createCudaContext();
            var bytes = new ByteArrayOutputStream();
            var nRecords = 0L;
            while (context.nextKeyValue()) {
                bytes.writeBytes(context.getCurrentValue().getBytes());
                ++nRecords;
            }

            var output = JcudaUtils.fuzzyPerformOps(bytes.toByteArray(), nRecords);
            FuzzyUtils.forEachRecord(output, nRecords, buf -> HadoopJobUtils.writeByteRecord(context, buf));
            JcudaUtils.freeResources(output);
        } finally {
            cleanup(context);
        }
    }
}
\end{lstlisting}
\newpage

\section*{Klasa CpuFuzzyComputeFunction} \label{ch:cpu-fuzzycompute-function}

Klasa CpuFuzzyComputeFunction (listing \ref{lst:cpu-fuzzycompute-function}) jest odpowiednikiem
klasy CpuFuzzyGenMapper (listting \ref{lst:cpu-fuzzycompute-mapper}) dla platformy Spark.

\begin{lstlisting}[
    language=Java,
    label=lst:cpu-fuzzycompute-function,
    caption={Funkcja na procesor główny dla programu FuzzyCompute.}
]
public class CpuFuzzyComputeFunction {

    public static Iterator<byte[]> call(List<byte[]> records) {
        for (var record : records) {
            FuzzyUtils.performOps(record);
        }
        return records.iterator();
    }
}
\end{lstlisting}
\newpage

\section*{Klasa JCudaFuzzyComputeFunction} \label{ch:jcuda-fuzzycompute-function}

Klasa JCudaFuzzyComputeFunction (listing \ref{lst:jcuda-fuzzycompute-function}) jest odpowiednikiem
klasy JCudaFuzzyComputeMapper (listting \ref{lst:jcuda-fuzzycompute-mapper}) dla platformy Spark.

\begin{lstlisting}[
    language=Java,
    label=lst:jcuda-fuzzycompute-function,
    caption={Funkcja na procesor graficzny dla programu FuzzyCompute.}
]
public class JcudaFuzzyComputeFunction {

    public static Iterator<byte[]> call(List<byte[]> records) throws IOException {
        var ctx = JcudaUtils.createCudaContext();
        var bytes = new ByteArrayOutputStream();
        for (var record : records) {
            bytes.writeBytes(record);
        }

        var output = JcudaUtils.fuzzyPerformOps(bytes.toByteArray(), records.size());
        var results = new ArrayList<byte[]>();
        FuzzyUtils.forEachRecord(output, records.size(), results::add);
        JcudaUtils.freeResources(output);
        return results.iterator();
    }
}
\end{lstlisting}
\newpage

\section*{Klasa CpuFuzzyFilterFunction} \label{ch:cpu-fuzzyfilter-function}

Klasy CpuFuzzyFilterFunction, FuzzyTNorm oraz FuzzyPredicate (listing \ref{lst:cpu-fuzzyfilter-function})
realizują operacje filtrowania na zbiorach rozmytych wykorzystując procesor główny.

\begin{lstlisting}[
    language=Java,
    label=lst:cpu-fuzzyfilter-function,
    caption={Funkcja na procesor główny dla programu FuzzyFilter.}
]
public class FuzzyTNorm implements FilterFunction<Row>, Serializable {

    private final float threshold;
    private final List<FuzzyPredicate> membershipList;

    public FuzzyTNorm(float threshold, List<FuzzyPredicate> membershipList) {
        this.threshold = threshold;
        this.membershipList = membershipList;
    }

    public float getThreshold() {
        return threshold;
    }

    public List<FuzzyPredicate> getMembershipList() {
        return membershipList;
    }

    @Override
    public boolean call(Row value) throws Exception {
        return threshold <= membershipList.stream()
                .mapToDouble(p -> (double) p.getMembership(value))
                .min()
                .orElse(1.0);
    }
}

public class FuzzyPredicate implements Serializable {

    private final float a;
    private final float b;
    private final float c;
    private final float d;
    private final String column;

    public FuzzyPredicate(String expr) {
        String[] parts = expr.split(" ", 5);
        this.a = Float.parseFloat(parts[0]);
        this.b = Float.parseFloat(parts[1]);
        this.c = Float.parseFloat(parts[2]);
        this.d = Float.parseFloat(parts[3]);
        this.column = parts[4];
    }

    public float getA() {
        return a;
    }

    public float getB() {
        return b;
    }

    public float getC() {
        return c;
    }

    public float getD() {
        return d;
    }

    public String getColumn() {
        return column;
    }

    public float getValue(Row row) {
        return Float.parseFloat(row.getString(row.fieldIndex(this.column)));
    }

    public float getMembership(Row row) {
        float value = getValue(row);
        if (value <= this.a || value >= this.d) {
            return 0;  // out of range
        } else if (value >= this.b && value <= this.c) {
            return 1;  // inside plateau
        } else if (value < this.b) { // (val > a && val < b)
            return (value - this.a) / (this.b - this.a); // left triangle
        } else {  // (val > c && val < d)
            return (this.d - value) / (this.d - this.c);  // right triangle
        }
    }
}

public class CpuFuzzyFilterFunction implements FuzzyTNormFilter {

    private final FuzzyTNorm tNorm;

    public CpuFuzzyFilterFunction(FuzzyTNorm tNorm) {
        this.tNorm = tNorm;
    }

    @Override
    public Dataset<Row> apply(Dataset<Row> dataset) {
        return dataset.filter(tNorm);
    }
}
\end{lstlisting}
\newpage

\section*{Klasa JCudaFuzzyFilterFunction} \label{ch:jcuda-fuzzyfilter-function}

Klasa JCudaFuzzyFilterFunction (listing \ref{lst:jcuda-fuzzyfilter-function})
realizuje operacje filtrowania na zbiorach rozmytych wykorzystując procesor graficzny.

\begin{lstlisting}[
    language=Java,
    label=lst:jcuda-fuzzyfilter-function,
    caption={Funkcja na procesor graficzny dla programu FuzzyFilter.}
]
public class JcudaFuzzyFilterFunction implements FuzzyTNormFilter, MapPartitionsFunction<Row, Row> {

    private static final int A = 0;
    private static final int B = 1;
    private static final int C = 2;
    private static final int D = 3;

    private final float threshold;
    private final float[] memberFnParams;
    private final String[] column;

    public JcudaFuzzyFilterFunction(FuzzyTNorm tNorm) {
        var membershipList = tNorm.getMembershipList();
        int nFilters = membershipList.size();
        this.memberFnParams = new float[nFilters * 4];
        this.column = new String[nFilters];
        for (int i = 0; i < nFilters; ++i) {
            fillMembershipFnParams(i, membershipList.get(i));
            this.column[i] = membershipList.get(i).getColumn();
        }

        this.threshold = tNorm.getThreshold();
    }

    @Override
    public Dataset<Row> apply(Dataset<Row> dataset) {
        return dataset.mapPartitions(this, dataset.encoder());
    }

    @Override
    public Iterator<Row> call(Iterator<Row> input) throws Exception {
        var ctx = JcudaUtils.createCudaContext();
        var rows = new ArrayList<Row>();
        input.forEachRemaining(rows::add);

        int nRows = rows.size();
        var memberPtr = JcudaUtils.fuzzyFilterRows(getRowValues(rows), memberFnParams, threshold, nRows, column.length);
        short[] member = new short[nRows];
        cuMemcpyDtoH(Pointer.to(member), memberPtr, (long) nRows * Sizeof.SHORT);

        var result = new ArrayList<Row>();
        for (int i = 0; i < nRows; ++i) {
            if (member[i] == 1) {
                result.add(rows.get(i));
            }
        }
        JcudaUtils.freeResources(memberPtr);
        return result.iterator();
    }

    private float[] getRowValues(List<Row> rows) {
        int nRows = rows.size();
        int nFilters = column.length;
        float[] values = new float[nFilters * nRows];

        for (int i = 0; i < nFilters; ++i) {
            for (int j = 0; j < nRows; ++j) {
                var row = rows.get(j);
                var value = row.getString(row.fieldIndex(column[i]));
                values[i * nRows + j] = Float.parseFloat(value);
            }
        }
        return values;
    }

    private void fillMembershipFnParams(int i, FuzzyPredicate fuzzyPredicate) {
        this.memberFnParams[i * 4 + A] = fuzzyPredicate.getA();
        this.memberFnParams[i * 4 + B] = fuzzyPredicate.getB();
        this.memberFnParams[i * 4 + C] = fuzzyPredicate.getC();
        this.memberFnParams[i * 4 + D] = fuzzyPredicate.getD();
    }
}
\end{lstlisting}
\newpage
