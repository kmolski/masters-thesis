% !TeX spellcheck = pl_PL
%%%%%%%%%%%%%%%%%%%%%%%%%%%%%%%%%%%%%%%%%%%
%                                        %
% Szablon pracy dyplomowej magisterskiej %
% zgodny  z aktualnymi  przepisami  SZJK %
%                                        %
%%%%%%%%%%%%%%%%%%%%%%%%%%%%%%%%%%%%%%%%%%
%                                        %
%  (c) Krzysztof Simiński, 2018-2023     %
%                                        %
%%%%%%%%%%%%%%%%%%%%%%%%%%%%%%%%%%%%%%%%%%
%                                        %
% Najnowsza wersja szablonów jest        %
% podstępna pod adresem                  %
% github.com/ksiminski/polsl-aei-theses  %
%                                        %
%%%%%%%%%%%%%%%%%%%%%%%%%%%%%%%%%%%%%%%%%%
%
%
% Projekt LaTeXowy zapewnia odpowiednie formatowanie pracy,
% zgodnie z wymaganiami Systemu zapewniania jakości kształcenia.
% Proszę nie zmieniać ustawień formatowania (np. fontu,
% marginesów, wytłuszczeń, kursywy itd. ).
%
% Projekt można kompilować na kilka sposobów.
%
% 1. kompilacja pdfLaTeX
%
% pdflatex main
% bibtex   main
% pdflatex main
% pdflatex main
%
%
% 2. kompilacja XeLaTeX
%
% Kompilatacja przy użyciu XeLaTeXa różni się tym, że na stronie
% tytułowej używany jest font Calibri. Wymaga to jego uprzedniego
% zainstalowania.
%
% xelatex main
% bibtex  main
% xelatex main
% xelatex main
%
%
%%%%%%%%%%%%%%%%%%%%%%%%%%%%%%%%%%%%%%%%%%%%%%%%%%%%%
% W przypadku pytań, uwag, proszę pisać na adres:   %
%      krzysztof.siminski(małpa)polsl.pl            %
%%%%%%%%%%%%%%%%%%%%%%%%%%%%%%%%%%%%%%%%%%%%%%%%%%%%%
%
% Chcemy ulepszać szablony LaTeXowe prac dyplomowych.
% Wypełniając ankietę spod poniższego adresu pomogą
% Państwo nam to zrobić. Ankieta jest całkowicie
% anonimowa. Dziękujemy!


% https://docs.google.com/forms/d/e/1FAIpQLScyllVxNKzKFHfILDfdbwC-jvT8YL0RSTFs-s27UGw9CKn-fQ/viewform?usp=sf_link
%
%%%%%%%%%%%%%%%%%%%%%%%%%%%%%%%%%%%%%%%%%%%%%%%%%%%%%%%%%%%%%%%%%%%%%%%%%

%%%%%%%%%%%%%%%%%%%%%%%%%%%%%%%%%%%%%%%%%%%%%%%
%                                             %
% PERSONALIZACJA PRACY – DANE PRACY           %
%                                             %
%%%%%%%%%%%%%%%%%%%%%%%%%%%%%%%%%%%%%%%%%%%%%%%
\newcommand{\FirstNameAuthor}{Krzysztof}
\newcommand{\SurnameAuthor}{Molski}
\newcommand{\IdAuthor}{282711}

\newcommand{\FirstNameCoauthor}{}
\newcommand{\SurnameCoauthor}{}
\newcommand{\IdCoauthor}{}
%%%%%%%%%%

\newcommand{\Supervisor}{dr hab. inż. Dariusz Mrozek}
\newcommand{\Title}{Budowa klastra Big Data w środowisku minikomputerów jednopłytkowych z akceleracją GPU}
\newcommand{\TitleAlt}{Building a Big Data cluster in an environment of single-board mini-computers with GPU acceleration}
\newcommand{\Program}{Informatyka}
\newcommand{\Specialisation}{Internet i technologie sieciowe}
\newcommand{\Departament}{Informatyki Stosowanej}

\newcommand{\Consultant}{}

%%%%%%%%%%%%%%%%%%%%%%%%%%%%%%%%%%%%%%%%%%


%%%%%%%%%%%%%%%%%%%%%%%%%%%%%%%%%%%%%%%%%%%%%%%
%                                             %
% KONIEC PERSONALIZACJI PRACY                 %
%                                             %
%%%%%%%%%%%%%%%%%%%%%%%%%%%%%%%%%%%%%%%%%%%%%%%

%%%%%%%%%%%%%%%%%%%%%%%%%%%%%%%%%%%%%%%%

\input{config/settings.tex} % Proszę nie modyfikować pliku settings.tex

\usepackage{listings}
\lstset{
morekeywords={__syncthreads, __global__, __shared__, __device__, __host__},
commentstyle=\textit,
identifierstyle=\textsf,
keywordstyle=\sffamily\bfseries,
tabsize=4,
frame=lines,
numbers=left,
numberstyle=\tiny,
numbersep=5pt,
breaklines=true,
escapeinside={@*}{*@},
}
 % Tutaj proszę umieścić swoje pakiety, makra, ustawienia itd.

%%%%%%%%%%%%%%%%%%%%%%%%%%%%%%%%%%%%%%%%


\begin{document}
%\kslistofremarks

\frontmatter
\input{config/titlepage.tex}  % Proszę nie modyfikować pliku titlepage.tex

\cleardoublepage

\rmfamily\normalfont
\pagestyle{empty}


%%% No to zaczynamy pisać pracę :-) %%%%

\subsubsection*{Tytuł pracy}
\Title

\subsubsection*{Streszczenie}
Celem pracy jest budowa i analiza wydajności klastra Big Data składającego się z
minikomputerów Nvidia Jetson Nano. W ramach badań zostaną opracowane i przetestowane
programy wykorzystujące zarówno procesor główny komputera, jak i procesor graficzny.
Przygotowane progamy będą wykorzystywać platformy do obliczeń rozproszonych Apache Hadoop
oraz Apache Spark. Wydajność klastra zostanie porównana z konkurencyjnymi minikomputerami
jednopłytkowymi.

\subsubsection*{Słowa kluczowe}
Big Data, CUDA, Obliczenia rozproszone, Programowanie równoległe

\subsubsection*{Thesis title}
\begin{otherlanguage}{british}
    \TitleAlt
\end{otherlanguage}

\subsubsection*{Abstract}
\begin{otherlanguage}{british}
    The objective of this study is to build and analyse the performance of a Big Data
    cluster consisting of Nvidia Jetson Nano minicomputers. This will include the development
    and benchmarking of programs using the computer's CPU and GPU. The prepared applications
    will use Apache Hadoop and Apache Spark distributed computing platforms. The performance
    of the cluster will be compared with competing single-board minicomputers.
\end{otherlanguage}

\subsubsection*{Key words}
\begin{otherlanguage}{british}
    Big Data, CUDA, Distributed computing, Parallel programming
\end{otherlanguage}
 % informacje redakcyjne


%%%%%%%%%%%%%%%%%% SPIS TRESCI %%%%%%%%%%%%%%%%%%%%%%
% Add \thispagestyle{empty} to the toc file (main.toc), because \pagestyle{empty} doesn't work if the TOC has multiple pages
\addtocontents{toc}{\protect\thispagestyle{empty}}
\tableofcontents

%%%%%%%%%%%%%%%%%%%%%%%%%%%%%%%%%%%%%%%%%%%%%%%%%%%%%
\setcounter{stronyPozaNumeracja}{\value{page}}
\mainmatter
\pagestyle{empty}

\cleardoublepage

\pagestyle{NumeryStronNazwyRozdzialow}

%%%%%%%%%%%%%% wlasciwa tresc pracy %%%%%%%%%%%%%%%%%

\chapter{Wstęp}

Wraz ze zwiększającą się skalą integracji i miniaturyzacją procesorów powstają nowe,
coraz mniejsze formaty komputerów, czego dobrym przykładem są modele jednopłytkowe
(ang. \english{Single-board computer, SBC}).
Ten rodzaj komputera, w odróżnieniu od tradycyjnych rozwiązań, łączy procesor, pamięć
operacyjną oraz interfejsy wejścia/wyjścia na jednej płytce drukowanej.

Najpopularniejsze urządzenie w tej klasie pojawiło się w 2012 roku, wraz z premierą
pierwszej generacji komputera Raspberry Pi. Jego podstawą był system na czipie (ang.
\english{system on a chip, SoC}) łączący następujące układy scalone \cite{rpi-spec}:
\begin{itemize}
    \item jednordzeniowy procesor ARM o taktowaniu 700 MHz,
    \item pamięć RAM o pojemności 512 MiB,
    \item procesor graficzny,
    \item kontroler Fast Ethernet,
    \item kontroler USB 2.0.
\end{itemize}

Oficjalne wsparcie dla systemu operacyjnego Linux, mały pobór prądu oraz rosnąca wydajność
kolejnych generacji sprawiły, że minikomputery Raspberry Pi stały się dobrą podstawą dla
klastrów obliczeniowych do aplikacji \english{Big Data} \cite{rpi-cluster-2}. Takie klastry
charakteryzują się atrakcyjnym stosunkiem wydajności do ceny, nawet jeśli maksymalna
wydajność jest ograniczona w porównaniu z komercyjnymi rozwiązaniami.

Co więcej, popularne platformy do obliczeń na dużych zbiorach danych, takie jak Apache Hadoop
i Apache Spark, w najnowszych wersjach zarządzają również procesorami graficznymi, co ułatwia
wykorzystanie tych jednostek w obliczeniach rozproszonych \cite{spark-gpu}. W połączeniu z
komputerami jednopłytkowymi wspierającymi akcelerację GPU daje to szansę na osiągnięcie
większej mocy obliczeniowej i lepszego stosunku wydajności do ceny.

Celem pracy jest utworzenie i zbadanie wydajności klastra do obliczeń Big Data z~wykorzystaniem
minikomputerów wspierających akcelerację GPU. Klaster powinien realizować obliczenia rozproszone
z wykorzystaniem popularnych platform do przetwarzania \english{Big Data}, zapewniając jednocześnie
konkurencyjną wydajność oraz niski koszt budowy.

W ramach pracy zostaną opracowane programy wykorzystujące zarówno procesory główne, jak
i procesory graficzne minikomputerów, zaimplementowane jako rozproszone zadania dla platform
Apache Hadoop oraz Apache Spark. Języki programowania oraz narzędzia do programowania GPU zostaną
dobrane z uwzględnieniem kompatybilności i~możliwości ponownego użycia kodu.

Kolejnym ważnym aspektem będzie zautomatyzowanie procesu testowania opracowanych programów.
Opracowane zostaną narzędzia do instalacji, konfiguracji i efektywnego zarządzania klastrem.
Umożliwi to zbadanie wydajności dla wielu kombinacji programów, platform obliczeniowych,
zbiorów danych oraz konfiguracji klastra. Zebrane w ten sposób dane posłużą do porównania
klastra z innymi rozwiązaniami pod kątem wydajności i~opłacalności.

\section*{Charakterystyka rozdziałów}

Rozdział "\nameref{ch:analiza-tematu}" zawiera definicję zbiorów danych \english{Big Data} oraz opis
podejść wykorzystywanych do przetwarzania tych danych. Opisano też popularne platformy do obliczeń
rozproszonych i modele programowania procesorów graficznych. Omówiono istniejące klastry obliczeniowe
złożone z minikomputerów jednopłytkowych.

W rozdziale "\nameref{ch:budowa-klastra}" znajduje się opis klastra obliczeniowego zbudowanego
na potrzeby pracy, poruszający kwestie doboru sprzętu, programów, a także podejścia do zarządzania
klastra. Przedstawiono kompletny proces instalacji oprogramowania i konfiguracji środowiska.
Zawarto też opis programów testowych, które zostały opracowane w ramach badań.

Rozdział "\nameref{ch:badania}" opisuje badania wydajności przeprowadzone dla programów testowych.
Opisano metodykę badań, zwracając szczególną uwagę na charakterystykę programów w~językach Java
oraz CUDA. Przedstawiono wyniki dla programów PiEstimation, FuzzyGen, FuzzyCompute i~FuzzyFilter,
a~także porównania z~konkurencyjnymi rozwiązaniami.

Kończący pracę rozdział "\nameref{ch:podsumowanie}" zawiera podsumowanie zadań wykonanych w~ramach
pracy i~ocenę powstałego w~jej wyniku rozwiązania. W~rozdziale przedstawiono też porównanie ogólnej
wydajności klastra z~innymi rozwiązaniami, w~tym z~klastrem opartym o~komputery Raspberry Pi 4B
oraz pojedynczym komputerem stacjonarnym.
  % wstęp
\chapter{Analiza tematu} \label{ch:analiza-tematu}

\section{Definicja Big Data}

Termin \english{Big Data} odnosi się do zbiorów danych o dużej objętości i specjalnych technik
opracowanych do ich przetwarzania. Ze względu na wysoki stopień personalizacji istniejących
systemów, popularne definicje tego terminu znacząco się od siebie różnią. Mimo że idea przetwarzania
ogromnych ilości danych jest wspólnym mianownikiem, różne organizacje, naukowcy i eksperci mogą
mieć odmienne podejście do definicji tego terminu w~zależności od kontekstu, w którym jest używany.

\english{Big Data} może być rozumiane jako zdolność do przetwarzania dużych zbiorów danych
w celu wyciągania wniosków ważnych dla przedsiębiorstw \cite{big-data-2}, takich jak lepsze
zrozumienie klienta, optymalizacja operacji czy tworzenie nowych produktów. Z kolei dla
naukowców termin ten może oznaczać możliwość zbierania i analizowania skomplikowanych
zestawów danych w celu badania pewnych zjawisk na skalę wcześniej nieosiągalną.

Inne definicje mogą podkreślać technologie i narzędzia wykorzystywane do transformacji danych,
takie jak klastry obliczeniowe, algorytmy analizy danych czy techniki uczenia maszynowego.
Z czasem z pojęciem \english{Big Data} skojarzono kilka cech (nazywanych powszechnie czterema "V"),
które odnoszą się do technicznych trudności związanych z~przetwarzaniem tego typu danych \cite{big-data-1}:
\begin{itemize}
      \item objętość (ang. \english{volume}) -- dane są generowane w bardzo dużych ilościach.
            Przykładowo, w mediach społecznościowych generowane są miliardy interakcji dziennie,
      \item szybkość (ang. \english{velocity}) -- dane są generowane w szybkim tempie.
            Wiele aplikacji wymaga przetwarzania tych danych w czasie rzeczywistym lub niemal rzeczywistym,
      \item różnorodność (ang. \english{variety}) -- dane napływają w różnych formatach, od
            relacyjnych baz danych do niestrukturalnych treści takich jak e-maile, zdjęcia i filmy,
      \item prawdziwość (ang. \english{veracity}) -- dane mają ograniczoną dokładność, nie wszystkie
            są przydatne, przez co ważne jest zrozumienie ich wiarygodności i dokładności.
\end{itemize}

Poza wymienionymi wyżej cechami \english{Big Data} jest często kojarzone również z wysoką rozdzielczością,
możliwością indeksowania, modyfikacji i dodawania atrybutów, niedokładnością, relacyjnym charakterem oraz
kompleksowością \cite{big-data-1}.

\section{Przetwarzanie Big Data}

W dzisiejszym świecie, w którym dane z różnych źródeł są produkowane w ogromnych ilościach,
stajemy przed wyzwaniem ich efektywnego przechowywania i przetwarzania. Społeczeństwo
generuje codziennie petabajty informacji, które muszą być analizowane, przechowywane i
przetwarzane w odpowiedni sposób \cite{big-data-2}.

Ze względu na tę ogromną objętość i szybki przyrost danych, tradycyjne metody stają się
niewystarczające. Dlatego też coraz bardziej istotne staje się korzystanie z odpowiedniego
sprzętu i oprogramowania, które są w stanie sprostać tym wyzwaniom. Jednym z~najważniejszych
kryteriów dla takich narzędzi jest ich skalowalność, czyli zdolność systemu do zwiększenia
swojej wydajności w odpowiedzi na rosnące obciążenie. Wyróżniamy skalowanie poziome i pionowe.

\subsection*{Skalowanie poziome}

Skalowanie poziome opiera się o dodawanie maszyn do istniejącego systemu w celu rozłożenia obciążenia
i zwiększenia całkowitej wydajności. W przeciwieństwie do skalowania pionowego, które polega na
ulepszaniu pojedynczej maszyny, skalowanie poziome opiera się o łączenie komputerów w klastry.

Główną zaletą skalowania poziomego jest lepsza opłacalność i elastyczność, z drugiej strony systemy
rozproszone są dalece bardziej złożone i trudniejsze w analizie.
Poniżej wymieniono kilka narzędzi używanych do realizacji obliczeń rozproszonych na zbiorach
\english{Big Data} \cite{big-data-3}:
\begin{itemize}
      \item Apache Hadoop -- platforma do przechowywania i przetwarzania danych,\newline
            oparta o system plików HDFS i model obliczeń MapReduce,
      \item Apache Spark –- platforma do przetwarzania wsadowego i strumieniowego,\newline
            wspiera obliczenia z wykorzystaniem rozproszonej pamięci współdzielonej,
      \item Apache Pig –- platforma do analizy danych oparta o system plików HDFS i klastry Apache Hadoop.
            Programy dla tej platformy tworzone są w języku Pig Latin, który oferuje wyższy poziom abstrakcji
            od programów MapReduce w języku Java.
\end{itemize}

Instalacja oprogramowania oraz konfiguracja klastra będącego przedmiotem badań to skomplikowane i
wymagające zadanie. Aby zebrać odpowiednią ilość informacji i odpowiednio ocenić wydajność systemu
kluczowe jest, aby metody zarządzania klastrem pozwalały na szybkie wprowadzanie zmian konfiguracji.

\subsection*{Skalowanie pionowe}

Skalowaniem pionowym nazywamy dodawanie zasobów do pojedynczego komputera w~celu zwiększenia jego
wydajności \cite{big-data-3}. Główną zaletą skalowania pionowego jest prostsza implementacja, która
nie narzuca kosztów związanych z konfiguracją dodatkowych maszyn i oprogramowania.

Takie rozwiązanie ma jednak swoje ograniczenia -- istnieje górny limit zasobów, które można dodać
do jednej maszyny, a odporność na awarie pojedynczego systemu może okazać się niewystarczająca
w niektórych zastosowaniach. Skalowanie pionowe można realizować poprzez wykorzystanie:
\begin{itemize}
      \item systemów wieloprocesorowych (ang. \english{Symmetric Multiprocessing, SMP}),
      \item procesorów wielordzeniowych,
      \item procesorów graficznych do obliczeń ogólnego przeznaczenia \newline
            (ang. \english{General-Purpose Compute on Graphics Processing Units, GPGPU}),
      \item programowalnych macierzy logicznych.
\end{itemize}

Wykorzystanie wymienionych metod do przyspieszania klastra \english{Big Data}
wymaga odpowiedniego doboru sprzętu oraz dogłębnego zrozumienia jego architektury. Współczesne
procesory wielordzeniowe i karty graficzne mają skomplikowane hierarchie pamięci, które
znacząco różnią się pod względem szybkości i pojemności \cite{computer-arch}. Optymalne
wykorzystanie tych hierarchii w programach będzie kluczowe dla uzyskania maksymalnej wydajności.

\section{Apache Hadoop}

Apache Hadoop to otwarte rozwiązanie do rozproszonego przechowywania i przetwarzania
dużych zbiorów danych, którego architektura została zainspirowana publikacjami firmy Google
na temat Google File System i MapReduce \cite{big-data-3}. Hadoop został zaprojektowany
do działania na klastrach złożonych z powszechnie dostępnego sprzętu, zapewniając przy
tym skalowalność i odporność na awarie.

Klastry Hadoop wykorzystują informacje o swojej topologii do planowania obliczeń
i~dzielenia danych tak, aby osiągnąć możliwie największą wydajność. Dane są zwykle
przetwarzane na tym samym komputerze, na którym są przechowywane, dzięki czemu
ograniczany jest przesył danych w obrębie klastra. \newpage

Klaster Apache Hadoop składa się z trzech głównych części:
\begin{enumerate}
      \item \english{Hadoop Distributed File System (HDFS)} –- rozproszony system plików dla
            platformy Hadoop. Pliki są dzielone na bloki (zazwyczaj o wielkości 128 MB lub 256 MB)
            i przechowywane w wielu egzemplarzach na różnych węzłach klastra (Rys. \ref{fig:hdfs}),
            zapewniając w ten sposób lokalność danych i odporność na awarie. System plików HDFS
            składa się z następujących komponentów:
            \begin{itemize}
                  \item NameNode –- węzeł zarządzający metadanymi i strukturą katalogów,
                  \item DataNode –- węzły przechowujące bloki danych i sumy kontrolne.
            \end{itemize}
      \item \english{MapReduce} –- domyślny model przetwarzania danych w Hadoop, wywodzący się
            z~systemów opracowanych przez Google. Na jego implementację w Apache Hadoop składają
            się następujące fazy (Rys. \ref{fig:mapreduce}):
            \begin{itemize}
                  \item \english{Map} –- przetwarzanie podzielonych danych wejściowych i utworzenie
                        par klucz-wartość, realizowane przez komponenty \english{Mapper},
                  \item \english{Shuffle/Sort} –- sortowanie par klucz-wartość i przesyłanie ich między węzłami,
                  \item \english{Reduce} –- łączenie wartości o tym samym kluczu, realizowane przez
                        komponenty o nazwie \english{Reducer}.
            \end{itemize}
            Użytkownicy definiują etapy \english{Map} i \english{Reduce}, natomiast etap sortowania
            dostarczany jest przez Hadoop.
      \item \english{Yet Another Resource Negotiator (YARN)} –- menedżer zasobów i system przydzielania
            zadań dla Hadoop, który pozwala wielu aplikacjom współdzielić zasoby w~jednym środowisku.
            Składa się z trzech typów węzłów:
            \begin{itemize}
                  \item ResourceManager –- odpowiedzialny za alokację zasobów w klastrze,
                  \item ApplicationMaster –- zarządza pojedynczym zadaniem oraz śledzi jego postęp,
                  \item NodeManager -– zarządza pojedynczym węzłem wykonawczym, komunikuje informacje
                        o dostępnych zasobach.
            \end{itemize}
\end{enumerate}

\begin{figure}
      \centering
      \includegraphics[width=0.7\textwidth]{./graf/HDFS.png}
      \caption{Schemat systemu plików HDFS}
      \label{fig:hdfs}
\end{figure}

\begin{figure}
      \centering
      \includegraphics[width=0.8\textwidth]{./graf/MapReduce.png}
      \caption{Schemat przykładowego programu \english{MapReduce}}
      \label{fig:mapreduce}
\end{figure}

Choć implementacje faz \english{Map} i \english{Reduce} są pisane głównie w języku Java, to
narzędzie \english{Hadoop Streaming} pozwala na użycie dowolnego programu, który komunikuje się
za pomocą standardowych strumieni wejścia/wyjścia. Wskutek tego możliwe jest tworzenie
komponentów \english{Mapper} oraz \english{Reducer} w dowolnym języku programowania.

Hadoop i model MapReduce zostały zaprojektowane głównie dla aplikacji wsadowych, co oznacza,
że są lepiej przystosowane do przetwarzania dużych, zamkniętych zbiorów danych niż do ciągłego
przetwarzania strumieniowego lub iteracyjnego.

\section{Apache Spark}

Apache Spark to otwarta platforma do rozproszonego przetwarzania dużych zbiorów danych,
która oferuje wsparcie dla programów wsadowych, obróbki danych w czasie rzeczywistym,
interaktywnej analizy danych i uczenia maszynowego. Spark został opracowany w celu pokonania
ograniczeń wynikających z szybkości dysków twardych.

Platforma Spark jest oparta o struktury danych \english{Resilient Distributed Dataset (RDD)}.
Są to obiekty przechowywane w pamięci węzłów klastra w wielu egzemplarzach, które mogą być
przetwarzane równolegle. Ich użycie pozwala na nawet stukrotne przyspieszenie w porównaniu z
systemami opartymi o Apache Hadoop, zachowując jednocześnie wysoką odporność na awarie \cite{big-data-3}.

Spark zapewnia wsparcie dla wielu źródeł danych. Są to między innymi HDFS, bazy Apache
Cassandra, relacyjne bazy danych SQL i źródła Amazon S3. System ten wspiera też wiele
języków programowania, są to na przykład Scala, Python, Java oraz R.

Klaster Apache Spark może działać niezależnie, w połączeniu z systemem plików HDFS lub
z innymi systemami zarządzania klastrem:
\begin{itemize}
      \item YARN -- menedżer zasobów dla platformy Apache Hadoop,
      \item Apache Mesos -- menedżer klastra zgodny z Hadoop oraz Spark,
      \item Kubernetes -- platforma do wdrażania i skalowania aplikacji kontenerowych.
\end{itemize}

\subsubsection*{Spark SQL}

Moduł Spark SQL umożliwia przetwarzanie danych w formacie tabelarycznym przy użyciu języka
SQL \cite{spark-sql}. Spark SQL zapewnia też dostęp do danych z Apache Hive i interfejsy
dostępu do relacyjnych baz danych (JDBC oraz ODBC), co ułatwia integrację z~istniejącymi
aplikacjami.

Jednym z kluczowych elementów Spark SQL jest DataFrame, czyli rozproszona kolekcja danych
zorganizowana w kolumny. DataFrame można przyrównać do tabeli w relacyjnej bazie danych,
ponieważ wspiera on wykonywanie operacji takich jak wybieranie, filtrowanie czy agregacja.

Wraz z DataFrame pojawiły się także struktury Dataset, które łączą zalety RDD i~DataFrame,
oferując zarówno typowanie statyczne, jak i operacje na poziomie kolumn. Dataset to kolekcja
obiektów określonego typu, co oznacza, że błędy związane z typami mogą być wykrywane w czasie
kompilacji, a nie tylko podczas wykonania.

Statyczne typowanie zapewnia większe bezpieczeństwo i pewność podczas pisania kodu w
porównaniu do operacji wykonywanych na DataFrame'ach, które są typowane dynamicznie.
Przykład wykorzystania Dataset pokazano na listingu \ref{lst:spark-sql}.
\newpage

%TC:ignore
\begin{lstlisting}[
    language=Java,
    label=lst:spark-sql,
    caption={Przykładowy program wykorzystujący moduł Spark SQL.}
]
public class SparkSQLExample {

    public static void main(String[] args) {
        SparkSession spark = SparkSession.builder()
            .appName("Spark SQL w Javie")
            .master("local")
            .getOrCreate();

        // Wczytanie danych z pliku CSV
        Dataset<Row> df = spark.read()
            .option("header", "true")
            .option("inferSchema", "true")
            .csv("osoby.csv");

        // Utworzenie widoku tymczasowego
        df.createOrReplaceTempView("osoby");

        // Wykonanie zapytania SQL
        Dataset<Row> wynik = spark.sql(
            "SELECT imie, wiek FROM osoby WHERE wiek > 25"
        );
        wynik.show(); // Pokazanie wyniku w konsoli
        spark.stop();
    }
}
\end{lstlisting}
%TC:endignore

Catalyst to optymalizator używany przez Spark SQL, który optymalizuje zapytania SQL na dwóch
poziomach: najpierw na poziomie planu logicznego, następnie na poziomie planu fizycznego
\cite{spark-sql-catalyst}. Plan logiczny opisuje tylko etapy potrzebne do obliczenia wyniku,
nie uwzględniając jednak struktury klastra. Informacje te posiada plan fizyczny, który składa
się z optymalnych struktur RDD i algorytmów, dopasowanych do konkretnego klastra Spark.

Ważną cechą optymalizatora Catalyst jest oddzielenie optymalizacji od logiki zapytania. Taki
podział ułatwia wdrażanie aktualizacji i modyfikacji, a same optymalizacje mogą być używane
w wielu zapytaniach jednocześnie, co wpływa korzystnie na długoterminowe utrzymanie całego
systemu. Co więcej, optymalizacje są dobierane dynamicznie, dzięki czemu to samo zapytanie
może być wykonywane optymalnie na klastrach o różnych rozmiarach, topologiach czy
dostępnych zasobach.

\section{Architektura CUDA}

CUDA (ang. \english{Compute Unified Device Architecture}) to jedna z najbardziej popularnych
równoległych architektur dla procesorów graficznych, opracowana przez firmę NVIDIA.

Architektura CUDA realizuje model SIMT \cite{computer-arch} (ang. \english{Single Instruction, Multiple Threads},
jedna instrukcja -- wiele wątków), który łączy zalety wielowątkowości i instrukcji wektorowych.
Podstawowym elementem tego modelu jest wątek CUDA, czyli najmniejsza jednostka obliczeń, która
może być niezależnie zaplanowana do wykonania przez procesor.

Podobnie jak w modelu SIMD (ang. \english{Single Instruction, Multiple Data}, jedna instrukcja -- wiele danych),
w architekturze CUDA instrukcje są wykonywane równolegle na wielu danych. Z drugiej strony,
w modelu SIMT każdy wątek posiada własny kontekst, składający się ze stosu i rejestrów.

Wątki CUDA są zorganizowane w grupy zwane blokami (Rys. \ref{fig:cuda-grid}), w których obrębie
mogą wymieniać się danymi przez pamięć współdzieloną i synchronizować się za pomocą
barier. Zarówno bloki wątków, jak i siatki składające się z tych bloków mogą być organizowane na
różne sposoby przy użyciu struktur jedno-, dwu- lub trójwymiarowych. Ta elastyczność pomaga w
tworzeniu wydajnych programów, które pozostają proste i łatwe w utrzymaniu.

\begin{figure}
      \centering
      \includegraphics[width=0.8\textwidth]{graf/CUDA-Grid.png}
      \caption {
            Podział przykładowego programu CUDA na siatkę (ang. \english{grid}),
            bloki wątków (ang. \english{thread block}) i wątki CUDA (ang. \english{thread}).
            \cite{computer-arch}
      }
      \medskip \small
      Program mnoży elementy tablic B i C, wynik jest zapisywany do tablicy A.
      \label{fig:cuda-grid}
\end{figure}

W architekturze CUDA hierarchia pamięci odgrywa kluczową rolę w efektywnym wykorzystaniu
układów graficznych. Model programowania CUDA udostępnia następujące rodzaje pamięci:
\begin{itemize}
      \item rejestry -- najszybsza dostępna pamięć, używana przez pojedyncze wątki, przechowuje
            zmienne lokalne. Liczba dostępnych rejestrów jest ograniczona i zależy od generacji procesora CUDA.
      \item pamięć współdzielona -- szybka pamięć procesora, dostępna dla wątków w obrębie jednego bloku.
            Ma ograniczoną pojemność, ale dostęp do niej jest znacznie szybszy niż do pamięci globalnej.
            Aby zapewnić maksymalną wydajność, ważne jest unikanie konfliktów banków tej pamięci.
      \item pamięć globalna -- zewnętrzna pamięć procesora, dostępna dla wszystkich bloków i~wątków.
            Jest największa pod względem pojemności, ale jednocześnie ma najwyższe opóźnienia dostępu.
            Wspiera dostępy łączone, które znacznie przyspieszają operacje czytania oraz zapisu.
      \item pamięć stałych -- specjalny segment pamięci globalnej o ograniczonej pojemności, który
            jest przechowywany w pamięci podręcznej. Przeznaczony dla parametrów funkcji oraz danych,
            które nie zmieniają się podczas wykonywania programu. \newpage
      \item pamięć tekstur -- zoptymalizowana dla specyficznych wzorców dostępu, typowych dla programów
            związanych z grafiką komputerową. Dane podlegają przechowywaniu w~pamięci podręcznej procesora.
      \item pamięć lokalna -- segment pamięci globalnej, który jest używany do przechowywania zmiennych
            lokalnych wątku, gdy przekroczy on dostępną liczbę rejestrów. Charakteryzuje się większymi
            opóźnieniami od pamięci współdzielonej, ponieważ jest fizycznie przechowywana w zewnętrznej pamięci DRAM.
\end{itemize}


Układy graficzne firmy NVIDIA oferują jednakowy model programowania i wykorzystują wspólny pakiet narzędzi
programistycznych, co ułatwia tworzenie przenośnych programów. Częścią modelu programowania jest język
programowania CUDA \cite{cuda-by-example}, który jest w istocie rozszerzeniem języka C++ o hierarchiczny
model pamięci urządzenia, funkcje do synchronizacji wątków, typy wektorowe oraz możliwość definiowania
funkcji wykonywanych na procesorze graficznym, określanych też jako \english{kernel}.

Kod napisany w języku CUDA może być kompilowany do przenośnego kodu pośredniego PTX
(ang. \english{Parallel Thread Execution}), co umożliwia jego późniejszą optymalizację i wykonanie
na urządzeniach o różnych wersjach architektury CUDA.

Dzięki szerokiej dostępności bibliotek, programy CUDA mogą być wywoływane także z poziomu
innych języków, takich jak Python, Java, MATLAB oraz Rust. Niektóre narzędzia, takie jak CudaPy
dla języka Python, pozwalają też na automatyczne tłumaczenie funkcji na kod CUDA.

\section{Standard OpenCL}

OpenCL (ang. \english{Open Computing Language}) to otwarty standard umożliwiający tworzenie programów
wykorzystujących akcelerację sprzętową, w tym różnorodne kombinacje procesorów głównych, graficznych oraz
innych akceleratorów \cite{opencl}.

W porównaniu z technologią CUDA, której rozwojem kieruje NVIDIA, OpenCL został stworzony przez konsorcjum
Khronos Group określające jedynie specyfikację interfejsów. Dostarczenie odpowiednich sterowników i
implementacji OpenCL jest z kolei zadaniem należącym do producenta sprzętu.

Ze względu na otwartość tego standardu, OpenCL umożliwia pisanie kodu, który może być łatwo przenoszony i
dostosowywany do różnych architektur i platform sprzętowych. Pozwala to na większą elastyczność w projektowaniu
systemów prowadzących obliczenia.

Jednostka obliczeń (ang. \english{work-item}), odpowiadająca wątkowi w modelu CUDA, stanowi najmniejsze
możliwą do zaplanowania i wykonania zadanie. Tak jak w modelu CUDA, jednostki obliczeń mogą być organizowane
w grupy robocze (ang. \english{workgroup}), które są odpowiednikami bloków wątków. 

Komunikacja w obrębie grupy roboczej jest realizowana poprzez pamięć lokalną. Mechanizmy synchronizacji, takie jak bariery, pozwalają
na koordynację wykonania jednostek w ramach grupy roboczej.

OpenCL oferuje programistom złożony, ale jednakowy dla wszystkich urządzeń model pamięci. Jego hierarchiczna
struktura składa się z następujących elementów:
\begin{itemize}
      \item pamięć prywatna -- pamięć dostępna tylko dla jednej jednostki obliczeń. Dane w pamięci prywatnej nie
            są dostępne dla innych jednostek wykonawczych. Zazwyczaj jest używana do przechowywania zmiennych
            lokalnych w funkcjach. Jest to ekwiwalent pamięci lokalnej w CUDA.
      \item pamięć lokalna -- dostępna dla wszystkich jednostek obliczeń w obrębie jednej grupy roboczej. Jest szybsza
            niż pamięć globalna i idealnie nadaje się do przechowywania tymczasowych danych. Podobna do pamięci
            współdzielonej w modelu CUDA.
      \item pamięć globalna -- pamięć dostępna dla wszystkich jednostek obliczeń w~obrębie kernela. Ze względu
            na jej globalny charakter, dostęp do niej może być wolniejszy w~porównaniu z innymi rodzajami pamięci,
            dlatego jej użycie powinno być ograniczane. Jest odpowiednikiem pamięci globalnej w CUDA.
      \item pamięć stałych -- specjalny segment pamięci globalnej, który może być odczytywany przez jednostki
            obliczeń. Na niektórych urządzeniach jest to część pamięci podręcznej. Odpowiednik pamięci stałych w CUDA.
\end{itemize}

Ze względu na konieczność zapewnienia wsparcia dla różnych typów procesorów, model pamięci OpenCL nie
zawiera pamięci tekstur, pozwalającej na optymalizację wzorców dostępu typowych dla aplikacji grafiki komputerowej.

Oprócz modelu obliczeń i pamięci, OpenCL opisuje też języki programowania przeznaczone dla
akceleratorów. Są to języki oparte o standardy C99, C++14 i C++17, rozszerzające je o funkcje
związane z zarządzaniem pamięcią, synchronizacją jednostek obliczeń oraz operacjami wektorowymi.

Podobnie jak w przypadku języka CUDA, kod OpenCL może być kompilowany do reprezentacji pośredniej.
W przypadku OpenCL jest to SPIR (ang. \english{Standard Portable Intermediate Representation}),
który przed wykonaniem programu jest optymalizowany dla architektury konkretnego urządzenia.
\newpage

\section{Klastry oparte o minikomputery jednopłytkowe}

Klastry obliczeniowe zbudowane z komputerów jednopłytkowych cechują się dużą skalowalnością.
Dobrym przykładem są systemy zbudowane przez firmy Balena i Oracle, które służą głównie jako
demonstratory technologii do rozproszonego przechowywania danych i~zarządzania serwerami,
składają się zatem z setek lub nawet tysięcy węzłów.

Klaster Beast v2 został skonstruowany przez firmę Balena jako następca wcześniejszej wersji,
Beast v1. Składa się on ze 144 minikomputerów Raspberry Pi 3B, połączonych siecią Gigabit
Ethernet. Konstrukcja ukazana na rysunku \ref{fig:balena-cluster} osiąga wysokość 2 metrów
i~masę około 150 kilogramów.

Klaster był pierwotnie przeznaczony do testowania i demonstracji oprogramowania resin.io,
a ostatecznie został wysłany do klienta, który sponsorował jego budowę. Wkrótce potem
rozpoczęto projektowanie kolejnej iteracji, nazwanej "Beast v3", która ma na celu dalsze
zwiększenie modułowości, gęstości i łatwości montażu \cite{balena-cluster}.

\begin{figure}[h]
      \centering
      \includegraphics[width=0.9\textwidth]{graf/Balena.jpg}
      \caption {
            Klaster Beast v2. \cite{balena-cluster}
      }
      \label{fig:balena-cluster}
\end{figure}

System firmy Oracle, nazwany \english{Raspberry Pi Supercomputer}, został zaprezentowany po raz
pierwszy na targach OpenWorld 2019 (Rys. \ref{fig:oracle-cluster}). Składa się on z 1060
minikomputerów Raspberry Pi 3B+ umieszczonych w szafach 2U, gdzie każda półka mieści 21 węzłów.
Węzły zostały połączone za pomocą sieci w standardzie Gigabit Ethernet.

Komputery Raspberry Pi zostały umieszczone w obudowach wykonanych w technologii druku 3D.
Zasilanie węzłów jest zapewniane przez zasilacze USB, które zostały wybrane ze względu na
mniejszy koszt i większą sprawność w porównaniu do zasilania przez interfejsy Ethernet
(ang. \english{Power over Ethernet, PoE}). Wdrożenie Oracle Linux zostało przeprowadzone
za pomocą centralnego serwera rozruchowego \cite{oracle-cluster-1}.

Przedstawiony klaster pokazał możliwości dystrybucji Oracle Linux, języka programowania Java
oraz bazy danych Oracle w kontekście integracji powszechnie dostępnych komponentów, takich jak
Raspberry Pi, w klastry obliczeniowe dużego formatu \cite{oracle-cluster-2}.

\begin{figure}[h]
      \centering
      \includegraphics[width=0.9\textwidth]{graf/Oracle.jpg}
      \caption {Klaster Raspberry Pi Supercomputer firmy Oracle. \cite{oracle-cluster-1}}
      \label{fig:oracle-cluster}
\end{figure}

Artykuł autorstwa A. Neto, J. Neto i E. Moreno \cite{rpi-cluster-1} opisuje klaster
\english{Big Data} wykorzystujący platformę Apache Hadoop i 9 minikomputerów Raspberry Pi 4B.
Minikomputery wykorzystują karty pamięci microSD jako dysk systemowy, a komunikacja
między węzłami klastra realizowana jest przez sieć Ethernet. Przepustowość tej sieci
nie została określona przez autorów.
\newpage

Oprócz opisu użytego sprzętu i oprogramowania artykuł zawiera również kompletny poradnik,
który przedstawia proces instalacji, konfiguracji oraz testowania wydajności. W artykule
są też załączone informacje na temat narzędzi wykorzystanych do testów, metodologii
testowania, użytych danych wejściowych i konfiguracji klastra.

Aby ocenić funkcjonalność i wydajność skonstruowanego klastra, autorzy wykorzystali narzędzia
TeraSort oraz TestDFSIO, które stanowią część oficjalnej dystrybucji Apache Hadoop.
Dla każdej liczby węzłów wykonawczych (2, 4, 8) zbadano wydajność dla 4 rodzajów danych
wejściowych. Wyniki zostały porównane pod kątem przyrostu wydajności w zależności
od liczby węzłów, nie przeprowadzono porównania z innymi klastrami.

Klaster opisany przez E. Lee, H. Oh oraz D. Park \cite{rpi-cluster-2} składa się z 6
minikomputerów Raspberry Pi 4B. W porównaniu do poprzedniego artykułu opisane jest
użycie platformy obliczeniowej Apache Spark. Węzły klastra zostały połączone siecią
Gigabit Ethernet.

Wydajność została zbadana nie tylko pod kątem liczby węzłów, ale także z uwzględnieniem
dysków systemowych o różnej wydajności. W przytoczonym badaniu są to karty pamięci
microSD o klasach szybkości UHS-1, UHS-3 oraz UFS.

Szerszy jest też zestaw przetestowanych programów -- są to TestDFSIO, Wordcount, TeraGen,
TeraSort oraz QuasiMonteCarlo. Wydajność klastra została porównana z klastrem Raspberry Pi
poprzedniej generacji oraz pojedynczym komputerem klasy PC.

W artykule poruszono też inne czynniki wpływające na wydajność klastra, są to między innymi
mechanizmy sum kontrolnych w HDFS, topologia rozproszonego systemu plików oraz chłodzenie
procesora. Przeprowadzona analiza opłacalności wykazała, że klaster Raspberry Pi złożony
z 7 węzłów byłby konkurencyjny z komputerem PC pod względem wydajności przy jednoczesnej
redukcji zużycia prądu.
 % [Analiza tematu]
\chapter{Budowa klastra obliczeniowego} \label{ch:budowa-klastra}

W przeszłości głównym zastosowaniem procesorów graficznych było przyspieszanie renderowania
grafiki, głównie w kontekście gier i graficznych interfejsów użytkownika. Wraz z ewolucją
tych układów, rozwinęła się ich zdolność do równoczesnego przetwarzania bardzo dużej liczby
wątków, co pozwala na znaczące przyspieszenie innych aplikacji o~wysokim stopniu równoległości.

Dobrym tego przykładem są nowoczesne układy graficzne, które udostępniają setki, a nawet
tysiące rdzeni dla obliczeń ogólnego przeznaczenia. Dzięki temu są stosunkowo tanim
narzędziem do równoległego przetwarzania danych na wielką skalę \cite{computer-arch}.

Aby zwiększyć wydajność proponowanego klastra względem istniejących rozwiązań, minikomputery
jednopłytkowe będące węzłami wykonawczymi powinny wspierać obliczenia ogólnego przeznaczenia
na układach graficznych (ang. \english{General-Purpose Compute on Graphics Processing Units, GPGPU}).
Szczególnie ważne w tym przypadku jest wsparcie dla popularnych narzędzi do programowania
GPGPU, takich jak CUDA oraz OpenCL.

Aby zapewnić efektywne przechowywanie, transfer i przetwarzanie danych, węzły klastra
powinny posiadać też:
\begin{itemize}
	\item dużą ilość szybkiej pamięci RAM,
	\item szybkie interfejsy dla pamięci trwałej (dyski HDD lub SSD),
	\item karty sieciowe o dużej przepustowości.
\end{itemize}

\section{Porównanie minikomputerów z akceleracją GPU}

Model Raspberry Pi 4, wydany w 2019 roku, przyniósł znaczące ulepszenia pod względem wydajności
w porównaniu do swoich poprzedników. Choć szybsze interfejsy sieciowe, większa ilość pamięci RAM
(nawet do 8 GB) oraz interfejsy USB 3.0 pozwoliły na znaczne zwiększenie wydajności w obliczeniach
\english{Big Data} \cite{rpi-cluster-2}, to brak wsparcia dla technologii CUDA i OpenCL czyni ten
model niewystarczającą podstawą dla węzłów wykonawczych.

Popularnym przedstawicielem minikomputerów jednopłytkowych jest też Intel NUC. Jednostki te
oferują wydajne procesory x86, wbudowane układy graficzne, obsługę szybkich dysków i dużą
rozszerzalność. Komputery NUC są dostępne zarówno w formie gotowych urządzeń, jak i zestawów
do samodzielnego złożenia, co pozwala na dostosowanie ich do konkretnych potrzeb użytkownika.
Najnowsze produkty z tej linii oferują wsparcie dla technologii OpenCL oraz akceleratory
sieci neuronowych, lecz nie wspierają technologii CUDA \cite{nuc-list}.

Nvidia Jetson to kolejna rodzina minikomputerów z akceleracją GPU, początkowo przeznaczona
przede wszystkim do rozwoju aplikacji graficznych 3D. Wkrótce komputery z tej serii stały
się też popularnym narzędziem do prototypowania aplikacji wizji komputerowej, uczenia
maszynowego i Internetu rzeczy.

Swoją popularność zawdzięczają wsparciu technologii CUDA i OpenCL, stosunkowo niskiemu
kosztowi, małym wymiarom oraz obszernemu pakietowi narzędzi Nvidia JetPack SDK.
Biorąc pod uwagę te zalety, do budowy klastra wybrano model Jetson Nano.

Zestaw Jetson Nano Developer Kit wyróżnia się \cite{jetson-spec}:
\begin{itemize}
	\item 128-rdzeniowym procesorem graficznym o maksymalnej wydajności 472 GFLOPS,
	      wspierającym technologie CUDA oraz (nieoficjalnie) OpenCL,
	\item czterordzeniowym procesorem ARM Cortex-A57 o taktowaniu 1.43GHz,
	\item 4 GiB pamięci RAM w technologii LPDDR4, o łącznej przepustowości 25.6 GB/s,
	\item kartą sieciową obsługującą standard Gigabit Ethernet,
	\item obsługą kart pamięci SD, pamięci Flash eMMC i dysków zewnętrznych USB 3 o przepustowości do 5 Gbit/s,
	\item konfigurowalnym poborem mocy (od 5 do 10W),
	\item zasilaniem poprzez port microUSB.
\end{itemize}

\section{Sprzęt i oprogramowanie}

\begin{figure}[h]
	\centering
	\includegraphics[width=0.7\textwidth]{graf/Cluster-picture.jpg}
	\caption{Klaster minikomputerów zbudowany na potrzeby pracy.}
	\medskip \small
	U góry, kolejno od lewej strony: minikomputer Raspberry Pi 4B, przełącznik sieciowy
	TP-Link oraz dysk zewnętrzny. Na dole widoczne minikomputery Nvidia Jetson Nano.
	\label{pic:cluster-photo}
\end{figure}

Klaster przedstawiony na fotografii \ref{pic:cluster-photo} składa się z czterech węzłów, w tym
jednego węzła nadrzędnego i trzech węzłów wykonawczych. Funkcję węzła nadrzędnego pełni
minikomputer Raspberry Pi 4B w wersji z 2 GB pamięci RAM, który został dodatkowo wyposażony w
dysk twardy USB 3.0 o pojemności 1 TB. Zasilanie minikomputera zapewniane jest przez
zasilacz USB-C dedykowany dla Raspberry Pi.
\newpage

Węzły wykonawcze klastra to minikomputery Nvidia Jetson Nano w wersji z 4 GB pamięci RAM, które
zasilane są przez ładowarkę USB firmy Aukey. Wybrana ładowarka dostarcza maksymalnie 12 W na
każdy z czterech portów, przez co każdy moduł Jetson Nano jest w stanie w pełni wykorzystać
limit poboru mocy.

Rolę dysków dla minikomputerów Jetson Nano pełnią karty microSD SanDisk o~pojemności 16 GB i
klasie szybkości UHS-1. Gwarantują one minimalną szybkość zapisu 10 MB/s i maksymalną
szybkość odczytu do 120 MB/s.

Zgodnie ze schematem przedstawionym na rysunku \ref{fig:cluster-network}, wszystkie węzły
klastra są połączone siecią Gigabit Ethernet poprzez przełącznik sieciowy TP-Link.
Aby ułatwić konfigurację klastra, adresy IP minikomputerów zostały przydzielone statycznie.

\begin{figure}[h]
	\centering
	\includegraphics[width=0.6\textwidth]{graf/Cluster-network.png}
	\caption{Schemat sieci komputerowej łączącej węzły klastra.}
	\label{fig:cluster-network}
\end{figure}

Węzeł nadrzędny klastra działa pod kontrolą systemu operacyjnego Ubuntu Server LTS w wersji 20.04.
W przypadku węzłów wykonawczych wykorzystywana jest dystrywbucja Nvidia Jetson Linux w wersji
R32.7.3 (Tab. \ref{tab:software}), która dostarcza sterowniki, biblioteki i narzędzia potrzebne do
rozwoju aplikacji CUDA na platformie Jetson.

Ze względu na brak oficjalnego wsparcia OpenCL dla minikomputera Jetson Nano wykorzystano
otwartą implementację PoCL. Dzięki zastosowaniu projektu LLVM do generacji kodu maszynowego
obsługuje ona wiele architektur, w tym x86, ARM oraz CUDA. Wersja PoCL zawarta w systemie
Jetson Linux domyślnie nie zawiera wsparcia dla architektury CUDA, dlatego też na potrzeby
pracy przygotowano własną kompilację.

Na klastrze zostały zainstalowane platformy Apache Hadoop oraz Apache Spark, a~także wymagane do
ich uruchomienia środowiska OpenJDK 11 i Scala 2.13.

%TC:ignore
\begin{table}[h]
	\centering
	\caption{Wykaz oprogramowania wykorzystanego w badaniach.}
	\begin{tabular}{ | p{7cm} | p{2cm} | p{3cm} | }
		\hline
		Nazwa programu lub biblioteki & Wersja   & Data wydania \\
		\hline
		Nvidia Jetson Linux           & R32.7.3  & 2022-11-22   \\
		Ubuntu Server                 & 20.04.6  & 2023-03-23   \\
		OpenJDK                       & 11.0.19  & 2023-04-18   \\
		Scala                         & 2.13.11  & 2023-06-07   \\
		CUDA Toolkit                  & 10.2.460 & 2021-03-02   \\
		PoCL                          & 3.0      & 2022-06-10   \\
		Apache Hadoop                 & 3.3.5    & 2023-03-22   \\
		Apache Spark                  & 3.4.0    & 2023-04-13   \\
		JCuda                         & 10.2.0   & 2020-02-09   \\
		Aparapi                       & 3.0.0    & 2021-07-12   \\
		Ansible                       & 2.15.5   & 2023-10-09   \\
		\hline
	\end{tabular}
	\label{tab:software}
\end{table}
%TC:endignore

Do implementacji programów testowych wybrano język Java 11, czyli statycznie typowany język
wysokiego poziomu wspierający programowanie obiektowe. Ponieważ jest on wspierany przez obie
platformy obliczeniowe, jego użycie ułatwia ponowne użycie kodu i~konstrukcję programów w
sposób modułowy.

Co więcej, Java jest językiem wieloplatformowym, co oznacza, że napisany i skompilowany kod może
być uruchamiany bez modyfikacji na różnych platformach, od telefonów komórkowych po systemy
mainframe. Programy w języku Java są kompilowane do niezależnego od platformy kodu bajtowego,
który jest następnie interpretowany lub kompilowany do kodu maszynowego przez maszynę wirtualną Javy.

Rysunek \ref{fig:software-stack} przedstawia ogólny widok stosu oprogramowania używanego w~badaniach.
Do wykonywania programów CUDA w programach Java wykorzystano bibliotekę JCuda. Są to bezpośrednie
wiązania do funkcji CUDA w~języku C++, które nie dostarczają dodatkowych warstw abstrakcji. Oprócz
podstawowego interfejsu JCuda używane jest rozszerzenie JCurand. Dostarcza ono wiązania dla
opracowanej przez firmę Nvidia biblioteki cuRAND, która zawiera wydajne implementacje generatorów
liczb losowych w~języku CUDA.

Oprócz generatorów liczb pseudolosowych, cuRAND oferuje też generator oparty o ciągi Sobola, który
cechuje się bardziej równomiernym próbkowaniem przestrzeni wielowymiarowych. Właściwość ta jest
szczególnie ważna w~aplikacjach metody Monte Carlo, ponieważ wpływa korzystnie na dokładność
otrzymywanych wyników.

Niestety, projekt JCuda nie dostarcza gotowych bibliotek natywnych dla architektury ARM,
wykorzystywanej przez minikomputery Jetson Nano. Z tego względu zostały przygotowane własne
kompilacje wymaganych zależności. Na jednym z węzłów klastra skompilowano projekt
\lstinline{jcuda-natives} zawierający komponenty interfejsu natywnego Javy (ang.
\english{Java Native Interface}). Następnie wykorzystując narzędzie Maven zbudowano i~opublikowano
w~\href{https://github.com/users/kmolski/packages?repo_name=masters-thesis}{repozytorium GitHub}
bibliotekę dla języka Java.

Aparapi to kolejna z używanych bibliotek, pozwalająca na wykorzystanie procesorów graficznych w
aplikacjach Java bez konieczności pisania kodu w języku specyficznym dla GPU. Główną zaletą
Aparapi jest zdolność do automatycznej konwersji kodu bajtowego Javy na kod OpenCL w czasie
wykonania. Jeśli program okaże się niemożliwy do przekształcenia, biblioteka automatycznie
cofa się do wykonania kodu na procesorze głównym.

\begin{figure}[h]
	\centering
	\includegraphics[width=1\textwidth]{graf/Stack.png}
	\caption{Schemat stosu oprogramowania używanego w badaniach.}
	\label{fig:software-stack}
\end{figure}

\section{Proces konfiguracji klastra}

Kluczowym elementem w budowie i ocenie wydajności klastra \english{Big Data} jest nie tylko
wybór odpowiedniego sprzętu, ale także zautomatyzowanie procesów związanych z instalacją
programów oraz konfiguracją środowiska. Zdolność do szybkiej zmiany konfiguracji pozwala na
sprawniejsze przeprowadzenie badań i dokładniejszą ocenę jego wydajności.

Wszystkie komputery zostały skonfigurowane za pomocą Ansible, czyli otwartego narzędzia
do administracji serwerami, zarządzania konfiguracją oraz wdrażania aplikacji.
\newpage

Narzędzie Ansible zostało wybrane ze względu na \cite{ansible-whitepaper}:
\begin{itemize}
	\item modułowość -- Ansible pozwala na organizację zadań w niezależne moduły
	      (nazywane też rolami), co ułatwia ponowne wykorzystanie i tworzenie złożonych zadań,
	\item idempotentność –- wielokrotne zastosowanie skryptów zawsze prowadzi do tego
	      samego stanu docelowego, zapewniając spójność konfiguracji w całym klastrze,
	\item standardowe protokoły -- Ansible komunikuje się z serwerami za pomocą
	      protokołu SSH, którego wsparcie zawarte jest w popularnych systemach operacyjnych,
	\item mechanizm inwentarzy –-  Ansible opisuje zarządzane serwery za pomocą plików w formacie
	      INI lub YAML, nazywanych też inwentarzem. W inwentarzu można tworzyć grupy złożone z
	      serwerów lub innych grup, a także zmienne przypisane do pojedynczego serwera czy grupy,
	\item mechanizm szablonów -- Ansible pozwala na generowanie plików konfiguracyjnych za pomocą
	      języka szablonów Jinja2, wspierającego użycie zmiennych, instrukcji warunkowych i pętli,
	\item wsparcie dla różnych platform –- Ansible działa na wszystkich popularnych
	      systemach operacyjnych i architekturach sprzętowych.
\end{itemize}

Ręczna część konfiguracji klastra ogranicza się do instalacji systemu operacyjnego
Jetson Linux. Po wgraniu obrazu systemu na kartę microSD i włączeniu minikomputera
uruchamiany jest instalator graficzny, w którym należy wykonać następujące czynności:
\begin{enumerate}
	\item Skonfigurować połączenie sieciowe oraz statyczny adres IP;
	\item Utworzyć konto użytkownika z hasłem;
	\item Wybrać tryb zarządzania energią "MAXN".
\end{enumerate}

Po zakończeniu pracy instalatora należy dodać statyczne adresy IP urządzeń do inwentarza
Ansible, a następnie za pomocą programu \lstinline{ansible-playbook} uruchomić kolejno
skrypty \lstinline{setup-os.yaml}, \lstinline{setup-pocl.yaml},
\lstinline{setup-hadoop.yaml} oraz \lstinline{setup-spark.yaml}.

\subsection*{Skrypty Ansible i narzędzia administracyjne}

W ramach pracy utworzono projekt Ansible, którego zadaniem jest automatyzacja procesów instalacji
programów oraz konfiguracji klastra. Skrypty Ansible wchodzące w~skład projektu zapewniają
niezawodność i powtarzalność procesów przy jednoczesnym zachowaniu czytelności
i łatwości użycia.

Kluczowym elementem każdego projektu Ansible jest inwentarz, który opisuje zbiór serwerów, na
którym mają zostać wykonane skrypty. W przypadku wspomnianego projektu inwentarz definiuje
statyczne adresy IP komputerów, podział węzłów na nadrzędne i wykonawcze, konfigurację
klienta Git, a także wersje instalowanych programów.

Kolejną częścią projektu są skrypty, które określają serię zadań do wykonania na węzłach
klastra. Zawierają one konkretne instrukcje opisane poniżej:
\begin{itemize}
	\item \lstinline{setup-os.yaml} -- przeprowadzenie wstępnej konfiguracji i aktualizacji systemu. \newline
	      Wyłączany jest tryb graficzny i ograniczenie czasu wykonania dla kerneli CUDA,
	\item \lstinline{setup-pocl.yaml} -- instalacja własnej kompilacji biblioteki PoCL,
	      dostarczającej nieoficjalne wsparcie OpenCL dla procesora graficznego w komputerze Jetson Nano,
	\item \lstinline{setup-hadoop.yaml} -- instalacja i konfiguracja platformy Apache Hadoop. \newline
	      W ramach tego skryptu formatowany jest też system plików HDFS,
	\item \lstinline{setup-spark.yaml} -- instalacja i konfiguracja platformy Apache Spark. \newline
	      W tym skrypcie instalowany jest również język programowania Scala.
\end{itemize}

Aby zwiększyć modułowość projektu, zgrupowano w role następujące zadania:
\begin{itemize}
	\item \lstinline{includes/updated} -- zadania związane z aktualizacją systemu operacyjnego,
	\item \lstinline{roles/common-tools} -- zadania do instalacji środowiska Java, konfiguracji
	      nazwy hosta, strefy czasowej oraz klienta Git,
	\item \lstinline{roles/security-config} -- konfiguracja zabezpieczeń usługi SSH.
\end{itemize}

Wykorzystując dane zawarte w inwentarzu i mechanizm szablonów udostępniany przez Ansible, skrypty
instalacyjne generują dla klastra następujące pliki konfiguracyjne:
\begin{itemize}
	\item \lstinline{core-site.xml} -- główna konfiguracja Hadoop, zawiera adres klastra HDFS,
	\item \lstinline{environment} -- konfiguracja zmiennej \lstinline{PATH}, służącej do lokalizacji programów,
	\item \lstinline{hdfs-site.xml} -- konfiguracja klastra HDFS zawierająca ustawienia replikacji danych,
	\item \lstinline{mapred-site.xml} -- konfiguracja komponentu MapReduce dla platformy Hadoop,
	\item \lstinline{yarn-site.xml} -- ustawienia menedżera zasobów YARN,
	\item \lstinline{workers} -- lista węzłów wykonawczych dla platformy Hadoop,
	\item \lstinline{spark-env.sh} -- konfiguracja platformy Spark do działania w trybie samodzielnym,
	\item \lstinline{slaves} -- lista węzłów wykonawczych dla platformy Spark.
\end{itemize}

Oprócz zadań administracyjnych, automatyzacji podlega także uruchamianie programów
testowych i zbieranie informacji o wydajności. Zapewnia to lepszą powtarzalność
pomiarów, a co za tym idzie -- większą przejrzystość i weryfikowalność badań.
Dlatego też do automatyzacji często wykonywanych zadań utworzono skrypty w języku Bash:
\begin{itemize}
	\item \lstinline{bench-*.sh} --
	      uruchamianie programów testowych dla wszystkich kombinacji
	      parametrów i danych wejściowych,
	\item \lstinline{start-hadoop.sh} i \lstinline{stop-hadoop.sh} --
	      uruchamianie i zamykanie platformy Apache Hadoop wraz z~systemem plików HDFS,
	\item \lstinline{start-spark.sh} i \lstinline{stop-spark.sh} --
	      uruchamianie i zamykanie platformy Apache Spark wraz z~systemem plików HDFS,
	\item \lstinline{clear-hdfs-out.sh} i \lstinline{clear-hdfs-tmp.sh} --
	      usuwanie plików wynikowych i plików tymczasowych programów,
	\item \lstinline{report-perf.sh} -- odczytywanie miar wydajności z
	      plików dziennika utworzonych przez programy testowe.
\end{itemize}

\section{Programy testowe}

Do badań wydajności klastra przygotowano zadania rozproszone w języku Java, które wykorzystują
platformy obliczeniowe Apache Hadoop i Apache Spark. Wszystkie programy testowe wspierają wykonanie
zarówno na procesorze głównym, jak i graficznym przy użyciu technologii CUDA. Jeden z algorytmów
został też zrealizowany przy użyciu standardu OpenCL.

Zestaw programów testowych składa się zarówno ze zmodyfikowanych wersji przykładów z dystrybucji
Hadoop oraz Spark, jak i nowych programów stworzonych na potrzeby pracy. Aby kompleksowo
ocenić wydajność klastra oraz opłacalność wykorzystania GPU w obliczeniach \english{Big Data},
zestaw testowy zawiera aplikacje o różnych charakterystykach użycia dysku i procesora.

Projekt zawierający programy testowe jest zarządzany przez Apache Maven, czyli narzędzie do
zarządzania projektami opartymi o język Java. Maven automatyzuje wiele etapów tworzenia
oprogramowania, takich jak kompilacja, testowanie, generowanie dokumentacji oraz budowanie
paczek JAR (ang. \english{Java ARchive}). Co więcej, centralne repozytorium Maven udostępnia
programistom wiele bibliotek dla języka Java, które mogą być dodawane do projektu poprzez
prostą zmianę konfiguracji.

Programy testowe zostały podzielone przy użyciu modułów Maven, czyli podprojektów wchodzących
w skład jednego projektu nadrzędnego. Moduły są kompilowane jako niezależne jednostki,
które mogą definiować zależności między sobą. Na przykład, moduł interfejsu konsolowego może
zależeć od modułu zawierającego logikę aplikacji.

Implementacje programów testowych zostały podzielone na trzy moduły:
\begin{itemize}
	\item moduł \lstinline{common} -- częściowe implementacje algorytmów w językach
	      Java oraz CUDA, funkcje ułatwiające wykonywanie programów CUDA i obsługę zasobów,
	\item moduł \lstinline{hadoop-benchmark} -- funkcje pomocnicze do zarządzania zadaniami w Apache Hadoop,
	      implementacje programów PiEstimation, FuzzyGen, FuzzyCompute oraz FuzzyFilter dla platformy Hadoop,
	\item moduł \lstinline{spark-benchmark} -- funkcje pomocnicze do zarządzania kontekstem w Apache Spark,
	      implementacje progrmaów PiEstimation, FuzzyGen, FuzzyCompute oraz FuzzyFilter dla platformy Spark.
\end{itemize}

Projekt nadrzędny może zawierać wspólną konfigurację, która może być następnie używana we
wszystkich modułach. Na przykład, zamiast określać wersje bibliotek w każdym module
osobno, można to zrobić jednokrotnie w nadrzędnym projekcie, co ułatwia aktualizację
zależności i zapewnia spójność ich wersji.

Maven posiada też wsparcie dla wtyczek, które umożliwiają dostosowywanie procesu budowy do
indywidualnych potrzeb projektu. W przypadku programów testowych wykorzystywane są wtyczki do
wykonywania narzędzia Make (wtyczka \lstinline{exec-maven-plugin}) oraz tworzenia obrazów aplikacji
wraz z wymaganymi do uruchomienia bibliotekami (wtyczka \lstinline{maven-assembly-plugin}).

\subsection{Program PiEstimation}

Pierwszym programem testowym jest PiEstimation, który wykorzystuje metodę Monte Carlo do szacowania
wartości liczby $\pi$. Wersja dla procesora głównego to rozwinięcie programu QuasiMonteCarlo
zawartego w oficjalnej dystrybucji Apache Hadoop, a jej odpowiedniki dla procesora graficznego
to autorskie konwersje w językach CUDA oraz OpenCL.

Metoda Monte Carlo to metoda numeryczna stosowana do rozwiązywania problemów poprzez próbkowanie
losowe. Oznacza to, że przeprowadzana jest symulacja wielu prób (lub "eksperymentów") w celu uzyskania
rozkładu wyników, których średnia może być następnie użyta do oszacowania oczekiwanej wartości rozwiązania.

Warianty tej metody są stosowane w wielu dziedzinach nauki, takich jak fizyka, matematyka, finanse,
czy medycyna. Jej przykładowe zastosowania to obliczanie wartości całek, szacowanie wartości
oczekiwanej, optymalizacja funkcji, symulacja sieci, symulacja zdarzeń dyskretnych i wiele innych.

Szacowanie liczby $\pi$ za pomocą metody Monte Carlo to klasyczny przykład zastosowania tej techniki.
Wykonuje się w nim losowe próbkowanie punktów wewnątrz kwadratu i określa ich położenie względem
wpisanego w ten kwadrat koła, co ilustruje rysunek \ref{fig:monte-carlo}.

\begin{figure}[h]
	\centering
	\includegraphics[width=0.8\textwidth]{graf/Pi-estimation.png}
	\caption{Rozkład wartości losowych używany do estymacji liczby $\pi$.}
	\medskip \small
	Czarne punkty znajdują się poza kołem, a punkty niebieskie -- wewnątrz koła.
	\label{fig:monte-carlo}
\end{figure}

\subsubsection*{Implementacja metody Monte Carlo}

Rozważmy kwadrat o boku długości 2, którego środek znajduje się w punkcie (0,0). W ten kwadrat wpisywane
jest koło o promieniu 1, również o środku w punkcie (0,0). Powierzchnia kwadratu wynosi więc 4, a
powierzchnia koła jest równa $\pi$.

Następnie wybieramy losowo punkty o losowych współrzędnych X i Y z zakresu [-1, 1], czyli znajdujących
się wewnątrz kwadratu. Dla każdego punktu sprawdzamy, czy leży on wewnątrz koła. Robimy to, obliczając
jego odległość od środka kwadratu (i koła) za pomocą wzoru \(d = x^2 + y^2\). Jeśli odległość
jest mniejsza lub równa 1, punkt leży wewnątrz koła, a w przeciwnym razie leży poza nim.

Po przeprowadzeniu wielu takich prób obliczamy stosunek liczby punktów wewnątrz koła do
całkowitej liczby punktów wewnątrz kwadratu. Wraz ze zwiększającą się liczbą wybranych
punktów stosunek punktów wewnątrz okręgu do punktów wewnątrz kwadratu będzie coraz
bardziej zbliżony do stosunku ich powierzchni, czyli \(\pi / 4\).

Programy komputerowe realizujące metodę Monte Carlo najczęściej używają generatora liczb
pseudolosowych, którego jakość, wraz z liczbą wybranych punktów, jest głównym czynnikiem
wpływającym na dokładność wyniku.

Użyty w programie generator liczb pseudolosowych oparty jest o ciąg Haltona, czyli
technikę generowania ciągów opartą o rozwinięcia w systemach liczbowych o różnych podstawach.
W przeciwieństwie do tradycyjnych sekwencji losowych sekwencje te zapewniają bardziej
jednolite próbkowanie przestrzeni, co ogranicza błąd obliczeń \cite{monte-carlo}.

Algorytm generacji punktów z użyciem sekwencji Haltona jest nastepujący:
\begin{enumerate}
	\item Dla każdego wymiaru wybierz unikalną liczbę pierwszą \( p \),
	\item Dla każdej liczby pierwszej \( p \) i kolejnych liczb \( i \),
	      wygeneruj rozwinięcia liczb \( i \) o~podstawie \( p \),
	\item Przekształć rozwinięcia liczb \( i \) na wartość ułamkową, aby uzyskać
	      elementy sekwencji dla tej podstawy,
	\item Połącz powstałe sekwencje w celu uzyskania wielowymiarowych punktów.
\end{enumerate}

Ponieważ wartości generowane przez sekwencję są stałe i zależą tylko od podstaw sekwencji,
testowanie implementacji w różnych językach sprowadzało się do porównania końcowych wyników.

Program dla platformy Hadoop wykorzystuje następujące komponenty \english{Mapper},
które zostały opisane w załączniku \ref{ch:techdoc}:
\begin{itemize}
	\item CpuQmcMapper (listing \ref{ch:cpu-qmc-mapper}) -- implementacja dla procesora głównego,
	\item AparapiQmcMapper (listing \ref{ch:aparapi-qmc-mapper}) -- implementacja w języku Java wykorzystująca bibliotekę Aparapi i OpenCL,
	\item JcudaQmcMapper (listing \ref{ch:jcuda-qmc-mapper}) -- implementacja algorytmu w języku CUDA.
\end{itemize}

Ich odpowiednikami dla platformy Spark są klasy CpuQmcFunction, AparapiQmcFunction oraz JcudaQmcFunction (listingi \ref{ch:cpu-qmc-function}, \ref{ch:aparapi-qmc-function} oraz
\ref{ch:jcuda-qmc-function}).

\subsubsection*{Specyfikacja zewnętrzna}

Programy dla platform Apache Hadoop oraz Apache Spark oczekują trzech argumentów wiersza poleceń:
\begin{itemize}
	\item \lstinline{nMaps} -- liczba niezależnych zadań używanych w estymacji,
	\item \lstinline{nSamples} -- liczba próbek przetwarzanych w jednym zadaniu,
	\item \lstinline{mapper} -- implementacja algorytmu (jedna z wartości "cpu", "opencl" lub "cuda").
\end{itemize}

Program wyświetla w konsoli liczbę próbek, algorytm, czas wykonania oraz oszacowaną wartość liczby $\pi$, co przedstawiono na rysunku \ref{fig:piestimation:run}.
Jeśli argumenty nie zostały dostarczone lub są nieprawidłowe, aplikacja wyświetli komunikat o
błędzie wraz z instrukcją poprawnego użycia.

\begin{figure}[h]
	\centering
	\includegraphics[width=0.8\textwidth]{graf/PiEstimation-interface.png}
	\caption{Przykładowe wywołanie programu PiEstimation.}
	\label{fig:piestimation:run}
\end{figure}

Aplikacja dla platformy Hadoop zapisuje pliki tymczasowe w katalogach \lstinline{/tmp} i \lstinline{/out}
systemu plików HDFS, które są usuwane po zakończeniu zadania.

\subsection{Program FuzzyGen}

Zadaniem kolejnego programu testowego jest generowanie zbiorów danych o różnej objętości dla programu
FuzzyCompute. Dane programu FuzzyCompute składają się z~wielu porcji, z~których każda zawiera 64
zbiory. Każdy zbiór zawiera stopnie przynależności dla 4 rekordów, w~zakresie od 0 do 1.

Porcje danych są zapisywane jako pliki binarne w formacie SequenceFile, dostarczanym przez platformę
Hadoop. Pliki SequenceFile przechowują pary klucz-wartość w skompresowanej formie, co pozwala
zmniejszyć zapotrzebowanie na przestrzeń dyskową i~zwiększyć efektywną szybkość systemu plików.

Algorytm dla procesora głównego wykorzystuje generator liczb losowych z biblioteki standardowej języka Java,
który jest oparty o liniowe generatory kongruentne. Generator ten jest oparty o deterministyczny algorytm i
zapewnia stosunkowo dobrą wydajność, nie jest jednak odpowiedni do zastosowań kryptograficznych.

Wersja algorytmu dla procesora graficznego wykorzystuje bibliotekę JCurand, która udostępnia
funkcjonalność biblioteki cuRAND w języku Java. Biblioteka cuRAND jest z kolei częścią zestawu narzędzi
deweloperskich CUDA, dostarczającą funkcje do generowania liczb losowych na procesorach graficznych
firmy NVIDIA.
\newpage

cuRAND oferuje kilka typów generatorów liczb pseudolosowych, są to między innymi:
\begin{itemize}
	\item generator XORWOW,
	\item generator MRG32K3A,
	\item generator Mersenne Twister (MT19937),
	\item generator Philox 4x32.
\end{itemize}

Za pomocą generatorów cuRAND możliwe jest generowanie liczb dla różnych rozkładów prawdopodobieństwa,
takich jak rozkład Poissona, jednostajny lub normalny.

Program Hadoop wykorzystuje następujące klasy komponentów \english{Mapper},
które zostały opisane w załączniku \ref{ch:techdoc}:
\begin{itemize}
	\item CpuFuzzyGenMapper (listing \ref{ch:cpu-fuzzygen-mapper}) -- algorytm dla procesora głównego,
	\item JcudaFuzzyGenMapper (listing \ref{ch:jcuda-fuzzygen-mapper}) -- algorytm używający funkcji z biblioteki JCurand.
\end{itemize}

W programie dla Apache Spark wykorzystywane są klasy CpuFuzzyGenFunction i~JcudaFuzzyGenFunction (listingi \ref{ch:cpu-fuzzygen-function} oraz \ref{ch:jcuda-fuzzygen-function}).

\subsubsection*{Specyfikacja zewnętrzna}

Aplikacje dla Apache Hadoop oraz Apache Spark oczekują trzech argumentów wiersza poleceń:
\begin{itemize}
	\item \lstinline{mapper} -- definiujący wykorzystywany algorytm (do wyboru: "cpu" lub "cuda"),
	\item \lstinline{nRecords} -- liczba porcji danych do wygenerowania,
	\item \lstinline{outputDir} -- katalog HDFS, w którym zostaną zapisane pliki wynikowe.
\end{itemize}

Program wyświetla w konsoli liczbę porcji danych, algorytm oraz czas wykonania, co przedstawiono
na rysunku \ref{fig:fuzzygen:run}. Gdy wymagane argumenty nie zostaną podane lub są błędne,
program pokazuje komunikat o~błędzie oraz instrukcję prawidłowego użycia.

\begin{figure}[h]
	\centering
	\includegraphics[width=0.8\textwidth]{graf/FuzzyGen-interface.png}
	\caption{Przykładowe wywołanie programu FuzzyGen.}
	\label{fig:fuzzygen:run}
\end{figure}

Zarówno program dla platformy Hadoop, jak i~odpowiednik dla platformy Spark zapisuje dane
wyjściowe w formacie SequenceFile bez użycia kompresji. Zadanie Hadoop definiuje tylko
etap \english{Map}, przez co pliki wynikowe są zapisywane w kilku częściach.

\subsection{Program FuzzyCompute}

Kolejny program jest testem wydajności dla operacji wykonywanych na zbiorach rozmytych.
Zbiory rozmyte stanowią centralny element logiki rozmytej i zostały zdefiniowane przez
Lotfi Zadeha jako rozwinięcie tradycyjnej teorii zbiorów. Umożliwiają one reprezentację
niejasności i wieloznaczności w klasyfikacji obiektów \cite{fuzzy-sets}.

Zbiory rozmyte znalazły szerokie zastosowanie w różnych dziedzinach, takich jak sterowanie
rozmyte, systemy ekspertowe, analiza ryzyka oraz przetwarzanie obrazów. Znacząco ułatwiają
one modelowanie sytuacji, w których granice decyzyjne są trudne do określenia.

W klasycznej teorii zbiorów element albo należy do zbioru (i ma przynależność równą 1),
albo do niego nie należy (i ma przynależność równą 0). W zbiorach rozmytych stopień
przynależności elementu jest opisany przez funkcję przynależności, która przyjmuje wartość
liczbową w zakresie od 0 do 1.

Rozważmy zbiór przedmiotów pochodzących z różnych epok. Używając klasycznej teorii zbiorów,
możemy określić jedynie to, że dany przedmiot albo jest antykiem, albo nim nie jest. Jeśli
wiek przedmiotu znajdzie się tuż pod przyjętym progiem, to zostanie sklasyfikowany jako nie-antyk.

Jeżeli jednak opiszemy te same przedmioty za pomocą zbiorów rozmytych, możemy uzyskać bardziej
użyteczną klasyfikację. Przykładowo, funkcja przedstawiona na rysunku \ref{fig:fuzzy-fn} przypisuje
20-letni przedmiot do zbioru antyków z przynależnością 0, 250-letni -- z~przynależnością 0.5, a
500-letni -- z przynależnością wynoszącą 1.

\begin{figure}[h]
	\centering
	\includegraphics[width=0.6\textwidth]{graf/Fuzzy-membership.png}
	\caption{Wykres przykładowej funkcji przynależności dla zbioru rozmytego.}
	\label{fig:fuzzy-fn}
\end{figure}

Podobnie jak w klasycznej teorii zbiorów, na zbiorach rozmytych definiowane są różne operacje,
takie jak suma, iloczyn, czy dopełnienie. Podczas gdy koncepcja tych operacji pozostaje podobna,
ich realizacje różnią się od tych znanych z klasycznej teorii zbiorów, ponieważ uwzględniają
rozmyte podejście do przynależności.

Suma zbiorów rozmytych może być zdefiniowana jako maksimum stopni przynależności każdego elementu do
obu zbiorów. Niech \(A\) i \(B\) będą zbiorami rozmytymi w przestrzeni \(X\) z funkcjami
przynależności odpowiednio \(\mu A\) i \(\mu B\). Suma zbiorów \(A\) i \(B\) jest zbiorem rozmytym
\(C\) z funkcją przynależności \(\mu C\), która jest definiowana następująco dla każdego elementu
w przestrzeni \(X\) za pomocą wzoru \ref{eq:fuzzy-sum}.

\begin{equation}
    \mu C(x) = max(\mu A(x), \mu B(x))
    \label{eq:fuzzy-sum}
\end{equation}

Dla każdego elementu \(x\) stopień przynależności do sumy \(C\) jest większym z dwóch stopni
przynależności do zbiorów \(A\) i \(B\). Taką operację binarną, używaną do modelowania operacji
"lub" w kontekście zbiorów rozmytych, nazywamy T-konormą lub S-normą. Do S-norm można zaliczyć
też sumę algebraiczną, sumę drastyczną, czy sumę Łukasiewicza.

Dopełnienie zbioru rozmytego polega na odwróceniu stopnia przynależności każdego elementu w
danym zbiorze rozmytym. Jeśli mamy zbiór rozmyty \(A\) na przestrzeni uniwersalnej \(X\) z
funkcją przynależności \(\mu A\), dopełnienie tego zbioru, oznaczone jako \(A'\) lub \(\lnot A\),
ma funkcję przynależności \(\mu A'\) określoną dla każdego elementu \(x\) z \(X\)
wzorem \ref{eq:fuzzy-compl}.

\begin{equation}
    \mu A'(x) = 1 - \mu A(x)
    \label{eq:fuzzy-compl}
\end{equation}

Aplikacje dla platform Hadoop oraz Spark przetwarzają podziały zbiorów wygenerowanych przez program
FuzzyGen. Składają się one z 64 zbiorów, z~których każdy zawiera cztery liczby losowe z~zakresu od
0 do 1. Po wczytaniu zbiorów rozmytych, do zbioru danych dodawane są ich dopełnienia.
Następnie wszystkie zbiory rozmyte są sumowane krzyżowo, co zwiększa złożoność obliczeniową programu.

Aplikacja dla platformy Hadoop wykorzystuje następujące komponenty \english{Mapper},
które zostały opisane w załączniku \ref{ch:techdoc}:
\begin{itemize}
	\item CpuFuzzyComputeMapper (listing \ref{ch:cpu-fuzzycompute-mapper}) -- algorytm dla procesora głównego,
	\item JcudaFuzzyComputeMapper (listing \ref{ch:jcuda-fuzzycompute-mapper}) -- algorytm używający kerneli w języku CUDA.
\end{itemize}

Ich odpowiednikami dla programu Apache Spark są klasy CpuFuzzyComputeFunction i JcudaFuzzyComputeFunction (listingi \ref{ch:cpu-fuzzycompute-function} oraz \ref{ch:jcuda-fuzzycompute-function}).

\subsubsection*{Specyfikacja zewnętrzna}

Dla programów opartych o Apache Hadoop i Apache Spark konieczne jest podanie trzech parametrów wiersza poleceń, które zostały również przedstawione na rysunku \ref{fig:fuzzycompute:run}:
\begin{itemize}
	\item \lstinline{mapper} -- określający typ algorytmu (do wyboru "cpu" lub "cuda"),
	\item \lstinline{inputDir} -- katalog HDFS z plikami wejściowymi,
	\item \lstinline{outputDir} -- katalog HDFS, w którym zostanie zapisany wynik działania programu.
\end{itemize}

Program prezentuje w konsoli wykorzystywany algorytm oraz czas jego działania. W~przypadku podania
nieprawidłowych parametrów program poinformuje o~błędzie i~przedstawi poprawną składnię parametrów.

\begin{figure}[h]
	\centering
	\includegraphics[width=0.8\textwidth]{graf/FuzzyCompute-interface.png}
	\caption{Przykładowe wywołanie programu FuzzyCompute.}
	\label{fig:fuzzycompute:run}
\end{figure}

Formatem plików wejściowych i wyjściowych dla programu jest SequenceFile.

\subsection{Program FuzzyFilter}

Ostatni z omawianych programów testowych implementuje filtrowanie rekordów z plików CSV przy użyciu
zbiorów rozmytych. Filtrowanie danych z użyciem zbiorów rozmytych polega na zastosowaniu logiki
rozmytej do określenia stopnia przynależności elementów, a następnie usunięciu rekordów, których
stopień przynależności znajduje się poniżej ustawionego progu.

Tradycyjne metody klasyfikacji przyjmują, że obiekt należy do jednej konkretnej klasy. W podejściu
opartym na zbiorach rozmytych obiekt może należeć do wielu klas jednocześnie z różnym stopniem
przynależności. Pozwala to na lepsze uwzględnienie niepewności i niejednoznaczności w danych,
dzięki czemu klasyfikacja jest bardziej elastyczna i lepiej odwzorowuje rzeczywistość.

Funkcje przynależności dla zbiorów rozmytych mogą przyjmować różne kształty, które pozwalają na opisanie
stopnia przynależności w zależności od charakteru danych i problemu. Oto kilka najpopularniejszych
typów funkcji przynależności \cite{fuzzy-sets}:
\begin{itemize}
	\item funkcja impulsowa -- funkcja ta ma wartość 1 dla konkretnego punktu w zbiorze uniwersalnym i
	      0 dla wszystkich innych punktów,
	\item funkcja trójkątna -- określona przez trzy punkty: punkt początkowy, punkt szczytowy i
	      punkt końcowy. Wartość funkcji przynależności wzrasta liniowo od punktu początkowego do punktu
	      szczytowego, a następnie maleje liniowo do punktu końcowego,
	\item funkcja trapezoidalna -- podobna do funkcji trójkątnej, ale z dodatkowym płaskim segmentem dla
	      wartości 1. Zdefiniowana jest przez cztery punkty: punkt początkowy, punkt początku segmentu
	      płaskiego, punkt końca segmentu płaskiego i punkt końcowy,
	\item funkcja Gaussa -- funkcja dzwonowa zdefiniowana przez średnią i odchylenie standardowe.
	      Pozwala na modelowanie stopniowej przynależności, która jest maksymalna w środku i maleje w
	      kierunku wartości krańcowych.
\end{itemize}

Program pozwala na definiowanie relacji przynależności na podstawie funkcji trapezoidalnej. Zaletą tej
funkcji jest jej elastyczność w modelowaniu różnych relacji przynależności. W zależności od wybranych punktów,
określających kształt trapezu, można osiągnąć różne kształty funkcji przynależności.

Ponieważ funkcja trójkątna jest specyficznym przypadkiem funkcji trapezoidalnej, użytkownik może zdefiniować
ją poprzez podanie takich samych wartości dla obu punktów definiujących segment płaski trapezu. Podobne
rozwiązanie umożliwia określenie funkcji impulsowej. Podanie tej samej wartości dla wszystkich parametrów
trapezu powoduje, że powstała funkcja przyjmuje wartość 1 tylko dla jednego punktu.

Podana przez użytkownika lista trapezoidalnych funkcji przynależności jest łączona za pomocą T-normy,
która reprezentuje operację koniunkcji w~dziedzinie zbiorów rozmytych. Przykładowe T-normy to minimum,
produkt algebraiczny oraz norma Łukasiewicza. W programie wykorzystano najprostszą z~nich, czyli
T-normę minimum, która dla dwóch stopni przynależności \( a \) i \( b \) z~zakresu [0, 1] jest
określana przez wzór \ref{eq:fuzzy-tnorm}.

\begin{equation}
    T_{\text{min}}(a, b) = \min(a, b)
    \label{eq:fuzzy-tnorm}
\end{equation}

T-norma minimum ma następujące właściwości:
\begin{itemize}
    \item monotoniczność -- minimum jest funkcją monotoniczną. Oznacza to, że jeżeli mamy dwie pary liczb,
          gdzie \( a \leq c \) i \( b \leq d \), to zawsze \( T_{\text{min}}(a, b) \leq T_{\text{min}}(c, d) \),
    \item idempotentność -- T-norma minimum jest idempotentna, co w praktyce oznacza, że \( T_{\text{min}}(a, a) \)
          zawsze równa się \( a \) dla dowolnej liczby \( a \),
    \item element neutralny --  dla funkcji minimum elementem neutralnym jest liczba 1. Innymi słowy,
          \( T_{\text{min}}(a, 1) \) zawsze równa się \( a \) dla dowolnego \( a \) z przedziału [0, 1],
    \item komutatywność -- oznacza to, że funkcja minimum zwraca ten sam wynik niezależnie od kolejności
          argumentów, czyli \( T_{\text{min}}(a, b) \) jest równa \( T_{\text{min}}(b, a) \)
          dla dowolnych liczb \( a \) i \( b \),
    \item łączność -- funkcja minimum jest również łączna. W praktyce oznacza to, że dla dowolnych
          liczb \( a \), \( b \) i \( c \) z przedziału [0, 1], \( T_{\text{min}}(a, T_{\text{min}}(b, c)) \)
          jest równa \( T_{\text{min}}(T_{\text{min}}(a, b), c) \).
\end{itemize}

Program dla platformy Spark wykorzystuje moduł Spark SQL, który został wybrany ze względu na zdolność
do operowania na plikach CSV bez definiowania schematu danych. Umożliwia to przetwarzanie danych o różnej
strukturze bez konieczności zmiany programu. Rozmyte filtrowanie rekordów jest implementowane przez klasy opisane w załączniku \ref{ch:techdoc}:
\begin{itemize}
    \item CpuFuzzyFilterFunction (listing \ref{ch:cpu-fuzzyfilter-function}) -- implementacja dla procesora głównego,
    \item JcudaFuzzyFilterFunction (listing \ref{ch:jcuda-fuzzyfilter-function}) -- implementacja używająca algorytmu w języku CUDA.
\end{itemize}

\subsubsection*{Specyfikacja zewnętrzna}

Program Spark oczekuje co najmniej czterech argumentów wiersza poleceń, których przykładowe użycie
zostało przedstawione na rysunku \ref{fig:fuzzyfilter:run}:
\begin{itemize}
	\item \lstinline{mapper} -- implementacja algorytmu (jeden z "cpu" lub "cuda"),
	\item \lstinline{inputFile} -- plik wejściowy w formacie CSV,
	\item \lstinline{outputFile} -- ścieżka pliku w HDFS, do którego zostaną zapisane rekordy,
	\item \lstinline{threshold} -- wartość progowa dla funkcji przynależności rekordu,
    \item \lstinline{filters} -- parametry określające trapezoidalną funkcję przynależności.
\end{itemize}

Składnia parametru \lstinline{filters} składa się z pięciu części:
\begin{itemize}
	\item punktu początkowego,
	\item punktu początku segmentu płaskiego,
	\item punktu końca segmentu płaskiego,
	\item punktu końcowego,
    \item kolumny, dla której obliczany jest stopień przynależności
\end{itemize}

Program wyświetla w konsoli wybrany algorytm, czas wykonania oraz liczbę rekordów spełniających
zadany warunek. W przypadku braku argumentów lub ich niepoprawności, aplikacja poinformuje użytkownika
o~błędzie i~przedstawi instrukcję właściwego użycia.

\begin{figure}[h]
	\centering
	\includegraphics[width=0.8\textwidth]{graf/FuzzyFilter-interface.png}
	\caption{Przykładowe wywołanie programu FuzzyFilter.}
	\label{fig:fuzzyfilter:run}
\end{figure}

Format plików odczytywanych i~zapisywanych przez program FuzzyFilter to CSV.
 % [Przedmiot pracy]
\chapter{Badania} \label{ch:badania}

\section{Metodyka badań}

Jednym z głównych celów pracy było zbadanie wydajności klastra do obliczeń \english{Big Data}
zbudowanego z~minikomputerów wspierających akcelerację GPU. Dla klastra zostały opracowane zadania
rozproszone z~wykorzystaniem dla platform Apache Hadoop oraz Apache Spark, które pełnią funkcję
testów wydajności.

Przeprowadzono analizę czasu wykonania każdego programu testowego dla zbiorów danych o różnej
wielkości, wykorzystując zarówno procesory główne, jak i~graficzne. Pozwoliło to na dokładniejszą
ocenę skalowalności klastra oraz poszczególnych aplikacji. Dodatkowo dla platformy Hadoop zostały
zebrane dane o~czasie przeznaczonym na obliczenia procesora głównego oraz czasy wykonania faz
MapReduce, które dają dodatkowy wgląd w proces wykonania programów na klastrze.

Aby zapewnić odpowiednią dokładność pomiaru czasu, programy testowe zostały uruchomione wielokrotnie
dla każdego zbioru danych i~implementacji algorytmu. Na potrzeby pracy były to co najmniej 4
uruchomienia każdej konfiguracji. Następnie z czasów cząstkowych obliczona została średnia
arytmetyczna i~odchylenie standardowe.

Szczególnej uwagi wymagały testy systemów opartych o maszynę wirtualną Javy. Wykorzystywana
w~badaniach implementacja HotSpot zawiera kompilator Just-In-Time, który optymalizuje kod podczas
jego wykonania. Kompilator ten monitoruje wykonanie programu i~identyfikuje często używane fragmenty
kodu, które są następnie przekształcane z~kodu bajtowego Javy do szybszego kodu maszynowego.

Optymalizacje przeprowadzane podczas wykonania programu generują pewne problemy związane
z~pomiarami wydajności. Przy wielokrotnym uruchamianiu programu często można zaobserwować zmieniające
się wyniki, które stabilizują się po pewnym czasie. Aby uzyskać stabilną wydajność, konieczne jest
kilkukrotne uruchomienie testowanych programów przed dokonaniem pomiarów.
\newpage

Podobnym zachowaniem wykazują się programy CUDA w~formacie PTX, który jest w~istocie przenośnym
językiem asemblerowym. Przy pierwszym uruchomieniu na danej karcie graficznej, program PTX jest
kompilowany do kodu maszynowego specyficznego dla procesora graficznego, a~następnie zapisywany
na dysku dla kolejnych wywołań.

Wydajność klastra minikomputerów Jetson Nano została porównana z~pojedynczym komputerem stacjonarnym
wyposażonym w~procesor AMD Ryzen 5600X, 16 GB pamięci RAM oraz dysk twardy o pojemności 2 TB.
Dodatkowo dla programu PiEstimation umieszczono porównanie z~klastrem Raspberry Pi 4B, opisanym
w~artykule \cite{rpi-cluster-2}.

\section{Program PiEstimation}

Program PiEstimation charakteryzuje się stosunkowo wysokim użyciem mocy obliczeniowej i~niską liczbą
operacji odczytu/zapisu. Szacowanie wartości liczby $\pi$ odbywa się wyłącznie na podstawie sekwencji
punktów z ciągu Haltona, przez co charakterystyka wydajności programu jest bardziej podobna do aplikacji
metod numerycznych niż do typowych zadań \english{Big Data}.

W ramach badań przeprowadzono pomiar czasu wykonania programu na platformach Hadoop oraz Spark.
Każda implementacja algorytmu została przetestowana dla różnych wartości liczby próbek. Badania
skupiały się na trzech wartościach: 1 miliona próbek, 100 milionów próbek oraz 1 miliarda próbek.

Dla ostatniej wartości, czyli 1 miliarda próbek, przeprowadzono dodatkowe analizy podziału zadań.
W pierwszej z~nich przetwarzanie próbek było podzielone na 10 zadań, z~których każde zawierało 100 milionów
próbek. W drugim scenariuszu podział obejmował aż 100 zadań, które składały się z 10 milionów próbek.

Po analizie czasów wykonania dla 1 miliona próbek okazało się, że badany klaster osiągnął gorsze wyniki
od komputera stacjonarnego i~klastra Raspberry Pi opisanego w~artykule \cite{rpi-cluster-2}. Implementacje
algorytmu w~językach Java i~CUDA okazały się najszybszymi dla klastra Hadoop złożonego z~minikomputerów
Jetson Nano (Rys. \ref{fig:piestimation:hadoop:1M}), osiągając trzykrotnie dłuższy czas wykonania w
porównaniu z komputerem stacjonarnym.

W przypadku platformy Spark wydajność implementacji w języku Java i CUDA była bardzo podobna, jednak nawet
najlepszy czas dla klastra Jetson Nano był ponad pięć razy większy od czasu wykonania na
komputerze stacjonarnym (Rys. \ref{fig:piestimation:spark:1M}).

\begin{figure}[h]
    \centering
    \begin{tikzpicture}
        \begin{axis}[
                ybar,
                width=0.8\textwidth,
                height=0.4\textheight,
                symbolic x coords={Pi 4B, CPU, OpenCL, CUDA, PC},
                xtick={Pi 4B, CPU, OpenCL, CUDA, PC},
                bar width=25pt,
                ymin=0,
                ylabel={Czas wykonania [s]}
            ]
            \addplot [error bars/.cd, y dir=both, y explicit]
            table [col sep=comma, x=Cluster, y=Avg, y error=STDev]{data/pi-1M-hadoop.csv};
        \end{axis}
    \end{tikzpicture}
    \caption{Czasy wykonania PiEstimation dla platformy Hadoop i 1 miliona próbek.}
    \medskip \small
    Etykiety CPU, OpenCL oraz CUDA odpowiadają wynikom dla klastra Jetson Nano.
    \label{fig:piestimation:hadoop:1M}
\end{figure}

\begin{figure}[h!]
    \centering
    \begin{tikzpicture}
        \begin{axis}[
                ybar,
                width=0.8\textwidth,
                height=0.4\textheight,
                symbolic x coords={CPU, OpenCL, CUDA, PC},
                xtick={CPU, OpenCL, CUDA, PC},
                bar width=25pt,
                ymin=0,
                ylabel={Czas wykonania [s]}
            ]
            \addplot [error bars/.cd, y dir=both, y explicit]
            table [col sep=comma, x=Cluster, y=Avg, y error=STDev]{data/pi-1M-spark.csv};
        \end{axis}
    \end{tikzpicture}
    \caption{Czasy wykonania PiEstimation dla platformy Spark i 1 miliona próbek.}
    \medskip \small
    Etykiety CPU, OpenCL oraz CUDA odpowiadają wynikom dla klastra Jetson Nano.
    \label{fig:piestimation:spark:1M}
\end{figure}

Dla algorytmu opartego o standard OpenCL wyniki były mniej obiecujące. Przy wartości 1~miliona próbek
nie udało się osiągnąć konkurencyjnej wydajności. Co więcej, próba uruchomienia programu OpenCL z~większą
liczbą próbek kończyła się przerywaniem zadań Hadoop i~Spark ze względu na zbyt duże użycie pamięci.
\newpage

Przy wartości 100 milionów próbek program Hadoop wykorzystujący CUDA był o~18\% szybszy od implementacji
na procesor główny, jednak w porównaniu z~komputerem stacjonarnym klaster Jetson Nano okazał się ponad
dwa razy wolniejszy (Rys. \ref{fig:piestimation:hadoop:100M}).

\begin{figure}[h]
    \centering
    \begin{tikzpicture}
        \begin{axis}[
                ybar,
                width=0.8\textwidth,
                height=0.4\textheight,
                symbolic x coords={Jetson/CPU, Jetson/CUDA, PC},
                xtick={Jetson/CPU, Jetson/CUDA, PC},
                bar width=35pt,
                ymin=0,
                ylabel={Czas wykonania [s]}
            ]
            \addplot [error bars/.cd, y dir=both, y explicit]
            table [col sep=comma, x=Cluster, y=Avg, y error=STDev]{data/pi-100M-hadoop.csv};
        \end{axis}
    \end{tikzpicture}
    \caption{Czasy wykonania PiEstimation dla platformy Hadoop i 100 milionów próbek.}
    \label{fig:piestimation:hadoop:100M}
\end{figure}

Dużo większe przyspieszenie zanotowano dla programu na platformę Spark. W~jego przypadku czas wykonania
algorytmu CUDA został skrócony ponad dwukrotnie w stosunku do implementacji na CPU i~był konkurencyjny
względem pojedynczego komputera stacjonarnego (Rys. \ref{fig:piestimation:spark:100M}).

\begin{figure}[h!]
    \centering
    \begin{tikzpicture}
        \begin{axis}[
                ybar,
                width=0.8\textwidth,
                height=0.4\textheight,
                symbolic x coords={Jetson/CPU, Jetson/CUDA, PC},
                xtick={Jetson/CPU, Jetson/CUDA, PC},
                bar width=35pt,
                ymin=0,
                ylabel={Czas wykonania [s]}
            ]
            \addplot [error bars/.cd, y dir=both, y explicit]
            table [col sep=comma, x=Cluster, y=Avg, y error=STDev]{data/pi-100M-spark.csv};
        \end{axis}
    \end{tikzpicture}
    \caption{Czasy wykonania PiEstimation dla platformy Spark i 100 milionów próbek.}
    \label{fig:piestimation:spark:100M}
\end{figure}
\newpage

W przypadku platformy Hadoop można zauważyć znaczny spadek użytego czasu procesora i~czasu spędzonego
w~fazie Map (Tab. \ref{tab:piestimation:mapreduce:100M}). Co ciekawe, niższy okazał się też czas spędzony
w~fazie Reduce, której obliczenia są wykonywane tylko na procesorze głównym. Pozorne przyspieszenie wynikało
z~faktu równoległego wykonania faz Map i~Reduce, przez co czas oczekiwania na etap Map był włączany do
czasu trwania etapu Reduce.

%TC:ignore
\begin{table}[h]
    \centering
    \caption{Czas procesora i wykonania faz PiEstimation dla 100 milionów próbek.}
    \begin{tabular}{ | l | r | r | r | }
        \hline
                              & Jetson (CPU) & Jetson (CUDA) & PC     \\
        \hline
        Czas CPU [ms]         & 775 065       & 243 322        & 143 557 \\
        Czas fazy Map [ms]    & 4 828 542      & 3 904 967       & 330 540 \\
        Czas fazy Reduce [ms] & 194 061       & 163 824        & 63 677  \\
        \hline
    \end{tabular}
    \label{tab:piestimation:mapreduce:100M}
\end{table}
%TC:endignore

Dla wartości 1 miliarda próbek wyniki były jeszcze bardziej imponujące. Program CUDA dla platformy Hadoop
osiągnął pięciokrotne przyspieszenie w porównaniu do implementacji na procesor główny (Rys. \ref{fig:piestimation:hadoop:1B}).

\begin{figure}[h]
    \centering
    \begin{tikzpicture}
        \begin{axis}[
                ybar,
                width=0.8\textwidth,
                height=0.4\textheight,
                symbolic x coords={Jetson/CPU, Jetson/CUDA, PC},
                xtick={Jetson/CPU, Jetson/CUDA, PC},
                bar width=20pt,
                ymin=0,
                ylabel={Czas wykonania [s]}
            ]
            \addplot [error bars/.cd, y dir=both, y explicit]
            table [col sep=comma, x=Cluster, y=Avg, y error=STDev]{data/pi-1B-hadoop.csv};
        \end{axis}
    \end{tikzpicture}
    \caption{Czasy wykonania PiEstimation dla platformy Hadoop}
    \medskip \small
    Podział 10x100 milionów próbek.
    \label{fig:piestimation:hadoop:1B}
\end{figure}

Dla platformy Spark przyrost wydajności był jeszcze bardziej zauważalny, ponieważ uzyskany czas wykonania
był ponad osiem razy krótszy (Rys. \ref{fig:piestimation:spark:1B}).

\begin{figure}[h!]
    \centering
    \begin{tikzpicture}
        \begin{axis}[
                ybar,
                width=0.8\textwidth,
                height=0.35\textheight,
                symbolic x coords={Jetson/CPU, Jetson/CUDA, PC},
                xtick={Jetson/CPU, Jetson/CUDA, PC},
                bar width=20pt,
                ymin=0,
                ylabel={Czas wykonania [s]}
            ]
            \addplot [error bars/.cd, y dir=both, y explicit]
            table [col sep=comma, x=Cluster, y=Avg, y error=STDev]{data/pi-1B-spark.csv};
        \end{axis}
    \end{tikzpicture}
    \caption{Czasy wykonania PiEstimation dla platformy Spark}
    \medskip \small
    Podział 100x10 milionów próbek.
    \label{fig:piestimation:spark:1B}
\end{figure}
\newpage

Co więcej, klaster minikomputerów Jetson Nano po raz pierwszy uzyskał lepszy czas wykonania od testowanego
komputera stacjonarnego. Na platformie Hadoop program wykonał się około 36\% szybciej, a~na platformie
Spark -- ponad dwukrotnie szybciej.

Kolejne obserwacje dotyczyły wpływu podziału zadań na czas wykonania. W przypadku implementacji
na procesor główny i~programów Spark szybszy okazał się podział na 100 mniejszych zadań
(Rys. \ref{fig:pisplit:spark:1B} oraz \ref{fig:pisplit:hadoop:1B}).

\begin{figure}[h!]
    \centering
    \begin{tikzpicture}
        \begin{axis}[
                ybar,
                width=1\textwidth,
                height=0.35\textheight,
                symbolic x coords={CPU (100x10M), CUDA (100x10M), CPU (10x100M), CUDA (10x100M)},
                xtick={CPU (100x10M), CUDA (100x10M), CPU (10x100M), CUDA (10x100M)},
                bar width=20pt,
                ymin=0,
                ylabel={Czas wykonania [s]}
            ]
            \addplot [error bars/.cd, y dir=both, y explicit]
            table [col sep=comma, x=Cluster, y=Avg, y error=STDev]{data/pi-1B-spark-b.csv};
        \end{axis}
    \end{tikzpicture}
    \caption{Czasy wykonania PiEstimation dla Spark wg. podziału próbek.}
    \medskip \small
    Etykiety CPU oraz CUDA odpowiadają wynikom dla klastra Jetson Nano.
    \label{fig:pisplit:spark:1B}
\end{figure}
\newpage

Z~drugiej strony, program Hadoop wykorzystujący implementację
w~środowisku CUDA uzyskał lepszy wynik przy podziale na 10 większych zadań (Rys. \ref{fig:pisplit:hadoop:1B}).

\begin{figure}[h]
    \centering
    \begin{tikzpicture}
        \begin{axis}[
                ybar,
                width=1\textwidth,
                height=0.4\textheight,
                symbolic x coords={CPU (100x10M), CUDA (100x10M), CPU (10x100M), CUDA (10x100M)},
                xtick={CPU (100x10M), CUDA (100x10M), CPU (10x100M), CUDA (10x100M)},
                bar width=20pt,
                ymin=0,
                ylabel={Czas wykonania [s]}
            ]
            \addplot [error bars/.cd, y dir=both, y explicit]
            table [col sep=comma, x=Cluster, y=Avg, y error=STDev]{data/pi-1B-hadoop-b.csv};
        \end{axis}
    \end{tikzpicture}
    \caption{Czasy wykonania PiEstimation dla Hadoop wg. podziału próbek.}
    \medskip \small
    Etykiety CPU oraz CUDA odpowiadają wynikom dla klastra Jetson Nano.
    \label{fig:pisplit:hadoop:1B}
\end{figure}

\subsection*{Optymalizacje}

Algorytm w języku CUDA podzielony jest na dwa kernele. Pierwszy z~nich generuje punkty na podstawie liczb
z~ciągu Haltona i~sprawdza ich przynależność do okręgu, a~następnie zapisuje wynik porównania do tablicy (listing \ref{lst:cuda-qmc-kernel}. Każdy wątek CUDA zajmuje się przetwarzanie pojedynczego punktu.

Drugi kernel (listing \ref{lst:cuda-reduction}) operuje na tablicy przynależności, obliczając na jej podstawie całkowitą liczbę punktów
zawartych w okręgu.

Kernel do obliczania liczby punktów w okręgu wykorzystuje pamięć współdzieloną, dostępną dla wątków
w~obrębie jednego bloku. Jest ona szybsza niż globalna pamięć urządzenia, ale ma ograniczoną pojemność.
Wykorzystanie pamięci współdzielonej pozwoliło na znaczące przyspieszenie tej operacji.
\newpage

%TC:ignore
\begin{lstlisting}[
    language=C,
    label=lst:cuda-qmc-kernel,
    caption={Kernel CudaQmcKernel realizujący próbkowanie punktów.}
]
constexpr int BASES[] = {2, 3};
constexpr int MAX_DIGITS[] = {63, 40};
__device__ float get_random_point(const long index, float q[], int d[], int base, int max_digits) {
    float value = 0.0f;
    long k = index;
    for (int i = 0; i < max_digits; i++) {
        q[i] = (i == 0 ? 1.0f : q[i - 1]) / base;
        d[i] = (int) (k % base);
        k = (k - d[i]) / base;
        value += d[i] * q[i];
    }
    for (int i = 0; i < max_digits; i++) {
        d[i]++;
        value += q[i];
        if (d[i] < base) {
            break;
        }
        value -= (i == 0 ? 1.0f : q[i - 1]);
    }
    return value;
}
extern "C"
__global__ void qmc_mapper(short guesses[], const int64_t count, const int64_t offset) {
    const int32_t index = blockIdx.x * blockDim.x + threadIdx.x;
    if (index < count) {
        const int64_t index_offset = index + offset;
        float q[64] = {0.0f};
        int d[64] = {0};
        const float x = get_random_point(index_offset, q, d, BASES[0], MAX_DIGITS[0]) - 0.5f;
        const float y = get_random_point(index_offset, q, d, BASES[1], MAX_DIGITS[1]) - 0.5f;
        guesses[index] = (x * x + y * y > 0.25f) ? 1 : 0;
    }
}
\end{lstlisting}
%TC:endignore

%TC:ignore
\begin{lstlisting}[
    language=C,
    label=lst:cuda-reduction,
    caption={Kernel CudaReduction implementujący zliczanie punktów.}
]
extern "C"
__global__ void reduce_short(const short input[],
        short output[], const int64_t count) {
    extern __shared__ short shared_data[];
    const int32_t tid = threadIdx.x;
    const int32_t index = blockIdx.x * blockDim.x + tid;
    if (index < count) {
        shared_data[tid] = input[index];
    }
    __syncthreads();
    for (int32_t s = blockDim.x / 2; s > 0; s >>= 1) {
        if (tid < s) {
            shared_data[tid] += shared_data[tid + s];
        }
        __syncthreads();
    }
    if (tid == 0) {
        output[blockIdx.x] = shared_data[0];
    }
}
\end{lstlisting}
%TC:endignore

Podjęto inne próby optymalizacji kernela do zliczania punktów, które opierały się na:
\begin{itemize}
    \item dostosowaniu rozmiaru siatki i bloków,
    \item wykonywaniu wielu operacji dodawania w każdym wątku,
    \item wykonywaniu ostatniego etapu zliczania na procesorze graficznym,
    \item wykorzystaniu synchronizacji wątków w tej samej osnowie (ang. \english{warp}), co pozwoliłoby
          na zmniejszenie liczby barier używanych do synchronizacji.
\end{itemize}

Opisane optymalizacje nie spowodowały znaczącego przyspieszenia algorytmu w języku CUDA, ponieważ
dla testowanych parametrów udział wybranego kernela w ogólnym czasie wykonania był zbyt mały.

W drugiej kolejności optymalizowano kernel realizujący próbkowanie w~ramach metody Monte Carlo.
Z~generatora liczb losowych wydzielono część obliczeń niezależną od rozpatrywanego punktu,
którą następnie przeniesiono do etapu kompilacji, zapisując jej wyniki w~pamięci stałych.
Niestety, wspomniana optymalizacja również nie pociągnęła za sobą wzrostu wydajności.

\section{Program FuzzyGen}

Program FuzzyGen charakteryzuje się stosunkowo niskim użyciem mocy obliczeniowej i~liczbą operacji odczytu.
W~tym przypadku głównym czynnikiem wpływającym na wydajność jest szybkość zapisu danych. Jedynym
parametrem wejściowym algorytmów jest liczba generowanych rekordów, więc charakterystyka wydajności
programu jedynie częściowo odpowiada realnym aplikacjom \english{Big Data}.

Aplikacje stworzone dla platform Hadoop i~Spark zostały poddane testom wydajności, dla których parametr
liczby generowanych porcji zbiorów przyjmował wartości 10 tysięcy porcji oraz 100 tysięcy porcji.

W przypadku mniejszego zestawu, zawierającego 10 tysięcy porcji, implementacje na procesory główne
i~graficzne miały zbliżone wyniki pod względem czasu wykonania (Rys. \ref{fig:fuzzygen:hadoop:10K}
oraz \ref{fig:fuzzygen:spark:10K}). Różnice w wydajności implementacji nie przekraczały 4 procent,
lecz w~obu przypadkach klaster minikomputerów Jetson Nano okazał się wolniejszy od komputera
stacjonarnego.

\begin{figure}[h!]
    \centering
    \begin{tikzpicture}
        \begin{axis}[
                ybar,
                width=0.8\textwidth,
                height=0.4\textheight,
                symbolic x coords={Jetson/CPU, Jetson/CUDA, PC},
                xtick={Jetson/CPU, Jetson/CUDA, PC},
                bar width=20pt,
                ymin=0,
                ylabel={Czas wykonania [s]}
            ]
            \addplot [error bars/.cd, y dir=both, y explicit]
            table [col sep=comma, x=Cluster, y=Avg, y error=STDev]{data/fuzzygen-10K-hadoop.csv};
        \end{axis}
    \end{tikzpicture}
    \caption{Czasy wykonania FuzzyGen dla platformy Hadoop i 10 tysięcy porcji.}
    \label{fig:fuzzygen:hadoop:10K}
\end{figure}

\begin{figure}[h]
    \centering
    \begin{tikzpicture}
        \begin{axis}[
                ybar,
                width=0.8\textwidth,
                height=0.4\textheight,
                symbolic x coords={Jetson/CPU, Jetson/CUDA, PC},
                xtick={Jetson/CPU, Jetson/CUDA, PC},
                bar width=20pt,
                ymin=0,
                ylabel={Czas wykonania [s]}
            ]
            \addplot [error bars/.cd, y dir=both, y explicit]
            table [col sep=comma, x=Cluster, y=Avg, y error=STDev]{data/fuzzygen-10K-spark.csv};
        \end{axis}
    \end{tikzpicture}
    \caption{Czasy wykonania FuzzyGen dla platformy Spark i 10 tysięcy porcji.}
    \label{fig:fuzzygen:spark:10K}
\end{figure}

Podczas generowania zbioru o~objętości 100 tysięcy porcji na platformie Hadoop zauważono niewielkie
różnice w~czasie wykonania (Rys. \ref{fig:fuzzygen:hadoop:100K}). Implementacja w~języku CUDA była
około 2\% szybsza od wersji na procesor główny. Co ciekawe, czas potrzebny na wygenerowanie większego
zestawu danych nie różnił się znacząco od czasu przeznaczonego na wygenerowanie 10 tysięcy rekordów.

\begin{figure}[h!]
    \centering
    \begin{tikzpicture}
        \begin{axis}[
                ybar,
                width=0.8\textwidth,
                height=0.4\textheight,
                symbolic x coords={Jetson/CPU, Jetson/CUDA, PC},
                xtick={Jetson/CPU, Jetson/CUDA, PC},
                bar width=20pt,
                ymin=0,
                ylabel={Czas wykonania [s]}
            ]
            \addplot [error bars/.cd, y dir=both, y explicit]
            table [col sep=comma, x=Cluster, y=Avg, y error=STDev]{data/fuzzygen-100K-hadoop.csv};
        \end{axis}
    \end{tikzpicture}
    \caption{Czasy wykonania FuzzyGen dla platformy Hadoop i 100 tysięcy porcji.}
    \label{fig:fuzzygen:hadoop:100K}
\end{figure}

Większą różnicę zaobserwowano podczas generowania 100 tysięcy porcji danych programem dla platformy Spark
-- w~tym przypadku różnica wynosiła prawie 8\% na korzyść algorytmu w~języku CUDA (Rys. \ref{fig:fuzzygen:spark:100K}).

Mimo uzyskanego wzrostu wydajności klaster oparty o~minikomputery Jetson Nano znów okazał się
wolniejszy od pojedynczego komputera stacjonarnego.
Co ciekawe, w~przypadku platformy Hadoop implementacje dla CPU i~GPU nie różniły się znacząco pod
względem zużytego czasu procesora oraz czasu spędzonego w~fazie Map (Tab. \ref{tab:fuzzygen:mapreduce:100K}).

\begin{figure}[h!]
    \centering
    \begin{tikzpicture}
        \begin{axis}[
                ybar,
                width=0.8\textwidth,
                height=0.35\textheight,
                symbolic x coords={Jetson/CPU, Jetson/CUDA, PC},
                xtick={Jetson/CPU, Jetson/CUDA, PC},
                bar width=20pt,
                ymin=0,
                ylabel={Czas wykonania [s]}
            ]
            \addplot [error bars/.cd, y dir=both, y explicit]
            table [col sep=comma, x=Cluster, y=Avg, y error=STDev]{data/fuzzygen-100K-spark.csv};
        \end{axis}
    \end{tikzpicture}
    \caption{Czasy wykonania FuzzyGen dla platformy Spark i 100 tysięcy porcji.}
    \label{fig:fuzzygen:spark:100K}
\end{figure}

%TC:ignore
\begin{table}[h]
    \centering
    \caption{Czas procesora i wykonania faz FuzzyGen dla 100 tysięcy porcji.}
    \begin{tabular}{ | l | r | r | r | }
        \hline
                           & Jetson (CPU) & Jetson (CUDA) & PC   \\
        \hline
        Czas CPU [ms]      & 11 600        & 11 477         & 1 907 \\
        Czas fazy Map [ms] & 23 990        & 23 983         & 3 422 \\
        \hline
    \end{tabular}
    \label{tab:fuzzygen:mapreduce:100K}
\end{table}
%TC:endignore

\subsection*{Optymalizacje}

Pierwsza wersja programu dla procesora graficznego używała domyślnego generatora liczb pseudolosowych
biblioteki cuRAND, czyli generatora XORWOW. W~ramach optymalizacji programu sprawdzono też czas wykonania dla 
innych generatorów. Podane czasy zostały zmierzone podczas generowania zbiorów o~objętości 100 tysięcy porcji danych.

Używając domyślnego generatora liczb pseudolosowych o~nazwie XORWOW, udało się wygenerować zbiór danych
o~objętości 100 tysięcy porcji w~relatywnie krótkim czasie, wynoszącym 16.5 sekundy (Tab. \ref{tab:fuzzygen:generators}).

Generator MT19937 okazał się wolniejszy od domyślnego generatora XORWOW. Zmierzony czas wykonania okazał
się większy o~ponad 60\%. Czas potrzebny na wygenerowanie 100 tysięcy porcji wynosił 27.6 sekundy.

%TC:ignore
\begin{table}[h]
    \centering
    \caption{Czasy wykonania FuzzyGen dla platformy Spark i 100 tysięcy porcji.}
    \begin{tabular}{ | l | r | }
        \hline
        Generator liczb losowych & Czas wykonania [s] \\
        \hline
        XORWOW                   & 16.562             \\
        MT19937                  & 27.601             \\
        MRG32K3A                 & 16.012             \\
        MTGP32                   & 15.328             \\
        Philox 4x32              & 15.216             \\
        \hline
    \end{tabular}
    \label{tab:fuzzygen:generators}
\end{table}
%TC:endignore

Kolejny generator, czyli MRG32K3A osiągnął wydajność bardzo zbliżoną do generatora XORWOW. W~jego
przypadku zbiór danych został wygenerowany w~czasie wynoszącym dokładnie 16 sekund.

Pierwszy znaczący wzrost wydajności zaobserwowano dla generatora MTGP32. Czas potrzebny na wygenerowanie
zbioru 100 tysięcy porcji danych wyniósł 15.3 sekundy, co oznaczało poprawę wydajności o~7\% w porównaniu z
domyślnym generatorem XORWOW.

Ostatni z~analizowanych generatorów, czyli Philox 4x32, okazał się najszybszym z~całej grupy
(Tab. \ref{tab:fuzzygen:generators}). Wygenerował on dane składające się ze 100 tysięcy porcji zbiorów
w~zaledwie 15.2 sekundy. Jest to wynik o~8\% lepszy niż w przypadku generatora XORWOW.

\section{Program FuzzyCompute}

Program FuzzyCompute charakteryzuje się przeciętnym użyciem mocy obliczeniowej oraz wysoką liczbą 
operacji odczytu i~zapisu. Pod względem wykorzystania zasobów program jest więc podobny do typowych
aplikacji \english{Big Data}.

Podobnie jak w~przypadku programu FuzzyGen, przetestowano programy przeznaczone dla platform Hadoop i~Spark.
Testy zostały przeprowadzone dla dwóch zestawów danych, których wielkość wynosiła odpowiednio
10 tysięcy i~100 tysięcy porcji danych.

W przypadku programu Hadoop i~zestawu danych składającego się z~10 tysięcy porcji, dla implementacji
na procesor główny i~graficzny uzyskano zbliżone wyniki (Rys. \ref{fig:fuzzycompute:hadoop:10K}).
Różnice między wynikami były stosunkowo małe i~nie przekraczały 3 procent.

\begin{figure}[h]
    \centering
    \begin{tikzpicture}
        \begin{axis}[
                ybar,
                width=0.8\textwidth,
                height=0.4\textheight,
                symbolic x coords={Jetson/CPU, Jetson/CUDA, PC},
                xtick={Jetson/CPU, Jetson/CUDA, PC},
                bar width=20pt,
                ymin=0,
                ylabel={Czas wykonania [s]}
            ]
            \addplot [error bars/.cd, y dir=both, y explicit]
            table [col sep=comma, x=Cluster, y=Avg, y error=STDev]{data/fuzzycompute-10K-hadoop.csv};
        \end{axis}
    \end{tikzpicture}
    \caption{Czasy wykonania FuzzyCompute dla platformy Hadoop i 10 tys. porcji.}
    \label{fig:fuzzycompute:hadoop:10K}
\end{figure}
\newpage

Dla programu Spark testowanego z~tym samym zbiorem danych, algorytm korzystający z~procesora graficznego
był wolniejszy od odpowiednika działającego na CPU o~około 7~procent (Rys. \ref{fig:fuzzycompute:spark:10K}).
W~obu przypadkach pojedynczy komputer stacjonarny okazał się znacznie szybszy (od trzech do czterech razy) niż
klaster minikomputerów Jetson Nano.

\begin{figure}[h]
    \centering
    \begin{tikzpicture}
        \begin{axis}[
                ybar,
                width=0.8\textwidth,
                height=0.4\textheight,
                symbolic x coords={Jetson/CPU, Jetson/CUDA, PC},
                xtick={Jetson/CPU, Jetson/CUDA, PC},
                bar width=20pt,
                ymin=0,
                ylabel={Czas wykonania [s]}
            ]
            \addplot [error bars/.cd, y dir=both, y explicit]
            table [col sep=comma, x=Cluster, y=Avg, y error=STDev]{data/fuzzycompute-10K-spark.csv};
        \end{axis}
    \end{tikzpicture}
    \caption{Czasy wykonania FuzzyCompute dla platformy Spark i 10 tysięcy porcji.}
    \label{fig:fuzzycompute:spark:10K}
\end{figure}

Podczas przetwarzania 100 tysięcy porcji danych przy użyciu programu Hadoop czasy wykonania różniły się
nieznacznie -- algorytm wykorzystujący procesor graficzny okazał się o~2\% wolniejszy
(Rys. \ref{fig:fuzzycompute:hadoop:100K}). W~porównaniu z~wcześniejszymi testami programu FuzzyGen
wzrósł czas wykonania programu Hadoop dla większej objętości danych -- w~tym przypadku o~około 50 procent.

Przeprowadzając testy programu dla platformy Spark z~zestawem danych składającym się z~100 tysięcy
porcji, zauważono znacznie większe różnice w~wynikach. W tym przypadku algorytm w~języku CUDA był
wolniejszy od wersji w~Javie o~prawie 21 procent (Rys. \ref{fig:fuzzycompute:spark:100K}). Również
w~tym przypadku klaster oparty o~Jetsony Nano okazał się od czterech do pięciu razy wolniejszy od
komputera stacjonarnego.

\begin{figure}[h]
    \centering
    \begin{tikzpicture}
        \begin{axis}[
                ybar,
                width=0.8\textwidth,
                height=0.35\textheight,
                symbolic x coords={Jetson/CPU, Jetson/CUDA, PC},
                xtick={Jetson/CPU, Jetson/CUDA, PC},
                bar width=20pt,
                ymin=0,
                ylabel={Czas wykonania [s]}
            ]
            \addplot [error bars/.cd, y dir=both, y explicit]
            table [col sep=comma, x=Cluster, y=Avg, y error=STDev]{data/fuzzycompute-100K-hadoop.csv};
        \end{axis}
    \end{tikzpicture}
    \caption{Czasy wykonania FuzzyCompute dla platformy Hadoop i 100 tysięcy porcji.}
    \label{fig:fuzzycompute:hadoop:100K}
\end{figure}

\begin{figure}[h!]
    \centering
    \begin{tikzpicture}
        \begin{axis}[
                ybar,
                width=0.8\textwidth,
                height=0.35\textheight,
                symbolic x coords={Jetson/CPU, Jetson/CUDA, PC},
                xtick={Jetson/CPU, Jetson/CUDA, PC},
                bar width=20pt,
                ymin=0,
                ylabel={Czas wykonania [s]}
            ]
            \addplot [error bars/.cd, y dir=both, y explicit]
            table [col sep=comma, x=Cluster, y=Avg, y error=STDev]{data/fuzzycompute-100K-spark.csv};
        \end{axis}
    \end{tikzpicture}
    \caption{Czasy wykonania FuzzyCompute dla platformy Spark i 100 tysięcy porcji.}
    \label{fig:fuzzycompute:spark:100K}
\end{figure}
\newpage

\subsection*{Optymalizacje}

Procesory graficzne oparte o~architekturę CUDA są wyposażone w~multiprocesory strumieniowe
(ang. \english{Streaming Multiprocessor}), z~których każdy posiada dekoder instrukcji, pamięć
współdzieloną i~wiele jednostek arytmetycznych \cite{computer-arch}. Podczas wykonania programu
bloki wątków są przypisywane do jednego multiprocesora strumieniowego, a~kiedy liczba wątków w~bloku
przekracza liczbę jednostek arytmetycznych, procesor musi wykonać je szeregowo.

Model SIMT jest w rzeczywistości realizowany poprzez tzw. osnowę (ang. \english{warp}), czyli zespół wątków odpowiadający
pojedynczemu wątkowi w modelu SIMD. Osnowa składa się zwykle z~32 wątków, które działają równolegle
na jednym multiprocesorze strumieniowym i~wykonują tę samą instrukcję na wielu danych.
Jeśli podczas wykonania wątki wybiorą różne gałęzie programu, może wystąpić zjawisko rozbieżności
przetwarzania, gdzie wykonanie części wątków jest zawieszane, podczas gdy inne kontynuują obliczenia.

Prowadzi to do częściowego wykorzystania możliwości multiprocesora strumieniowego i~związanego
z~nim spadku wydajności. Użycie wyrażeń stałych (\lstinline{constexpr} w~języku CUDA) w~implementacji
kernela (listing \ref{lst:cuda-fuzzy-compute}) pozwoliło na usunięcie znacznej części instrukcji
warunkowych z~kodu wynikowego i~eliminację zjawiska rozbieżności wątków.

%TC:ignore
\begin{lstlisting}[
    language=C,
    label=lst:cuda-fuzzy-compute,
    caption={Kernel CudaFuzzyCompute realizujący obliczenia na zbiorach rozmytych.}
]
constexpr int SET_SIZE = 4;
constexpr int SETS_IN_RECORD = 64;

using Set = float[SET_SIZE];
using Record = Set[SETS_IN_RECORD];
using RecordEx = Set[SETS_IN_RECORD * 2];

__device__ void fuzzy_union(Set target, const Set source) {
    for (int i = 0; i < SET_SIZE; ++i) {
        target[i] = max(target[i], source[i]);
    }
}

__device__ void fuzzy_cmpl(Set target, const Set source) {
    for (int i = 0; i < SET_SIZE; ++i) {
        target[i] = 1.0f - source[i];
    }
}

__device__ void fuzzy_copy(Set target, const Set source) {
    for (int i = 0; i < SET_SIZE; ++i) {
        target[i] = source[i];
    }
}



extern "C"
__global__ void fuzzy_compute(Record records[], RecordEx temp[], const int64_t count) {
    const int64_t index = blockIdx.x * blockDim.x + threadIdx.x;
    if (index < count) {
        fuzzy_copy(temp[index][threadIdx.y], records[index][threadIdx.y]);
        fuzzy_cmpl(temp[index][threadIdx.y + SETS_IN_RECORD], records[index][threadIdx.y]);
        __syncthreads();
        for (int j = 0; j < SETS_IN_RECORD * 2; ++j) {
            int i = threadIdx.y;
            if (i != j) {
                fuzzy_union(temp[index][i], temp[index][j]);
            }
            i = threadIdx.y + SETS_IN_RECORD;
            if (i != j) {
                fuzzy_union(temp[index][i], temp[index][j]);
            }
        }
        __syncthreads();
        fuzzy_copy(records[index][threadIdx.y], temp[index][threadIdx.y]);
    }
}
\end{lstlisting}
%TC:endignore

\section{Program FuzzyFilter}

Program FuzzyFilter charakteryzuje się wysokim użyciem mocy obliczeniowej, przy jednoczesnym
wykonywaniu dużej liczby operacji odczytu i~zapisu. Ze względu na charakterystykę zawartych operacji,
program FuzzyFilter bardzo dobrze odzwierciedla wyzwania związane z przetwarzaniem \english{Big Data}.
Zbiory danych wykorzystane w testach wydajności uzyskano poprzez powielenie pliku CSV zawartego
w załączniku do pracy.

Podczas analizy wyników dla zestawu danych o~wielkości 100 MiB okazało się, że algorytm dla procesora
graficznego był wolniejszy od odpowiednika działającego na CPU. Różnica w czasach wykonania wynosiła
około 11 procent na niekorzyść wersji w~języku CUDA (Rys. \ref{fig:fuzzyfilter:100M}). Komputer stacjonarny
uzyskał wynik ośmiokrotnie lepszy od najkrótszego wykonania na klastrze minikomputerów Jetson Nano.

\begin{figure}[h]
    \centering
    \begin{tikzpicture}
        \begin{axis}[
                ybar,
                width=0.8\textwidth,
                height=0.4\textheight,
                symbolic x coords={Jetson/CPU, Jetson/CUDA, PC},
                xtick={Jetson/CPU, Jetson/CUDA, PC},
                bar width=20pt,
                ymin=0,
                ylabel={Czas wykonania [s]}
            ]
            \addplot [error bars/.cd, y dir=both, y explicit]
            table [col sep=comma, x=Cluster, y=Avg, y error=STDev]{data/fuzzyfilter-100M.csv};
        \end{axis}
    \end{tikzpicture}
    \caption{Czasy wykonania FuzzyFilter dla platformy Spark i zbioru danych 100 MB.}
    \label{fig:fuzzyfilter:100M}
\end{figure}
\newpage

Dla zestawu danych o~objętości 500 MiB implementacja przeznaczona na procesor graficzny znów okazała
się wolniejsza. W tym przypadku różnica w~czasie wykonania wynosiła około 21 procent na korzyść wersji
w~języku Java (Rys. \ref{fig:fuzzyfilter:500M}). Utrzymała się również proporcja między klastrem minikomputerów Jetson, a~komputerem
stacjonarnym, dla którego czas wykonania znów był osiem razy krótszy.

\begin{figure}[h!]
    \centering
    \begin{tikzpicture}
        \begin{axis}[
                ybar,
                width=0.8\textwidth,
                height=0.4\textheight,
                symbolic x coords={Jetson/CPU, Jetson/CUDA, PC},
                xtick={Jetson/CPU, Jetson/CUDA, PC},
                bar width=20pt,
                ymin=0,
                ylabel={Czas wykonania [s]}
            ]
            \addplot [error bars/.cd, y dir=both, y explicit]
            table [col sep=comma, x=Cluster, y=Avg, y error=STDev]{data/fuzzyfilter-500M.csv};
        \end{axis}
    \end{tikzpicture}
    \caption{Czasy wykonania FuzzyFilter dla platformy Spark i zbioru danych 500 MB.}
    \label{fig:fuzzyfilter:500M}
\end{figure}

Podobnie jak w~przypadku programu FuzzyCompute, implementacja w~języku CUDA nie przyniosła
oczekiwanego wzrostu wydajności w~porównaniu do wersji dla procesora głównego.

W~trakcie testów zauważono pewne problemy z~implementacją na procesor graficzny. Głównym objawem
było nieoczekiwane przerywanie zadań Spark, których przyczyną było ustawienie zbyt małego rozmiaru
sterty dla maszyny wirtualnej Javy. Problem rozwiązano przez zwiększenie maksymalnego rozmiaru sterty,
co zapewniło stabilne działanie programu.

\subsection*{Optymalizacje}

W~przypadku implementacji dla środowiska CUDA (listing \ref{lst:cuda-fuzzy-filter}), każdy wątek
CUDA zajmuje się obliczaniem wartości wszystkich funkcji przynależności dla danego rekordu CSV.
Wartości te są następnie łączone za pomocą T-normy minimum.

Wątki w~obrębie jednej osnowy (ang. \english{warp}), czyli grupy 32 wątków, powinny czytać dane z~pamięci w sposób łączony,
tj. wątki powinny czytać sąsiednie adresy pamięci. Dostępy łączone zmniejszają liczbę wykonywanych
transakcji pamięci i~zwiększają jej przepustowość. W~pewnych przypadkach dostęp do pamięci można
przyspieszyć poprzez zmianę kolejności, w~jakiej wątki czytają lub zapisują dane.

W~ramach optymalizacji programu na procesor graficzny rozmieszczenie danych w~pamięci zostało
zmienione w~taki sposób, aby wartości kolumn dla każdego filtra rozmytego znajdowały się w~sąsiednich
komórkach pamięci.
Kopiowanie danych między pamięcią główną komputera a~pamięcią urządzenia jest kosztowną operacją.
Z~tego względu ograniczono operacje kopiowania tylko do kolumn, na których zostały określone
filtry rozmyte.
\newpage

%TC:ignore
\begin{lstlisting}[
    language=C,
    label=lst:cuda-fuzzy-filter,
    caption={Kernel CudaFuzzyFilter obliczający stopień przynależności dla rekordu CSV.}
]
#define A params[0]
#define B params[1]
#define C params[2]
#define D params[3]

__device__ float membership(const float value, const float params[4]) {
    if (value <= A || value >= D) {
        return 0;  // out of range
    } else if (value >= B && value <= C) {
        return 1;  // inside plateau
    } else if (value < B) { // (val > A && val < B)
        return (value - A) / (B - A); // left triangle
    } else {  // (val > c && val < d)
        return (D - value) / (D - C);  // right triangle
    }
}
extern "C"
__global__ void fuzzy_filter(
    short member[],
    const float row_values[],
    const float member_fn_params[],
    const float threshold,
    const int32_t n_rows,
    const int32_t n_filters
) {
    const int32_t index = blockIdx.x * blockDim.x + threadIdx.x;
    if (index < n_rows) {
        float t_norm = 1.0;
        for (int32_t i = 0; i < n_filters; ++i) {
            const float value = row_values[i * n_rows + index];
            const float* params = &member_fn_params[i * 4];
            t_norm = min(t_norm, membership(value, params));
        }
        member[index] = t_norm > threshold;
    }
}
\end{lstlisting}
%TC:endignore
 % Badania
\chapter{Podsumowanie} \label{ch:podsumowanie}

Głównym celem pracy była budowa klastra \english{Big Data} składającego się z~minikomputerów
wspierających akcelerację GPU, a~następnie badanie jego wydajności dla rozproszonych programów na
platformy Apache Hadoop oraz Apache Spark.

Mając na uwadze potencjalną wydajność i~koszt dostępnych minikomputerów jednopłytkowych, do realizacji
klastra wybrano model Jetson Nano w~wersji z~4 GB pamięci RAM. Zbudowano klaster składający się z~trzech
takich minikomputerów, pełniących funkcję węzłów wykonawczych, a~także z~minikomputera Raspberry Pi
4B, którego zadaniem było zarządzanie całym klastrem.

Proces instalacji oprogramowania i~konfiguracji klastra został w~dużej mierze zautomatyzowany przy
użyciu programu Ansible, co pozwoliło na łatwą instalację platform Hadoop oraz Spark w~wielu
środowiskach. Opracowano też skrypty do automatycznego uruchamiania testów wydajności i~zbierania
ich wyników, co ułatwiło porównanie klastra minikomputerów Jetson Nano z~pojedynczym komputerem
stacjonarnym.

Następnie przygotowano zestaw programów testowych, który posłużył do przeprowadzenia testów wydajności
na zbudowanym klastrze. Każdy program testowy umożliwia wykonanie zarówno na procesorze głównym,
jak i~z~wykorzystaniem technologii CUDA, dzięki czemu określono przyrost wydajności wynikający
z~użycia akceleracji GPU. Co więcej, dla jednego z~programów testowych dostarczono też obsługę
standardu OpenCL.

Dla testowego zestawu programów wykonano pomiary czasu wykonania, których wyniki posłużyły do dalszej
analizy i~porównania wydajności klastra z~konkurencyjnymi rozwiązaniami. Testy wydajności zostały
przeprowadzone z~uwzględnieniem procesu kompilacji Just-In-Time dla języków Java oraz CUDA.
\newpage

\section{Wnioski}

W testach programu PiEstimation z~próbkowaniem 1 miliona punktów klaster minikomputerów Jetson Nano
nie osiągnął przyspieszenia względem klastra Raspberry Pi 4 opisanego w~artykule \cite{rpi-cluster-2}.
Oznaczało to, że klaster komputerów Jetson nie był też konkurencyjny pod względem stosunku wydajności
do ceny.

Charakterystyka wydajności dla implementacji w języku CUDA stawała się coraz lepsza wraz ze wzrostem
liczby próbkowanych punktów, jednak dla tych wartości jedynym punktem odniesienia był komputer
stacjonarny bazujący na procesorze AMD Ryzen.

Dla programu PiEstimation i~próbkowania 100 milionów punktów klaster Jetson Nano osiągnął wynik
porównywalny z~komputerem PC, a~przy próbkowaniu 1 miliarda punktów okazał się szybszy o~ponad 36\%.
Z~drugiej strony, estymacja liczby $\pi$ była zadaniem, które w~najmniejszym stopniu przypominało
aplikacje do przetwarzania zbiorów \english{Big Data}.

Niestety test PiEstimation to jedyny przypadek, w~którym klaster oparty o~minikomputery jednopłytkowe
mógł równać się z~komputerem stacjonarnym. Testy programów FuzzyGen, FuzzyCompute oraz FuzzyFilter
wykazały, że wydajność klastra Jetson Nano nie jest konkurencyjna dla zadań rozproszonych
o~charakterystyce typowej dla obliczeń \english{Big Data}.

Za główny czynnik ograniczający wydajność programów uznano system plików HDFS, który przechowywał
dane na kartach microSD. W~kolejnych badaniach warto rozpatrzyć użycie innych nośników danych,
takich jak karty microSD o~wyższych klasach wydajności i~dyski twarde obsługujące interfejs USB 3.0.

W~przypadku programu FuzzyFilter interesującym kierunkiem badań jest dalsza optymalizacja kernela
CUDA oraz rozszerzenie funkcjonalności programu o~operacje łączeń, grupowania i~filtrowania z~użyciem
bardziej złożonych operacji logicznych. W tym celu pomocne może okazać się użycie platformy Apache
Hive wraz z~biblioteką FuzzyHive \cite{fuzzy-hive}, która dostarcza gotowych implementacji operacji
rozmytych dla dużych zbiorów danych. 

Kolejnym wartym rozważenia kierunkiem badań jest wpływ akceleracji GPU na aplikacje w~innych
dziedzinach. Minikomputery jednopłytkowe Jetson Nano posiadają wydajne procesory graficzne,
których zastosowanie może znacząco przyspieszyć aplikacje grafiki komputerowej i~uczenia maszynowego.

Programy testowe oraz skrypty administracyjne wykonane na potrzeby pracy stały się dobrą podstawą do
pomiarów i~analizy wydajności klastrów obliczeniowych \english{Big Data}. Szczególnie pomocne okazały
się skrypty instalacyjne Ansible, które znacznie przyspieszyły proces konfiguracji klastra w~wielu
środowiskach, co znacząco usprawniło proces badań.
 % Podsumowanie

\backmatter

%\bibliographystyle{plplain}  % bibtex
%\bibliography{biblio/biblio} % bibtex
\printbibliography           % biblatex
\addcontentsline{toc}{chapter}{Bibliografia}

\begin{appendices}

\chapter{Dokumentacja techniczna} \label{ch:techdoc}

\section*{Klasa QmcUtils}

Klasa QmcUtils (listing \ref{lst:qmc-utils}) dostarcza metod użytecznych w implementacjach metody Monte Carlo.

\begin{lstlisting}[
    language=Java,
    label=lst:qmc-utils,
    caption={Implementacja metod użytecznych w implementacjach metody Monte Carlo.}
]
public final class QmcUtils {

    private static final int[] BASES = {2, 3};
    private static final int DIMENSIONS = BASES.length;
    private static final int[] MAX_DIGITS = {63, 40};

    private QmcUtils() {}

    public static double[] getRandomPoint(long index) {
        double[] point = new double[DIMENSIONS];

        double[][] q = new double[DIMENSIONS][];
        int[][] d = new int[DIMENSIONS][];
        for (int i = 0; i < DIMENSIONS; i++) {
            q[i] = new double[MAX_DIGITS[i]];
            d[i] = new int[MAX_DIGITS[i]];
        }

        for (int i = 0; i < DIMENSIONS; i++) {
            long k = index;
            point[i] = 0;

            for (int j = 0; j < MAX_DIGITS[i]; j++) {
                q[i][j] = (j == 0 ? 1.0 : q[i][j - 1]) / BASES[i];
                d[i][j] = (int) (k % BASES[i]);
                k = (k - d[i][j]) / BASES[i];
                point[i] += d[i][j] * q[i][j];
            }

            for (int j = 0; j < MAX_DIGITS[i]; j++) {
                d[i][j]++;
                point[i] += q[i][j];
                if (d[i][j] < BASES[i]) {
                    break;
                }
                d[i][j] = 0;
                point[i] -= (j == 0 ? 1.0 : q[i][j - 1]);
            }
        }

        return point;
    }

    public static boolean isPointInside(double[] point) {
        final double x = point[0] - 0.5;
        final double y = point[1] - 0.5;

        return x * x + y * y <= 0.25;
    }
}
\end{lstlisting}
\newpage

\section*{Klasa FuzzyUtils}

Klasa FuzzyUtils (listing \ref{lst:fuzzy-utils}) implementuje operacje na zbiorach rozmytych na procesorze głównym.

\begin{lstlisting}[
    language=Java,
    label=lst:fuzzy-utils,
    caption={Implementacja operacji rozmytych na procesor główny.}
]
public final class FuzzyUtils {

    private static final int SET_SIZE = 4;
    public static final int SETS_IN_RECORD = 64;
    public static final int RECORD_SIZE = SET_SIZE * SETS_IN_RECORD;
    public static final int RECORD_BYTES = RECORD_SIZE * Float.BYTES;

    private FuzzyUtils() {}

    public static float[] fuzzyUnion(float[] target, float[] source) {
        int n = Math.min(target.length, source.length);
        for (int i = 0; i < n; ++i) {
            target[i] = Math.max(target[i], source[i]);
        }
        return target;
    }

    public static float[] fuzzyComplement(float[] set) {
        int n = set.length;
        float[] result = new float[n];
        for (int i = 0; i < n; ++i) {
            result[i] = 1.0f - set[i];
        }
        return result;
    }

    public static void generateRandomSets(long nRecords, Consumer<byte[]> consumer) {
        var random = new Random();
        for (int i = 0; i < nRecords; ++i) {
            var bytes = new byte[RECORD_BYTES];
            var buffer = ByteBuffer.wrap(bytes);
            for (int j = 0; j < RECORD_SIZE; ++j) {
                buffer.putFloat(random.nextFloat());
            }

            consumer.accept(bytes);
        }
    }

    public static void performOps(byte[] inputRecord) {
        var buffer = ByteBuffer.wrap(inputRecord);

        var sets = new float[SETS_IN_RECORD * 2][];
        for (int i = 0; i < SETS_IN_RECORD; ++i) {
            var set = sets[i] = new float[SET_SIZE];
            for (int j = 0; j < SET_SIZE; ++j) {
                set[j] = buffer.getFloat();
            }
        }
        for (int i = 0; i < SETS_IN_RECORD; ++i) {
            sets[i + SETS_IN_RECORD] = FuzzyUtils.fuzzyComplement(sets[i]);
        }

        var results = Arrays.stream(sets).map(float[]::clone).toArray(float[][]::new);
        for (int i = 0; i < sets.length; ++i) {
            for (int j = 0; j < sets.length; ++j) {
                if (i != j) {
                    results[i] = FuzzyUtils.fuzzyUnion(results[i], results[j]);
                }
            }
        }

        buffer.rewind();
        for (int i = 0; i < SETS_IN_RECORD; ++i) {
            for (int j = 0; j < SET_SIZE; ++j) {
                buffer.putFloat(results[i][j]);
            }
        }
    }

    public static void forEachRecord(CUdeviceptr inputRecords, long nRecords, Consumer<byte[]> consumer) {
        long byteCount = nRecords * RECORD_BYTES;
        var bytes = new byte[(int) byteCount];
        cuMemcpyDtoH(Pointer.to(bytes), inputRecords, byteCount);

        var buffer = ByteBuffer.wrap(bytes);
        for (int i = 0; i < nRecords; ++i) {
            var writableBuf = new byte[RECORD_BYTES];
            buffer.get(writableBuf, 0, RECORD_BYTES);
            consumer.accept(writableBuf);
        }
    }
}
\end{lstlisting}
\newpage

\section*{Klasa JcudaUtils}

Klasa JcudaUtils (listing \ref{lst:jcuda-utils}) zawiera metody pomocnicze do wywoływania programów na procesorze graficznym.

\begin{lstlisting}[
    language=Java,
    label=lst:jcuda-utils,
    caption={Metody pomocnicze do wywoływania programów na procesorze graficznym.}
]
public final class JcudaUtils {

    private JcudaUtils() {}

    public static CUcontext createCudaContext() {
        JCudaDriver.setExceptionsEnabled(true);
        JCurand.setExceptionsEnabled(true);
        cuInit(0);

        var device = new CUdevice();
        cuDeviceGet(device, 0);

        var ctx = new CUcontext();
        cuCtxCreate(ctx, 0, device);
        return ctx;
    }

    public static CUdeviceptr allocateDeviceMemory(long size) {
        var deviceMemory = new CUdeviceptr();
        cuMemAlloc(deviceMemory, size);
        return deviceMemory;
    }

    public static curandGenerator createRandomGenerator() {
        var generator = new curandGenerator();
        curandCreateGenerator(generator, CURAND_RNG_PSEUDO_PHILOX4_32_10);
        curandSetPseudoRandomGeneratorSeed(generator, System.currentTimeMillis());
        return generator;
    }

    public static void freeResources(NativePointerObject... resources) {
        Objects.requireNonNull(resources);

        for (var res : resources) {
            if (res instanceof curandGenerator) {
                curandDestroyGenerator((curandGenerator) res);
            } else if (res instanceof CUdeviceptr) {
                cuMemFree((CUdeviceptr) res);
            } else {
                throw new IllegalArgumentException(String.format("No handler for resource type: %s", res.getClass().getName()));
            }
        }
    }

    public static CUdeviceptr qmcGeneratePoints(long nPoints, long seqOffset) throws IOException {
        int blockSizeX = 256;
        int gridSizeX = (int) Math.ceil((double) nPoints / blockSizeX);

        var guesses = JcudaUtils.allocateDeviceMemory(nPoints * Sizeof.SHORT);
        var kernelParams = Pointer.to(
                Pointer.to(guesses),
                Pointer.to(new long[]{nPoints}),
                Pointer.to(new long[]{seqOffset})
        );

        cuLaunchKernel(
                JcudaUtils.loadFunctionFromPtx("/CudaQmcKernel.ptx", "qmc_mapper"),
                gridSizeX, 1, 1,
                blockSizeX, 1, 1,
                0, null,
                kernelParams, null
        );
        cuCtxSynchronize();

        return guesses;
    }

    public static long sumShortArray(CUdeviceptr inputArray, long size) throws IOException {
        Objects.requireNonNull(inputArray);

        int blockSizeX = 256;
        int gridSizeX = (int) Math.ceil((double) size / blockSizeX);

        var partialSums = JcudaUtils.allocateDeviceMemory((long) gridSizeX * Sizeof.SHORT);
        var kernelParams = Pointer.to(
                Pointer.to(inputArray),
                Pointer.to(partialSums),
                Pointer.to(new long[]{size})
        );

        cuLaunchKernel(
                JcudaUtils.loadFunctionFromPtx("/CudaReduction.ptx", "reduce_short"),
                gridSizeX, 1, 1,
                blockSizeX, 1, 1,
                blockSizeX * Sizeof.SHORT, null,
                kernelParams, null
        );
        cuCtxSynchronize();

        var sums = new short[gridSizeX];
        cuMemcpyDtoH(Pointer.to(sums), partialSums, (long) gridSizeX * Sizeof.SHORT);
        JcudaUtils.freeResources(partialSums);
        return IntStream.range(0, sums.length).map(i -> sums[i]).sum();
    }

    public static CUdeviceptr fuzzyPerformOps(byte[] inputRecords, long nRecords) throws IOException {
        Objects.requireNonNull(inputRecords);

        int blockSizeX = 1024 / FuzzyUtils.SETS_IN_RECORD;
        int gridSizeX = (int) Math.ceil((double) inputRecords.length / blockSizeX);

        long nBytes = nRecords * FuzzyUtils.RECORD_BYTES;
        var records = JcudaUtils.allocateDeviceMemory(nBytes);
        var temp = JcudaUtils.allocateDeviceMemory(nBytes * 2);
        cuMemcpyHtoD(records, Pointer.to(inputRecords), nBytes);

        var kernelParams = Pointer.to(
                Pointer.to(records),
                Pointer.to(temp),
                Pointer.to(new long[]{nRecords})
        );
        cuLaunchKernel(
                JcudaUtils.loadFunctionFromPtx("/CudaFuzzyCompute.ptx", "fuzzy_compute"),
                gridSizeX, 1, 1,
                blockSizeX, FuzzyUtils.SETS_IN_RECORD, 1,
                0, null,
                kernelParams, null
        );
        cuCtxSynchronize();

        return records;
    }

    public static CUdeviceptr fuzzyFilterRows(
            float[] rowValues, float[] membershipFnParams, float threshold, int nRows, int nFilters
    ) throws IOException {
        Objects.requireNonNull(rowValues);
        Objects.requireNonNull(membershipFnParams);

        long rowValueBytes = (long) rowValues.length * Sizeof.FLOAT;
        var rowValuesPtr = JcudaUtils.allocateDeviceMemory(rowValueBytes);
        cuMemcpyHtoD(rowValuesPtr, Pointer.to(rowValues), rowValueBytes);

        long membershipFnParamsBytes = (long) membershipFnParams.length * Sizeof.FLOAT;
        var membershipFnParamsPtr = JcudaUtils.allocateDeviceMemory(membershipFnParamsBytes);
        cuMemcpyHtoD(membershipFnParamsPtr, Pointer.to(membershipFnParams), membershipFnParamsBytes);

        int blockSizeX = 256;
        int gridSizeX = (int) Math.ceil((double) nRows / blockSizeX);
        var member = JcudaUtils.allocateDeviceMemory((long) nRows * Sizeof.SHORT);
        var kernelParams = Pointer.to(
                Pointer.to(member),
                Pointer.to(rowValuesPtr),
                Pointer.to(membershipFnParamsPtr),
                Pointer.to(new float[]{threshold}),
                Pointer.to(new int[]{nRows}),
                Pointer.to(new int[]{nFilters})
        );

        cuLaunchKernel(
                JcudaUtils.loadFunctionFromPtx("/CudaFuzzyFilter.ptx", "fuzzy_filter"),
                gridSizeX, 1, 1,
                blockSizeX, 1, 1,
                0, null,
                kernelParams, null
        );
        cuCtxSynchronize();

        JcudaUtils.freeResources(rowValuesPtr, membershipFnParamsPtr);
        return member;
    }

    public static CUfunction loadFunctionFromPtx(String ptxPath, String funcName) throws IOException {
        Objects.requireNonNull(ptxPath);
        Objects.requireNonNull(funcName);

        var resource = JcudaUtils.class.getResourceAsStream(ptxPath);
        var ptx = JcudaUtils.toNullTerminatedByteArray(resource);

        var module = new CUmodule();
        cuModuleLoadData(module, ptx);

        var func = new CUfunction();
        cuModuleGetFunction(func, module, funcName);
        return func;
    }

    private static byte[] toNullTerminatedByteArray(InputStream inStream) throws IOException {
        Objects.requireNonNull(inStream);

        var outStream = new ByteArrayOutputStream();
        inStream.transferTo(outStream);
        outStream.write(0);
        return outStream.toByteArray();
    }
}
\end{lstlisting}
\newpage

\section*{Klasa CpuQmcMapper} \label{ch:cpu-qmc-mapper}

Klasa CpuQmcMapper (listing \ref{lst:cpu-qmc-mapper}) implementuje komponent Mapper dla
platformy Apache Hadoop, którego zadaniem w programie PiEstimation jest generowanie punktów na
podstawie sekwencji Haltona i~zliczanie tych, które znajdują się w obrębie koła. Komponent
próbkuje punkty dla pewnego przedziału, określonego przez parametry \lstinline{offset} 
i~\lstinline{"size"}. Wyniki są zapisywane do kontekstu z~kluczem typu \lstinline{BooleanWritable}
o~wartości true lub false (w~zależności od tego, czy punkt jest wewnątrz czy na zewnątrz)
i~wartością typu LongWritable, reprezentującą liczbę punktów.

\begin{lstlisting}[
    language=Java,
    label=lst:cpu-qmc-mapper,
    caption={Implementacja komponentu Mapper na procesor główny dla programu PiEstimation.}
]
public class CpuQmcMapper extends Mapper<LongWritable, LongWritable, BooleanWritable, LongWritable> {

    public void map(LongWritable offset, LongWritable size, Context context) throws IOException, InterruptedException {
        long numInside = 0L;
        long numOutside = 0L;

        for (long i = 0; i < size.get(); ++i) {
            final double[] point = QmcUtils.getRandomPoint(offset.get() + i);
            if (QmcUtils.isPointInside(point)) {
                numInside++;
            } else {
                numOutside++;
            }
        }

        context.write(new BooleanWritable(true), new LongWritable(numInside));
        context.write(new BooleanWritable(false), new LongWritable(numOutside));
    }
}
\end{lstlisting}
\newpage

\section*{Klasa AparapiQmcMapper} \label{ch:aparapi-qmc-mapper}

Klasa AparapiQmcMapper (listing \ref{lst:aparapi-qmc-mapper}) implementuje komponent Mapper dla
platformy Apache Hadoop, którego zadaniem jest generowanie punktów na podstawie sekwencji
Haltona i~zliczanie punktów znajdujących się w~środku koła. Generowanie i~klasyfikacja punktów
jest wywoływane za pomocą klasy Kernel z biblioteki Aparapi, przez co może być wykonywane na
procesorze graficznym. 

\begin{lstlisting}[
    language=Java,
    label=lst:aparapi-qmc-mapper,
    caption={Implementacja komponentu Mapper dla programu PiEstimation, używająca biblioteki Aparapi.}
]
public class AparapiQmcMapper extends Mapper<LongWritable, LongWritable, BooleanWritable, LongWritable> {

    @Override
    public void map(LongWritable offset, LongWritable size, Context context) throws IOException, InterruptedException {
        int isize = (int) size.get();
        int ioffset = (int) offset.get();
        final boolean[] guesses = new boolean[isize];

        float[][] q = new float[isize][];
        int[][] d = new int[isize][];
        for (int i = 0; i < isize; i++) {
            q[i] = new float[63];
            d[i] = new int[63];
        }

        Kernel kernel = new Kernel() {

            private float getRandomPoint(long index, float[] q, int[] d, int base, int maxDigits) {

                float value = 0.0f;
                long k = index;

                for (int i = 0; i < maxDigits; i++) {
                    q[i] = (i == 0 ? 1.0f : q[i - 1]) / base;
                    d[i] = (int) (k % base);
                    k = (k - d[i]) / base;
                    value += d[i] * q[i];
                }

                boolean cont = true;
                for (int i = 0; i < maxDigits; i++) {
                    if (cont) {
                        d[i]++;
                        value += q[i];
                        if (d[i] < base) {
                            cont = false;
                        } else {
                            value -= (i == 0 ? 1.0f : q[i - 1]);
                        }
                    }
                }

                return value;
            }

            @Override
            public void run() {
                final int i = getGlobalId();
                final long indexOffset = i + ioffset;

                final float x = getRandomPoint(indexOffset, q[i], d[i], 2, 63) - 0.5f;
                final float y = getRandomPoint(indexOffset, q[i], d[i], 3, 40) - 0.5f;

                guesses[i] = (x * x + y * y > 0.25f);
            }
        };
        kernel.execute(Range.create(isize));

        long numInside = 0L;
        long numOutside = 0L;
        for (boolean isOutside : guesses) {
            if (isOutside) {
                numOutside++;
            } else {
                numInside++;
            }
        }

        kernel.dispose();
        context.write(new BooleanWritable(true), new LongWritable(numInside));
        context.write(new BooleanWritable(false), new LongWritable(numOutside));
    }
}
\end{lstlisting}
\newpage

\section*{Klasa JcudaQmcMapper} \label{ch:jcuda-qmc-mapper}

Klasa JcudaQmcMapper (listing \ref{lst:jcuda-qmc-mapper}) implementuje komponent Mapper dla
platformy Hadoop, którego zadaniem jest generowanie punktów na podstawie sekwencji Haltona
i~zliczanie punktów znajdujących się w~środku koła. Generowanie i~klasyfikacja punktów jest
wywoływane za pomocą biblioteki JCuda, prez co jest wykonywane na procesorze graficznym. 

\begin{lstlisting}[
    language=Java,
    label=lst:jcuda-qmc-mapper,
    caption={Implementacja komponentu Mapper dla programu PiEstimation, używająca biblioteki JCuda.}
]
public class JcudaQmcMapper extends Mapper<LongWritable, LongWritable, BooleanWritable, LongWritable> {

    @Override
    public void map(LongWritable offset, LongWritable size, Context context) throws IOException, InterruptedException {
        var ctx = JcudaUtils.createCudaContext();

        var guesses = JcudaUtils.qmcGeneratePoints(size.get(), offset.get());
        long numOutside = JcudaUtils.sumShortArray(guesses, size.get());
        long numInside = size.get() - numOutside;

        context.write(new BooleanWritable(true), new LongWritable(numInside));
        context.write(new BooleanWritable(false), new LongWritable(numOutside));
        JcudaUtils.freeResources(guesses);
    }
}
\end{lstlisting}
\newpage

\section*{Klasa CpuQmcFunction} \label{ch:cpu-qmc-function}

Klasa CpuQmcFunction (listing \ref{lst:cpu-qmc-function} jest odpowiednikiem klasy CpuQmcMapper
(listing \ref{lst:cpu-qmc-mapper}) dla platformy Spark.

\begin{lstlisting}[
    language=Java,
    label=lst:cpu-qmc-function,
    caption={Funkcja na procesor główny dla programu PiEstimation.}
]
public class CpuQmcFunction {

    public static Long call(Tuple2<Long, Long> input) {
        long nSamples = input._2();
        long offset = input._1() * nSamples;

        long numInside = 0L;
        for (long i = 0; i < nSamples; ++i) {
            final double[] point = QmcUtils.getRandomPoint(offset + i);
            if (QmcUtils.isPointInside(point)) {
                numInside++;
            }
        }

        return numInside;
    }
}
\end{lstlisting}
\newpage

\section*{Klasa AparapiQmcFunction} \label{ch:aparapi-qmc-function}

Klasa AparapiQmcFunction (listing \ref{lst:aparapi-qmc-function} jest odpowiednikiem klasy AparapiQmcMapper
(listing \ref{lst:aparapi-qmc-mapper}) dla platformy Spark.

\begin{lstlisting}[
    language=Java,
    label=lst:aparapi-qmc-function,
    caption={Funkcja dla programu PiEstimation, wykorzystująca bibliotekę Aparapi.}
]
public class AparapiQmcFunction {

    public static Long call(Tuple2<Long, Long> input) {
        int nSamples = Math.toIntExact(input._2());
        int offset = (int) (input._1() * nSamples);
        final var guesses = new boolean[nSamples];

        var q = new float[nSamples][];
        var d = new int[nSamples][];
        for (int i = 0; i < nSamples; ++i) {
            q[i] = new float[63];
            d[i] = new int[63];
        }

        var kernel = new Kernel() {

            private float getRandomPoint(long index, float[] q, int[] d, int base, int maxDigits) {

                float value = 0.0f;
                long k = index;

                for (int i = 0; i < maxDigits; ++i) {
                    q[i] = (i == 0 ? 1.0f : q[i - 1]) / base;
                    d[i] = (int) (k % base);
                    k = (k - d[i]) / base;
                    value += d[i] * q[i];
                }

                boolean cont = true;
                for (int i = 0; i < maxDigits; ++i) {
                    if (cont) {
                        d[i]++;
                        value += q[i];
                        if (d[i] < base) {
                            cont = false;
                        } else {
                            value -= (i == 0 ? 1.0f : q[i - 1]);
                        }
                    }
                }

                return value;
            }

            @Override
            public void run() {
                final int i = getGlobalId();
                final long indexOffset = i + offset;

                final float x = getRandomPoint(indexOffset, q[i], d[i], 2, 63) - 0.5f;
                final float y = getRandomPoint(indexOffset, q[i], d[i], 3, 40) - 0.5f;

                guesses[i] = (x * x + y * y <= 0.25f);
            }
        };
        kernel.execute(Range.create(nSamples));

        long numInside = 0L;
        for (boolean isInside : guesses) {
            if (isInside) {
                numInside++;
            }
        }
        kernel.dispose();

        return numInside;
    }
}
\end{lstlisting}
\newpage

\section*{Klasa JcudaQmcFunction} \label{ch:jcuda-qmc-function}

Klasa JcudaQmcFunction (listing \ref{lst:jcuda-qmc-function} jest odpowiednikiem klasy JcudaQmcMapper
(listing \ref{lst:jcuda-qmc-mapper}) dla platformy Spark.

\begin{lstlisting}[
    language=Java,
    label=lst:jcuda-qmc-function,
    caption={Funkcja dla programu PiEstimation, wykorzystująca bibliotekę JCuda.}
]
public class JcudaQmcFunction {

    public static Long call(Tuple2<Long, Long> input) throws IOException {
        long nSamples = input._2();
        long offset = input._1() * nSamples;
        var ctx = JcudaUtils.createCudaContext();

        var guesses = JcudaUtils.qmcGeneratePoints(nSamples, offset);
        long numOutside = JcudaUtils.sumShortArray(guesses, nSamples);
        long numInside = nSamples - numOutside;

        JcudaUtils.freeResources(guesses);
        return numInside;
    }
}
\end{lstlisting}
\newpage

\section*{Klasa CpuFuzzyGenMapper} \label{ch:cpu-fuzzygen-mapper}

Klasa CpuFuzzyGenMapper (listing \ref{lst:cpu-fuzzygen-mapper}) generuje podaną liczbę losowych
wartości zmiennoprzecinkowych na procesorze głównym.

\begin{lstlisting}[
    language=Java,
    label=lst:cpu-fuzzygen-mapper,
    caption={Komponent Mapper na procesor główny dla programu FuzzyGen.}
]
public class CpuFuzzyGenMapper extends Mapper<LongWritable, NullWritable, NullWritable, BytesWritable> {

    @Override
    public void map(LongWritable key, NullWritable ignored, Context context) {
        var nRecords = key.get();
        FuzzyUtils.generateRandomSets(nRecords, buf -> HadoopJobUtils.writeByteRecord(context, buf));
    }
}
\end{lstlisting}
\newpage

\section*{Klasa JCudaFuzzyGenMapper} \label{ch:jcuda-fuzzygen-mapper}

Klasa JCudaFuzzyGenMapper (listing \ref{lst:jcuda-fuzzygen-mapper}) generuje podaną liczbę losowych
wartości zmiennoprzecinkowych na procesorze graficznym.

\begin{lstlisting}[
    language=Java,
    label=lst:jcuda-fuzzygen-mapper,
    caption={Komponent Mapper na procesor graficzny dla programu FuzzyGen.}
]
public class JcudaFuzzyGenMapper extends Mapper<LongWritable, NullWritable, NullWritable, BytesWritable> {

    @Override
    public void map(LongWritable key, NullWritable ignored, Context context) {
        var ctx = JcudaUtils.createCudaContext();
        var nRecords = key.get();
        var floatCount = nRecords * RECORD_SIZE;
        var byteCount = floatCount * Sizeof.FLOAT;

        var randOutput = JcudaUtils.allocateDeviceMemory(byteCount);
        var generator = JcudaUtils.createRandomGenerator();

        curandGenerateUniform(generator, randOutput, floatCount);
        cuCtxSynchronize();

        FuzzyUtils.forEachRecord(randOutput, nRecords, buf -> HadoopJobUtils.writeByteRecord(context, buf));
        JcudaUtils.freeResources(generator, randOutput);
    }
}
\end{lstlisting}
\newpage

\section*{Klasa CpuFuzzyGenFunction} \label{ch:cpu-fuzzygen-function}

Klasa CpuFuzzyGenFunction (listing \ref{lst:cpu-fuzzygen-function}) jest odpowiednikiem
klasy CpuFuzzyGenMapper (listting \ref{lst:cpu-fuzzygen-mapper}) dla platformy Spark.

\begin{lstlisting}[
    language=Java,
    label=lst:cpu-fuzzygen-function,
    caption={Funkcja na procesor główny dla programu FuzzyGen.}
]
public class CpuFuzzyGenFunction {

    public static Iterator<byte[]> call(Long nRecords) {
        var results = new ArrayList<byte[]>();
        FuzzyUtils.generateRandomSets(nRecords, results::add);
        return results.iterator();
    }
}
\end{lstlisting}
\newpage

\section*{Klasa JCudaFuzzyGenFunction} \label{ch:jcuda-fuzzygen-function}

Klasa JCudaFuzzyGenFunction (listing \ref{lst:jcuda-fuzzygen-function}) jest odpowiednikiem
klasy JCudaFuzzyGenMapper (listting \ref{lst:jcuda-fuzzygen-mapper}) dla platformy Spark.

\begin{lstlisting}[
    language=Java,
    label=lst:jcuda-fuzzygen-function,
    caption={Funkcja na procesor graficzny dla programu FuzzyGen.}
]
public class JcudaFuzzyGenFunction {

    public static Iterator<byte[]> call(Long nRecords) {
        var ctx = JcudaUtils.createCudaContext();
        long floatCount = nRecords * RECORD_SIZE;
        long byteCount = floatCount * Sizeof.FLOAT;

        var randOutput = JcudaUtils.allocateDeviceMemory(byteCount);
        var generator = JcudaUtils.createRandomGenerator();

        curandGenerateUniform(generator, randOutput, floatCount);
        cuCtxSynchronize();

        var results = new ArrayList<byte[]>();
        FuzzyUtils.forEachRecord(randOutput, nRecords, results::add);
        JcudaUtils.freeResources(generator, randOutput);
        return results.iterator();
    }
}
\end{lstlisting}
\newpage

\section*{Klasa CpuFuzzyComputeMapper} \label{ch:cpu-fuzzycompute-mapper}

Klasa CpuFuzzyComputeMapper (listing \ref{lst:cpu-fuzzycompute-mapper}) wykonuje operacje na zbiorach rozmytych wykorzystując procesor główny.

\begin{lstlisting}[
    language=Java,
    label=lst:cpu-fuzzycompute-mapper,
    caption={Komponent Mapper na procesor główny dla programu FuzzyCompute.}
]
public class CpuFuzzyComputeMapper extends Mapper<NullWritable, BytesWritable, NullWritable, BytesWritable> {

    @Override
    public void map(NullWritable ignored, BytesWritable value, Context context) throws IOException, InterruptedException {
        var bytes = value.getBytes();
        FuzzyUtils.performOps(bytes);
        context.write(NullWritable.get(), new BytesWritable(bytes));
    }
}
\end{lstlisting}
\newpage

\section*{Klasa JCudaFuzzyComputeMapper} \label{ch:jcuda-fuzzycompute-mapper}

Klasa JCudaFuzzyComputeMapper (listing \ref{lst:jcuda-fuzzycompute-mapper}) wykonuje operacje na zbiorach rozmytych wykorzystując procesor graficzny.

\begin{lstlisting}[
    language=Java,
    label=lst:jcuda-fuzzycompute-mapper,
    caption={Komponent Mapper na procesor graficzny dla programu FuzzyCompute.}
]
public class JcudaFuzzyComputeMapper extends Mapper<NullWritable, BytesWritable, NullWritable, BytesWritable> {

    @Override
    public void run(Context context) throws IOException, InterruptedException {
        setup(context);

        try {
            var ctx = JcudaUtils.createCudaContext();
            var bytes = new ByteArrayOutputStream();
            var nRecords = 0L;
            while (context.nextKeyValue()) {
                bytes.writeBytes(context.getCurrentValue().getBytes());
                ++nRecords;
            }

            var output = JcudaUtils.fuzzyPerformOps(bytes.toByteArray(), nRecords);
            FuzzyUtils.forEachRecord(output, nRecords, buf -> HadoopJobUtils.writeByteRecord(context, buf));
            JcudaUtils.freeResources(output);
        } finally {
            cleanup(context);
        }
    }
}
\end{lstlisting}
\newpage

\section*{Klasa CpuFuzzyComputeFunction} \label{ch:cpu-fuzzycompute-function}

Klasa CpuFuzzyComputeFunction (listing \ref{lst:cpu-fuzzycompute-function}) jest odpowiednikiem
klasy CpuFuzzyGenMapper (listting \ref{lst:cpu-fuzzycompute-mapper}) dla platformy Spark.

\begin{lstlisting}[
    language=Java,
    label=lst:cpu-fuzzycompute-function,
    caption={Funkcja na procesor główny dla programu FuzzyCompute.}
]
public class CpuFuzzyComputeFunction {

    public static Iterator<byte[]> call(List<byte[]> records) {
        for (var record : records) {
            FuzzyUtils.performOps(record);
        }
        return records.iterator();
    }
}
\end{lstlisting}
\newpage

\section*{Klasa JCudaFuzzyComputeFunction} \label{ch:jcuda-fuzzycompute-function}

Klasa JCudaFuzzyComputeFunction (listing \ref{lst:jcuda-fuzzycompute-function}) jest odpowiednikiem
klasy JCudaFuzzyComputeMapper (listting \ref{lst:jcuda-fuzzycompute-mapper}) dla platformy Spark.

\begin{lstlisting}[
    language=Java,
    label=lst:jcuda-fuzzycompute-function,
    caption={Funkcja na procesor graficzny dla programu FuzzyCompute.}
]
public class JcudaFuzzyComputeFunction {

    public static Iterator<byte[]> call(List<byte[]> records) throws IOException {
        var ctx = JcudaUtils.createCudaContext();
        var bytes = new ByteArrayOutputStream();
        for (var record : records) {
            bytes.writeBytes(record);
        }

        var output = JcudaUtils.fuzzyPerformOps(bytes.toByteArray(), records.size());
        var results = new ArrayList<byte[]>();
        FuzzyUtils.forEachRecord(output, records.size(), results::add);
        JcudaUtils.freeResources(output);
        return results.iterator();
    }
}
\end{lstlisting}
\newpage

\section*{Klasa CpuFuzzyFilterFunction} \label{ch:cpu-fuzzyfilter-function}

Klasy CpuFuzzyFilterFunction, FuzzyTNorm oraz FuzzyPredicate (listing \ref{lst:cpu-fuzzyfilter-function})
realizują operacje filtrowania na zbiorach rozmytych wykorzystując procesor główny.

\begin{lstlisting}[
    language=Java,
    label=lst:cpu-fuzzyfilter-function,
    caption={Funkcja na procesor główny dla programu FuzzyFilter.}
]
public class FuzzyTNorm implements FilterFunction<Row>, Serializable {

    private final float threshold;
    private final List<FuzzyPredicate> membershipList;

    public FuzzyTNorm(float threshold, List<FuzzyPredicate> membershipList) {
        this.threshold = threshold;
        this.membershipList = membershipList;
    }

    public float getThreshold() {
        return threshold;
    }

    public List<FuzzyPredicate> getMembershipList() {
        return membershipList;
    }

    @Override
    public boolean call(Row value) throws Exception {
        return threshold <= membershipList.stream()
                .mapToDouble(p -> (double) p.getMembership(value))
                .min()
                .orElse(1.0);
    }
}

public class FuzzyPredicate implements Serializable {

    private final float a;
    private final float b;
    private final float c;
    private final float d;
    private final String column;

    public FuzzyPredicate(String expr) {
        String[] parts = expr.split(" ", 5);
        this.a = Float.parseFloat(parts[0]);
        this.b = Float.parseFloat(parts[1]);
        this.c = Float.parseFloat(parts[2]);
        this.d = Float.parseFloat(parts[3]);
        this.column = parts[4];
    }

    public float getA() {
        return a;
    }

    public float getB() {
        return b;
    }

    public float getC() {
        return c;
    }

    public float getD() {
        return d;
    }

    public String getColumn() {
        return column;
    }

    public float getValue(Row row) {
        return Float.parseFloat(row.getString(row.fieldIndex(this.column)));
    }

    public float getMembership(Row row) {
        float value = getValue(row);
        if (value <= this.a || value >= this.d) {
            return 0;  // out of range
        } else if (value >= this.b && value <= this.c) {
            return 1;  // inside plateau
        } else if (value < this.b) { // (val > a && val < b)
            return (value - this.a) / (this.b - this.a); // left triangle
        } else {  // (val > c && val < d)
            return (this.d - value) / (this.d - this.c);  // right triangle
        }
    }
}

public class CpuFuzzyFilterFunction implements FuzzyTNormFilter {

    private final FuzzyTNorm tNorm;

    public CpuFuzzyFilterFunction(FuzzyTNorm tNorm) {
        this.tNorm = tNorm;
    }

    @Override
    public Dataset<Row> apply(Dataset<Row> dataset) {
        return dataset.filter(tNorm);
    }
}
\end{lstlisting}
\newpage

\section*{Klasa JCudaFuzzyFilterFunction} \label{ch:jcuda-fuzzyfilter-function}

Klasa JCudaFuzzyFilterFunction (listing \ref{lst:jcuda-fuzzyfilter-function})
realizuje operacje filtrowania na zbiorach rozmytych wykorzystując procesor graficzny.

\begin{lstlisting}[
    language=Java,
    label=lst:jcuda-fuzzyfilter-function,
    caption={Funkcja na procesor graficzny dla programu FuzzyFilter.}
]
public class JcudaFuzzyFilterFunction implements FuzzyTNormFilter, MapPartitionsFunction<Row, Row> {

    private static final int A = 0;
    private static final int B = 1;
    private static final int C = 2;
    private static final int D = 3;

    private final float threshold;
    private final float[] memberFnParams;
    private final String[] column;

    public JcudaFuzzyFilterFunction(FuzzyTNorm tNorm) {
        var membershipList = tNorm.getMembershipList();
        int nFilters = membershipList.size();
        this.memberFnParams = new float[nFilters * 4];
        this.column = new String[nFilters];
        for (int i = 0; i < nFilters; ++i) {
            fillMembershipFnParams(i, membershipList.get(i));
            this.column[i] = membershipList.get(i).getColumn();
        }

        this.threshold = tNorm.getThreshold();
    }

    @Override
    public Dataset<Row> apply(Dataset<Row> dataset) {
        return dataset.mapPartitions(this, dataset.encoder());
    }

    @Override
    public Iterator<Row> call(Iterator<Row> input) throws Exception {
        var ctx = JcudaUtils.createCudaContext();
        var rows = new ArrayList<Row>();
        input.forEachRemaining(rows::add);

        int nRows = rows.size();
        var memberPtr = JcudaUtils.fuzzyFilterRows(getRowValues(rows), memberFnParams, threshold, nRows, column.length);
        short[] member = new short[nRows];
        cuMemcpyDtoH(Pointer.to(member), memberPtr, (long) nRows * Sizeof.SHORT);

        var result = new ArrayList<Row>();
        for (int i = 0; i < nRows; ++i) {
            if (member[i] == 1) {
                result.add(rows.get(i));
            }
        }
        JcudaUtils.freeResources(memberPtr);
        return result.iterator();
    }

    private float[] getRowValues(List<Row> rows) {
        int nRows = rows.size();
        int nFilters = column.length;
        float[] values = new float[nFilters * nRows];

        for (int i = 0; i < nFilters; ++i) {
            for (int j = 0; j < nRows; ++j) {
                var row = rows.get(j);
                var value = row.getString(row.fieldIndex(column[i]));
                values[i * nRows + j] = Float.parseFloat(value);
            }
        }
        return values;
    }

    private void fillMembershipFnParams(int i, FuzzyPredicate fuzzyPredicate) {
        this.memberFnParams[i * 4 + A] = fuzzyPredicate.getA();
        this.memberFnParams[i * 4 + B] = fuzzyPredicate.getB();
        this.memberFnParams[i * 4 + C] = fuzzyPredicate.getC();
        this.memberFnParams[i * 4 + D] = fuzzyPredicate.getD();
    }
}
\end{lstlisting}
\newpage
 % dokumentacja techniczna
\chapter{Spis skrótów i symboli}

\begin{itemize}
    \item[SoC] -- system na czipie (ang. \english{system on a chip}),
    \item[GPGPU] -- obliczenia ogólnego przeznaczenia na układach graficznych (ang. \english{General-Purpose Compute on Graphics Processing Units}),
    \item[SBC] -- komputer jednopłytkowy (ang. \english{single-board computer}),
    \item[SMP] -- wieloprocesorowość symetryczna (ang. \english{Symmetric Multiprocessing}),
    \item[HDFS] -- rozproszony system plików Hadoop (ang. \english{Hadoop Distributed File System}),
    \item[SIMD] -- jedna instrukcja, wiele danych (ang. \english{Single Instruction, Multiple Data}),
    \item[SIMT] -- jedna instrukcja, wiele wątków (ang. \english{Single Instruction, Multiple Threads}).
\end{itemize} % Spis skrótów i symboli
\chapter{Lista dodatkowych plików}

W systemie do pracy dołączono dodatkowe pliki zawierające:
\begin{itemize}
    \item źródła programu,
    \item skrypty konfiguracyjne i~administracyjne,
    \item zbiory danych użyte w~eksperymentach.
\end{itemize}
 % Lista dodatkowych plików

\listoffigures
\addcontentsline{toc}{chapter}{Spis rysunków}
\listoftables
\addcontentsline{toc}{chapter}{Spis tabel}

\end{appendices}

\end{document}


%% Finis coronat opus.

